\documentclass[12pt]{article}
\usepackage[utf8]{inputenc}
\usepackage{amsfonts}
\usepackage{amsmath}
\usepackage{siunitx}
\usepackage{amssymb}
\usepackage{enumitem}
\usepackage{mathtools}
\usepackage[brazil]{babel}
\usepackage{geometry}
\usepackage{graphicx}
\usepackage{bussproofs}
\usepackage[table]{xcolor}
\usepackage{gensymb}
\graphicspath{{./images}}
\geometry{verbose,a4paper,left=2cm,top=2cm,right=3cm,bottom=3cm}
\title{Lista 4 - Álgebra Linear}
\author{}
\date{}
\linespread{1.5}
\newcommand{\real}{\mathbb{R}}
\newcommand{\lk}{\left\{}
  \newcommand{\rk}{\right\}}
\newcommand{\product}[3]{\displaystyle\prod_{#1}^#2 #3}
\newcommand{\gsum}[3]{\displaystyle\sum_{#1}^#2 #3}
\newcommand{\mytitle}[1]{\textbf{\underline{#1}}}
\begin{document}
\maketitle
\noindent Exercícios transcritos do livro Gilbert Strang - Álgebra Linear e suas aplicações
$\\\\$ 
\textbf{Exercício 1}. Find the eigenvalues and eigenvectors of the matrix $A = \left[ 
  \begin{matrix}
    1 & -1 \\
    2 & 4
  \end{matrix}\right]$. Verify that the trace equals the sum of the eigenvalues, and the determinant equals their product.\\
Para achar os auto-valores, eu preciso encontrar os $\lambda$ tal que $det(A-\lambda I)=0$.\\
$$
\begin{bmatrix}
  1 & -1 \\
  2 & 4
\end{bmatrix}-
\begin{bmatrix}
  \lambda & 0 \\
  0 & \lambda
\end{bmatrix}=
\begin{bmatrix}
  1-\lambda & -1\\
  2 & 4-\lambda
\end{bmatrix}$$
$$
\begin{vmatrix}
  1-\lambda & -1\\
  2 & 4-\lambda
\end{vmatrix}=
4-5\lambda +\lambda^2+2=\lambda^2-5\lambda+6=0$$
$$\lambda \in \{2,3\}$$\\
Agora, para os auvto-vetores tenho que achar o espaço nulo da matriz $A-\lambda I$ para cada um dos autovalores $\Rightarrow$
\begin{itemize}
\item $\lambda_1 = 2$
  \begin{center}
    $\begin{bmatrix}
      -1 & -1 \\
      2 & 2
    \end{bmatrix}\cdot
    \begin{bmatrix}
      x_1 \\
      x_2
    \end{bmatrix}=
    \begin{bmatrix}
      0 \\
      0
    \end{bmatrix}\Rightarrow
    x_1=-x_2\Rightarrow
    \alpha
    \begin{bmatrix}
      1 \\
      -1
    \end{bmatrix}, \alpha \in \real$
  \end{center}
\item $\lambda_2 = 3$
  \begin{center}
    $\begin{bmatrix}
      -2 & -1 \\
      2 & 1
    \end{bmatrix}\cdot
    \begin{bmatrix}
      y_1 \\
      y_2
    \end{bmatrix}=
    \begin{bmatrix}
      0 \\
      0
    \end{bmatrix}\Rightarrow
    -2y_1=y_2\Rightarrow
    \alpha
    \begin{bmatrix}
      1 \\
      -2
    \end{bmatrix}, \alpha \in \real$
  \end{center}
  $T(A)=5=\lambda_1+\lambda_2$ e $det(A)=6=\lambda_1\lambda_2$
\end{itemize}

\noindent\textbf{Exercício 5}. Find the eigenvalues and the eigenvectors of
$$A = \left[\begin{matrix}
    3 & 4 & 2 \\
    0 & 1 & 2 \\
    0 & 0 & 0
  \end{matrix}\right] \text{ and } B = \left[\begin{matrix}
    0 & 0 & 2 \\
    0 & 2 & 0 \\
    2 & 0 & 0
  \end{matrix}\right]$$
Check that $\lambda_1 + \lambda_2 + \lambda_3$ equals the trace and $\lambda_1\lambda_2\lambda_3$ equals the determinant.\\
Primeiro para $A$, nós temos:
\begin{center}
  $(A-\lambda I)=
  \begin{bmatrix}
    3-\lambda & 4 & 2\\
    0 & 1-\lambda & 2\\
    0 & 0 & -\lambda
  \end{bmatrix}\Rightarrow$
\end{center}
\begin{center}
  $\begin{vmatrix}
    3-\lambda & 4 & 2\\
    0 & 1-\lambda & 2\\
    0 & 0 & -\lambda
  \end{vmatrix}=3\lambda(\lambda-1)-\lambda^2(\lambda-1)=
  (3\lambda-\lambda^2)(\lambda-1)=0\Rightarrow
  $
\end{center}
\begin{center}
  $\lambda \in \{0,1,3\}$
\end{center}
Agora que temos os auto-valores, preciso encontrar os auto-vetores:
\begin{itemize}
\item $\lambda_1=0$
  \begin{center}
    $\begin{bmatrix}
      3 & 4 & 2\\
      0 & 1 & 2\\
      0 & 0 & 0
    \end{bmatrix}\cdot
    \begin{bmatrix}
      x_1 \\ x_2 \\ x_3
    \end{bmatrix}=
    \begin{bmatrix}
      0 \\ 0 \\ 0
    \end{bmatrix}\Rightarrow
    \alpha
    \begin{bmatrix}
      1 \\ -1 \\ \frac{1}{2}
    \end{bmatrix}, \alpha \in \real.
    $
  \end{center}
\item $\lambda_2=1$
  \begin{center}
    $\begin{bmatrix}
      2 & 4 & 2\\
      0 & 0 & 2\\
      0 & 0 & -1
    \end{bmatrix}\cdot
    \begin{bmatrix}
      x_1 \\ x_2 \\ x_3
    \end{bmatrix}=
    \begin{bmatrix}
      0 \\ 0 \\ 0
    \end{bmatrix}\Rightarrow
    \alpha
    \begin{bmatrix}
      -2 \\ 1 \\ 0
    \end{bmatrix}, \alpha \in \real.
    $
  \end{center}
\item $\lambda_3=3$
  \begin{center}
    $\begin{bmatrix}
      0 & 4 & 2\\
      0 & -2 & 2\\
      0 & 0 & -3
    \end{bmatrix}\cdot
    \begin{bmatrix}
      x_1 \\ x_2 \\ x_3
    \end{bmatrix}=
    \begin{bmatrix}
      0 \\ 0 \\ 0
    \end{bmatrix}\Rightarrow
    \alpha
    \begin{bmatrix}
      1 \\ 0 \\ 0
    \end{bmatrix}, \alpha \in \real.
    $
  \end{center}
  Agora para $B$, nós temos:
  \begin{center}
    $(B-\lambda I)=
    \begin{bmatrix}
      -\lambda & 0 & 2\\
      0 & 2-\lambda & 0\\
      2 & 0 & -\lambda
    \end{bmatrix}\Rightarrow$
  \end{center}
  \begin{center}
    $\begin{vmatrix}
      -\lambda & 0 & 2\\
      0 & 2-\lambda & 0\\
      2 & 0 & -\lambda
    \end{vmatrix}=(\lambda+2)(\lambda-2)(2-\lambda)=0\Rightarrow$
  \end{center}
  \begin{center}
    $\lambda \in \{-2,2\}$
  \end{center}
  Agora que temos os auto-valores, preciso encontrar os auto-vetores:
\item $\lambda_1=2$
  \begin{center}
    $\begin{bmatrix}
      -2 & 0 & 2\\
      0 & 0 & 0\\
      2 & 0 & -2
    \end{bmatrix}\cdot
    \begin{bmatrix}
      x_1 \\ x_2 \\ x_3
    \end{bmatrix}=
    \begin{bmatrix}
      0 \\ 0 \\ 0
    \end{bmatrix}\Rightarrow
    \alpha
    \begin{bmatrix}
      1 \\ \beta \\ 1
    \end{bmatrix}, \alpha, \beta \in \real.
    $
  \end{center}
\item $\lambda_2=-2$
  \begin{center}
    $\begin{bmatrix}
      2 & 0 & 2\\
      0 & 4 & 0\\
      2 & 0 & 2
    \end{bmatrix}\cdot
    \begin{bmatrix}
      x_1 \\ x_2 \\ x_3
    \end{bmatrix}=
    \begin{bmatrix}
      0 \\ 0 \\ 0
    \end{bmatrix}\Rightarrow
    \alpha
    \begin{bmatrix}
      1 \\ 0 \\ -1
    \end{bmatrix}, \alpha \in \real.
    $
  \end{center}
\end{itemize}
\textbf{Exercício 8}. Show that the determinant equals the product of the eigenvalues by imagining that the characteristic polynomial is factored into
\begin{center}
  $det(A-\lambda I) = (\lambda_1-\lambda)(\lambda_2-\lambda)\dots(\lambda_n-\lambda)$ \quad(16),
\end{center}
and making a clever choice of $\lambda$.\\
Atribuindo o valor de 0 a $\lambda$, temos que:\\
\begin{center}
  $det(A-0I)=\product{i=1}{n}{\lambda_n-0}\Rightarrow det(A)=\product{i=1}{n}{\lambda_n}$
\end{center}
\textbf{Exercício 9}. Show that the trace equals the sum of the eigenvalues, in two steps. First, find the coefficient of $(-\lambda)^{n-1}$ on the right side of equation (16). Next, find all the terms in
\begin{center}
  $det(A-\lambda I) = det$
  $\begin{bmatrix}
    a_{11}-\lambda & a_{12} & \cdots & a_{1n}\\
    a_{21} & a_{22}-\lambda & \cdots & a_{2n}\\
    \vdots & \vdots & & \vdots\\
    a_{n1} & a_{n2} & \cdots & a_{nn}-\lambda\\
  \end{bmatrix}$
\end{center}
that involve $(-\lambda)^{n-1}$. They all come from the main diagonal! Find that coefficient of $(-\lambda^{n-1})$ and compare.\\
De $(16)$, podemos ver que:
\begin{center}
  $det(A-\lambda I)=\product{i=1}{n}{(\lambda_n-\lambda)}$,
\end{center}
ou seja, temos que o coeficiente de $(-\lambda)^{n-1}$ será $\gsum{i=1}{n}{\lambda_n}$.\\
Do que foi dado no enunciado podemos ver que ao calcular o determinante de $A-\lambda I$, o coeficiente do termo $(-\lambda)^{n-1}$ será $\gsum{i=1}{n}{a_{ii}}$.\\
Desse modo, pode-se ver que ao calcularmos o determinante de $A-\lambda I$ de dois modos diferentes, chegamos em dois coeficientes para o termo $(-\lambda)^{n-1}$, logo eles devem ser iguais, pois o determinante é único $\Rightarrow$\\
\begin{center}
  $\Rightarrow T(A)=\gsum{i=1}{n}{a_{ii}}=\gsum{i=1}{n}{\lambda_i}$
\end{center}
\newpage
\noindent\textbf{Exercício 11}. \textit{\textbf{The eigenvalues of} A \textbf{equal the eigenvalues of}} $A^T$. This is because $det(A-\lambda I)$ equals $det(A^T-\lambda I)$. That is true because \_\_\_\_. Show by an example that the eigenvectors of $A$ and $A^T$ are not the same.\\
Sabemos que $det(M)=det(M^T)$ pra qualquer matriz $M$. Podemos então dizer que $det(A-\lambda I)=det\left[(A-\lambda I)^T \right]$. Porém, $(A-\lambda I)^T=A^T-\lambda I^T=A^T-\lambda I$. Logo, temos que $det(A-\lambda I)=det\left[(A-\lambda I)^T\right]=det(A^T-\lambda I)$.
\begin{center}
  $
  \begin{bmatrix}
    1 & 3 \\
    4 & 5
  \end{bmatrix}\Rightarrow
  \begin{vmatrix}
    1-\lambda & 3 \\
    4 & 5-\lambda
  \end{vmatrix}\Rightarrow
  \lambda^2-6\lambda-7=0\Rightarrow \lambda \in \{-1,7\}
  $
\end{center}
\begin{itemize}
\item $\lambda = -1$
  \begin{center}
    $
    \begin{bmatrix}
      2 & 3 \\
      4 & 6 
    \end{bmatrix}\cdot
    \begin{bmatrix}
      x_1 \\ x_2
    \end{bmatrix} =
    \begin{bmatrix}
      0 \\ 0
    \end{bmatrix}\Rightarrow
    \alpha
    \begin{bmatrix}
      3 \\ -2
    \end{bmatrix}
    $
  \end{center}
\item $\lambda = 7$
  \begin{center}
    $
    \begin{bmatrix}
      -6 & 3 \\
      4 & -2
    \end{bmatrix}\cdot
    \begin{bmatrix}
      x_1 \\ x_2
    \end{bmatrix}=
    \begin{bmatrix}
      0 \\ 0
    \end{bmatrix}\Rightarrow \alpha
    \begin{bmatrix}
      1 \\ 2
    \end{bmatrix}
    $
  \end{center}
  \begin{center}
    $
    \begin{bmatrix}
      1 & 4 \\
      3 & 5
    \end{bmatrix}\Rightarrow
    \begin{vmatrix}
      1-\lambda & 3 \\
      4 & 5-\lambda
    \end{vmatrix}\Rightarrow
    \lambda^2-6\lambda-7=0\Rightarrow \lambda \in \{-1,7\}
    $
  \end{center}
\end{itemize}
\begin{itemize}
\item $\lambda = -1$
  \begin{center}
    $
    \begin{bmatrix}
      2 & 4 \\
      3 & 6 
    \end{bmatrix}\cdot
    \begin{bmatrix}
      x_1 \\ x_2
    \end{bmatrix} =
    \begin{bmatrix}
      0 \\ 0
    \end{bmatrix}\Rightarrow
    \alpha
    \begin{bmatrix}
      -2 \\ 1
    \end{bmatrix}
    $
  \end{center}
\item $\lambda = 7$
  \begin{center}
    $
    \begin{bmatrix}
      -6 & 4 \\
      3 & -2
    \end{bmatrix}\cdot
    \begin{bmatrix}
      x_1 \\ x_2
    \end{bmatrix}=
    \begin{bmatrix}
      0 \\ 0
    \end{bmatrix}\Rightarrow \alpha
    \begin{bmatrix}
      2 \\ 3
    \end{bmatrix}
    $
  \end{center}
\end{itemize}
Logo os eigenvectors são diferentes.\\
\newpage
\textbf{Exercício 12}.
Find the eigenvalues and eigenvectors of 
$$A = \left[\begin{matrix}
    3 & 4 \\
    4 & -3
  \end{matrix}\right]\quad \quad 
A = \left[\begin{matrix}
    a & b \\
    b & a
  \end{matrix}\right]
$$
Primeiro acharei os eigenvalues:\\
\begin{center}
  $\begin{vmatrix}
    3-\lambda & 4 \\
    4 & -3-\lambda
  \end{vmatrix}=\lambda^2-25=0\Rightarrow\lambda=\pm 5
  \begin{vmatrix}
    a-\lambda & b \\
    b & a-\lambda
  \end{vmatrix}=\lambda^2-2a\lambda+a^2-b^2\Rightarrow \lambda=a\pm b
  $
\end{center}
Agora precisamos achar os eigenvectors:\\
Primeira Matriz:
\begin{itemize}
\item 5:
  \begin{center}
    $\begin{bmatrix}
      -2 & 4 \\
      4 & -8
    \end{bmatrix}\cdot
    \begin{bmatrix}
      x_1 \\ x_2
    \end{bmatrix}=
    \begin{bmatrix}
      0 \\ 0
    \end{bmatrix}\Rightarrow \alpha
    \begin{bmatrix}
      2 \\ 1
    \end{bmatrix}, \alpha \in \real
    $
  \end{center}
\item -5:
  \begin{center}
    $
    \begin{bmatrix}
      8 & 4 \\
      4 & 2
    \end{bmatrix}\cdot
    \begin{bmatrix}
      x_1 \\ x_2
    \end{bmatrix}=
    \begin{bmatrix}
      0 \\ 0
    \end{bmatrix}\Rightarrow \alpha
    \begin{bmatrix}
      1 \\ 2
    \end{bmatrix}, \alpha \in \real
    $
  \end{center}
\end{itemize}
Segunda Matriz:
\begin{itemize}
\item $a+b$
  \begin{center}
    $
    \begin{bmatrix}
      -b & b \\
      b & -b
    \end{bmatrix}\cdot
    \begin{bmatrix}
      x_1 \\ x_2
    \end{bmatrix}=
    \begin{bmatrix}
      0 \\ 0
    \end{bmatrix}\Rightarrow \alpha
    \begin{bmatrix}
      1 \\ 1
    \end{bmatrix}, \alpha \in \real
    $
  \end{center}
\item $a-b$
  \begin{center}
    $
    \begin{bmatrix}
      b & b \\
      b & b
    \end{bmatrix}\cdot
    \begin{bmatrix}
      x_1 \\ x_2
    \end{bmatrix}=
    \begin{bmatrix}
      0 \\ 0
    \end{bmatrix}\Rightarrow \alpha
    \begin{bmatrix}
      1 \\ -1
    \end{bmatrix}, \alpha \in \real
    $
  \end{center}
\end{itemize}
\newpage
\textbf{Exercício 21}. Compute the eigenvalues and eigenvectors of $A$ and $A^{-1}$:
$$A = 
\left[
  \begin{matrix}
    0 & 2 \\
    2 & 3
  \end{matrix}    
\right]\text{ and } 
A^{-1} = 
\left[
  \begin{matrix}
    -\frac{3}{4} & \frac{1}{2} \\
    \frac{1}{2} & 0
  \end{matrix}    
\right]
$$
$A^{-1}$ has the \_\_\_\_\_ eigenvectors as $A$. When $A$ has eigenvalues $\lambda_1$ and $\lambda_2$, its inverse has eigenvalues \_\_\_\_\_.\\
Primeiro vamos calcular os eigenvalues:\\
A:
\begin{center}
  $
  \begin{vmatrix}
    -\lambda & 2 \\
    2 & 3-\lambda
  \end{vmatrix}\Rightarrow \lambda \in \{-1,4\}
  $
\end{center}
B:
\begin{center}
  $
  \begin{vmatrix}
    -\frac{3}{4}-\lambda & \frac{1}{2} \\
    \frac{1}{2} & -\lambda
  \end{vmatrix}\Rightarrow \lambda \in \{-1,\frac{1}{4}\}
  $
\end{center}
Agora vamos achar os eigenvectors:\\
Primeira Matriz:
\begin{itemize}
\item -1:
  \begin{center}
    $
    \begin{bmatrix}
      1 & 2 \\
      2 & 4
    \end{bmatrix}\cdot
    \begin{bmatrix}
      x_1 \\ x_2
    \end{bmatrix}=
    \begin{bmatrix}
      0 \\ 0
    \end{bmatrix}\Rightarrow \alpha
    \begin{bmatrix}
      -2 \\ 1
    \end{bmatrix}, \alpha \in \real
    $
  \end{center}
\item -4:
  \begin{center}
    $
    \begin{bmatrix}
      4 & 2 \\
      2 & -1
    \end{bmatrix}\cdot
    \begin{bmatrix}
      x_1 \\ x_2
    \end{bmatrix}=
    \begin{bmatrix}
      0 \\ 0
    \end{bmatrix}\Rightarrow \alpha
    \begin{bmatrix}
      1 \\ 2
    \end{bmatrix}, \alpha \in \real
    $
  \end{center}
\end{itemize}
Segunda Matriz:
\begin{itemize}
\item -1:
  \begin{center}
    $
    \begin{bmatrix}
      \frac{1}{4} & \frac{1}{2} \\
      \frac{1}{2} & 1
    \end{bmatrix}\cdot
    \begin{bmatrix}
      x_1 \\ x_2
    \end{bmatrix}=
    \begin{bmatrix}
      0 \\ 0
    \end{bmatrix}\Rightarrow \alpha
    \begin{bmatrix}
      -2 \\ 1
    \end{bmatrix}, \alpha \in \real
    $
  \end{center}
\item -4:
  \begin{center}
    $
    \begin{bmatrix}
      -1 & \frac{1}{2} \\
      \frac{1}{2} & \frac{1}{4}
    \end{bmatrix}\cdot
    \begin{bmatrix}
      x_1 \\ x_2
    \end{bmatrix}=
    \begin{bmatrix}
      0 \\ 0
    \end{bmatrix}\Rightarrow \alpha
    \begin{bmatrix}
      1 \\ 2
    \end{bmatrix}, \alpha \in \real
    $
  \end{center}
\end{itemize}
\mytitle{Conclusão}: Os eigenvectors de $A$ e $A^{-1}$ são iguais e os eigenvalues de $A^{-1}$ são os inversos dos eigenvalues de $A$.\\
\textbf{Exercício 39}. Challenge problem: \textit{Is there a real 2 by 2 matrix} (other than \textit{I}) \textit{with} $A^3 = I$? Its eigenvalues must satisfy $\lambda^3 = I$. They can be $e^{2\pi i/3}$ and $e^{-2\pi i/3}$. What trace and determinant would this give? Contruct $A$.\\
A matriz $A$ seria a matriz de rotação de $120\degree$. Como seus eigenvalues são $e^{2\pi i/3}$ e $e^{-2\pi i/3}$, o traço será a soma desses 2, ou seja $e^{2\pi i/3}+e^{-2\pi i/3}$ e o determinante seria o produto, ou seja $e^{2\pi i/3}e^{-2\pi i/3}=e^0=1$\\
\textbf{Exercício 40}. There are six 3 by 3 permutation matrices $P$. What numbers can be the $determinants$ of $P$? What numbers can be $pivots$? What numbers can be the $trace$ of $P$? What \textit{four numbers} can be eigenvalues of $P$?\\
\begin{center}
  $
  \begin{bmatrix}
    1 & 0 & 0 \\
    0 & 1 & 0 \\
    0 & 0 & 1
  \end{bmatrix}
  \begin{bmatrix}
    1 & 0 & 0 \\
    0 & 0 & 1 \\
    0 & 1 & 0
  \end{bmatrix}
  \begin{bmatrix}
    0 & 1 & 0 \\
    1 & 0 & 0 \\
    0 & 0 & 1
  \end{bmatrix}
  \begin{bmatrix}
    0 & 1 & 0 \\
    0 & 0 & 1 \\
    1 & 0 & 0
  \end{bmatrix}
  \begin{bmatrix}
    0 & 0 & 1 \\
    1 & 0 & 0 \\
    0 & 1 & 0
  \end{bmatrix}
  \begin{bmatrix}
    0 & 0 & 1 \\
    0 & 1 & 0 \\
    1 & 0 & 0
  \end{bmatrix}
  $
\end{center}
O determinante pode ser $0$ ou $-1$.\\
O único número que pode ser o pivô é o $1$.\\
O traço pode ser $3$, $1$ e $0$.\\\
Fazendo as contas que fizemos até agora na lista toda (ficou no meu rascunho), os eigenvalues podem ser $1$, $-1$, $i$, $-i$. 
\end{document}