\documentclass[12pt, oneside]{article}
\usepackage[utf8]{inputenc}
\usepackage{amsfonts}
\usepackage{amsmath}
\usepackage{amssymb}
\usepackage{amsthm}
\usepackage{enumitem}
\usepackage{mathtools}
\usepackage{bussproofs}
\usepackage{calrsfs}
\usepackage{listings}
\usepackage{xcolor}
\usepackage[brazil]{babel}
\usepackage{geometry}
\usepackage{graphicx}
\usepackage[hidelinks]{hyperref}
\newcommand{\mytitle}[1]{\textbf{\underline{#1}}}

  
\geometry{verbose,a4paper,left=2cm,top=2cm,right=2cm,bottom=1cm}
\title{Questionário para Salvar minha Vida}
\author{Lourenço Henrique Moinheiro Martins Sborz Bogo - NUSP 11208005 - BCC}
\date{28/11/2019}
\linespread{1.5}

\begin{document}
\maketitle
\begin{enumerate}
\item Sim, li. A obra é interessante e me senti imerso e representado de certa maneira nela, pois já tive várias discussões sobre o assunto tratado na peça com meus amigos. Achei as personagens bem desenvolvidas para uma peça tão pequena e todos com personalidades fortes e conflitantes, dando um clima interessante e fervoroso aos diálogos.
\item
  \begin{enumerate}
  \item Vários personagens do poema são referenciados no filme. O George Clooney, por exemplo, faz o papel de Everett \mytitle{Ulysses} McGill, uma referência ao protagonista da odisseia, Ulysses. John Goodman é uma referência ao Ciclope, as lavadeiras são uma referência às sereias e o objetivo principal do protagonista é reencontrar sua esposa Penny (referência à esposa de Ulysses penélope), igual ao protagonista do poema.
  \item Já conhecia a peça, o que me ajudou a entender o filme. Dito isso, foi uma experiência tranquila, já que não tive que pensar muito sobre a história pois já a conhecia. Sendo honesto, preferi a adaptação que eu li, provavelmente por causa do momento em que li, há alguns anos. Eu sempre valorizei muito a história e a mitologia grega, então conhecer a peça de uma maneira mais formal me é mais atrativo, por outro lado, assistir o filme foi mais divertido e descontraído.
  \end{enumerate}
\item Duas.
\item Considerando que o objetivo da matéria era nos apresentar técnicas de teatro para que conseguíssemos nos soltar melhor perto de outras pessoas e, também, para melhorar nossa leitura e comunicação, sinto que evolui muito nesse aspecto e que participei bastante das aulas. Sempre tentei o meu melhor para aprender e sinto que estou melhor nos quesitos mencionados. Além disso, a matéria me despertou um interesse por obras e escritores diversos e, também, pelas minhas próprias emoções, as quais agora entendo muito melhor. Acho, então, que mereço uma nota boa, principalmente por ter participado bastante das aulas e sempre tentado o meu melhor para interpretar os papéis com o máximo de emoção possível.
\end{enumerate}
\end{document}