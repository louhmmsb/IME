\documentclass[12pt, oneside]{article}
\usepackage[utf8]{inputenc}
\usepackage{amsfonts}
\usepackage{amsmath}
\usepackage{amssymb}
\usepackage{amsthm}
\usepackage{enumitem}
\usepackage{mathtools}
\usepackage{bussproofs}
\usepackage{calrsfs}
\usepackage{listings}
\usepackage{xcolor}
\usepackage[brazil]{babel}
\usepackage{geometry}
\usepackage{graphicx}
\usepackage[hidelinks]{hyperref}
\newcommand{\mytitle}[1]{\textbf{\underline{#1}}}

  
\geometry{verbose,a4paper,left=2cm,top=2cm,right=2cm,bottom=1cm}
\title{Vamos falar de Hamlet}
\author{Lourenço Henrique Moinheiro Martins Sborz Bogo - NUSP 11208005 - BCC}
\date{28/11/2019}
\linespread{1.5}

\begin{document}
\maketitle
\begin{enumerate}
\item Sim, a maneira com que a loucura do Hamlet é evidenciada para o leitor, me pareceu extremamente interessante e isso me atraiu.
\item Não, não vi qualquer tipo de semelhança entre a minha pessoa e ele. Eu diria que todo o ser humano tem sua própria loucura e não se consegue ver isso em outros, logo não consegui reconhecer a minha no Hamlet.
\item
  \begin{itemize}
  \item Horácio: A maneira com que ele se relaciona com Hamlet é intrigante (Extremamente leal).
  \item Rainha: Por mais que eu não tenha gostado dela, devo admitir que a maneira com que sua relação com Hamlet evolui e transforma durante a peça é interessante.
  \item Coveiro: Me identifiquei com a maneira dele de pensar, ou seja, levando a morte como algo natural com a qual convivemos todos os dias.
  \item Laertes: Me interessei pela maneira com que ele desiste de sua honra (ao certar Hamlet despreparado), para vingar a morte de sua irmã. Ele coloca sua vingança acima da própria honra, mostrando sua dedicação.
  \end{itemize}
\item Acho que no caso dessa peça em específico não. Shakespeare escrevia de uma maneira tão cativante que fez imagino que as pessoas deveriam correr atrás do vocbulário para conseguir compreender a peça. No meu caso posso afirmar que não, não é suficiente.
\item O que torna uma obra clássica na minha opinião é, além da sua qualidade (óbvio), o quanto ela pode ser renovada sem perder a sua essência. Isso pode ser visto no Hamlet com a quantidade de versões da peça que existem. Não acho que exista uma grande utilidade, é uma questão de entretenimento e cultura. Sem a menor dúvida, vale a pena preservar esse conhecimento sim pois, como já foi dito, é uma questão de cultura e de entretenimento de qualidade.
\item Achei a metodologia boa, pois foi aos poucos introduzindo maneiras diferentes de ver a obra, despertando cada vez mais o interesse de entender as relações interpessoas das personagens. Acho que a mensagem do autor nunca fica clara, independente da peça. O que ele escreveu nunca será igual ao que entendemos da peça e esse é um dos motivos de ser tão interessante. O texto se tornou mais rico de significados a cada pedaço que me era introduzido, pois eu conseguia entender cada vez mais as intenções e emoções das personagens. Conhecer o momento histórico ajuda de certa maneira, mas acho que a obra é interessante independente disso.
\end{enumerate}
\end{document}