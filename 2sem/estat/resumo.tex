\documentclass[12pt, oneside]{article}
\usepackage[utf8]{inputenc}
\usepackage{amsfonts}
\usepackage{amsmath}
\usepackage{amssymb}
\usepackage{amsthm}
\usepackage{enumitem}
\usepackage{mathtools}
\usepackage{bussproofs}
\usepackage{calrsfs}
\usepackage{listings}
\usepackage{xcolor}
\usepackage[brazil]{babel}
\usepackage{geometry}
\usepackage{graphicx}
\usepackage[hidelinks]{hyperref}

\newcommand{\mytitle}[1]{\textbf{\underline{#1}}}
\newcommand{\product}[3]{\displaystyle\prod_{#1}^#2 #3}
\newcommand{\gsum}[3]{\displaystyle\sum_{#1}^#2 #3}
  
\geometry{verbose,a4paper,left=2cm,top=2cm,right=2cm,bottom=1cm}
\title{Cola $P_2$}
\author{Lourenço Henrique Moinheiro Martins Sborz Bogo - NUSP 11208005}
%\date{12/11/2019}
\linespread{1}

\begin{document}
\maketitle
\mytitle{Testes de Hipótese}:\\
\begin{description}
\item[Populações Independentes]:
  \begin{enumerate}
  \item Variâncias conhecidas: Nesse caso é a aplicação direta das fórmulas assumindo que as populações são normais:\\
    $\bar{X_1}-\bar{X_2}\sim N(\mu_1-\mu_2,\frac{\sigma_1^2}{n_1}+\frac{\sigma_2^2}{n_2})\Rightarrow \alpha = P(Rejeitar \; H_0|H_0 \; \acute{e} \: verdadeira)\\ = P(\bar{X_1}-\bar{X_2} \in Regi\tilde{a}o\; Cr\acute{i}tica)$\\
  \item Variâncias desconhecidas: Vamos testar as variâncias para saber se elas são iguais ou não, assmindo que as populações são normais:\\
    $H_0: \sigma_1^2 = \sigma_2^2 \quad{H_1}: \sigma_1^2\neq\sigma_2^2$\\
    $F=\frac{S_1^2}{S_2^2}$ Parâmetros $(n_1-1,n_2-1)$\\
    $\alpha = 5\% = P(\frac{S_1^2}{S_2^2}<l)+P(\frac{S_1^2}{S_2^2}>L)$\\
    Caso a fração esteja no intervalo $[l,L]$ não rejeitamos $H_0$.\\
    Agora separamos o problema em dois casos:
    \begin{itemize}
    \item No teste das variâncias nós não rejeitamos $H_0\Rightarrow\\ \Rightarrow\frac{\bar{X_1}-\bar{X_2}}{\sqrt{S_P^2(\frac{1}{n_1}+\frac{1}{n_2})}}\sim t(n_1+n_2-2),\quad{} S_p^2=(\frac{(n_1-1)S_1^2+(n_2-1)S_2^2}{n_1+n_2-2})$
    \item Caso tenhamos rejeitado $H_0$ no teste das variâncias $\Rightarrow \frac{\bar{X_1}-\bar{X_2}}{\sqrt{\frac{S_1^2}{n_1}+\frac{S_2^2}{n_2}}}\sim t(G) \quad{} \\G=\frac{(A+B)^2}{\frac{A^2}{n_1-1}+\frac{B^2}{n_2-1}} \quad{} A=\frac{S_1^2}{n_1} \; B=\frac{S_2^2}{n_2}$
    \end{itemize}
  \end{enumerate}
\item[Populações Dependentes]: Aqui vamos definir uma variável nova $D$, tal que $D=A-B$ e vamos supor que ela é normal.
\end{description}
\mytitle{PS}: $S_x^2=\frac{\gsum{i=0}{n}{(x_i)^2}-n\bar{X}^2}{n-1}$\\
\mytitle{Intervalo de Confiança}:\\
\begin{itemize}
\item $Z=\frac{\bar{X}-\mu_x}{\frac{\sigma}{\sqrt{n}}}\sim N(0,1)$
\item $Z=\frac{\bar{p_x}-\mu_x}{\frac{\sigma}{\sqrt{n}}}\sim N(0,1)$ Nesse caso podemos dizer que $\sigma=0.5$, pois é o pior caso.
\item $\frac{(n-1)S^2}{\sigma^2}\sim \chi^2$ Intervalo de confiança para $\sigma^2$.
\item Intervalo de confiança para a razão da porporção das variâncias vai ser dado pela F.
\end{itemize}
\end{document}
