% Created 2019-12-03 ter 00:50
% Intended LaTeX compiler: pdflatex
\documentclass[11pt]{article}
\usepackage[utf8]{inputenc}
\usepackage[T1]{fontenc}
\usepackage{graphicx}
\usepackage{grffile}
\usepackage{longtable}
\usepackage{wrapfig}
\usepackage{rotating}
\usepackage[normalem]{ulem}
\usepackage{amsmath}
\usepackage{textcomp}
\usepackage{amssymb}
\usepackage{capt-of}
\usepackage{hyperref}
\author{miguel}
\date{\today}
\title{}
\hypersetup{
 pdfauthor={miguel},
 pdftitle={},
 pdfkeywords={},
 pdfsubject={},
 pdfcreator={Emacs 26.3 (Org mode 9.2.6)}, 
 pdflang={English}}
\begin{document}

%\tableofcontents

\begin{enumerate}
\item Responda verdadeiro ou falso e justifique:
\begin{enumerate}
\item Não se deve normalmente colocar funções em arquivos de cabeçalhos (.h) em programas  escritos em (.c).\\
\textbf{Resposta:} Verdadeiro. O cabeçalho serve basicamente para interface e só deve conter declarações.
\item Autômatos finitos servem apenas para construção de analisadores léxicos.\\
\textbf{Resposta:} Falso. São modelos genéricos, embora não sejam universais.
\item A saída padrão é o terminal onde o programa é executado.\\
\textbf{Resposta:} Falso. A saída padrão é o \uline{Stream} de saída do processo. A associação com o terminal é apenas uma das possibilidades.
\item No Unix o significado de \textbf{executável} depende do tipo de cada arquivo.\\
\textbf{Resposta:} Verdadeiro. Um script é texto e deve ser interpretado por um outro programa. Um arquivo ELF pode ou não ser executável, dependendo da arquitetura para o qual foi gerado
\end{enumerate}
\item int recur (int a, int b).\\
\textbf{Resposta:} O que deveria ser feito é calcular o tamanho do frame
\item Responda e justifique:
\begin{enumerate}
\item Por que a chamada para o Kernel só pode ser feita com \emph{syscall} ou \emph{int 0x80}?\\
\textbf{Resposta:} O modo Kernel é protegido. Assim precisamos de uma instrução especial.
\item Por que o protótipo de funções é importatnte para o Compilador?\\
\textbf{Resposta:} Para que o compilador consiga identificar o tipo de retorno e os argumentos.
\item Por que o getchar devolve um int e não um char?\\
\textbf{Resposta:} Para retornar o código para EOF.
\item Impacto no \uline{make} ao \emph{atrasar} o relógio.\\
\textbf{Resposta:} Os arquivos alterados ficarão com uma data no passado e o make poderá não identifcar a necessidade de reconstrução. Outro impacto é nos alvos gerados, que ficarão com data antiga, e serão refeitos sem necessidade na vez seguinte.
\item Impacto no \uline{make} ao \emph{adiantar} o relógio:\\
\textbf{Resposta:} Nenhum. É como se cada operação demorasse mais tempo do que o normal.
\end{enumerate}
\end{enumerate}
\end{document}