\documentclass[12pt]{article}
\usepackage[utf8]{inputenc}
\usepackage{amsfonts}
\usepackage{amsmath}
\usepackage{siunitx}
\usepackage{amssymb}
\usepackage{enumitem}
\usepackage{mathtools}
\usepackage[brazil]{babel}
\usepackage{geometry}
\usepackage{graphicx}
\usepackage{bussproofs}
\usepackage[table]{xcolor}
\usepackage{gensymb}
\graphicspath{{./images}}
\geometry{verbose,a4paper,left=2cm,top=2cm,right=3cm,bottom=3cm}
\title{README só que bonito}
\author{Leonardo, Lourenço, Miguel}
\date{}
\linespread{1.5}
\newcommand{\real}{\mathbb{R}}
\newcommand{\product}[3]{\displaystyle\prod_{#1}^#2 #3}
\newcommand{\gsum}[3]{\displaystyle\sum_{#1}^#2 #3}
\newcommand{\mytitle}[1]{\textbf{\underline{#1}}}
\begin{document}
\maketitle
\noindent\mytitle{Preparando o Programa}:\\
Para iniciar o programa rode o comando "make install" e rode o jogo usando ./spaceWars.bin 1". Caso queira compartilhar o programa use o comando "make tar" que criará um tar com tudo que é necessário para rodar o programa.\\
\mytitle{OBS}: As imagens demoram um pouco para serem criadas, então seja paciente :D!\\
\mytitle{Controles}: Para controlar as naves use:
\begin{description}
\item[Nave 1]:
  \begin{itemize}
  \item $\uparrow$ para acelerar
  \item $\downarrow$ para atirar
  \item $\leftarrow$ e $\rightarrow$ para rotacionar a nave
  \end{itemize}
\item[Nave 2]:
  \begin{itemize}
  \item $W$ para acelerar
  \item $S$ para atirar
  \item $A$ e $D$ para rotacionar a nave
  \end{itemize}
\end{description}
Este projeto foi feito por:
\begin{itemize}
\item Leonardo Costa Santos - 10783142
\item Lourenço Henrique Moinheiro Martins Sborz Bogo - 11208005
\item Miguel de Mello Carpi - 11208502
\end{itemize}
\end{document}