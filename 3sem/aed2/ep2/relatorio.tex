\documentclass[12pt]{article}
\usepackage[utf8]{inputenc}
\usepackage{amsfonts}
\usepackage{amsmath}
\usepackage{siunitx}
\usepackage{amssymb}
\usepackage{enumitem}
\usepackage{mathtools}
\usepackage[brazil]{babel}
\usepackage{geometry}
\usepackage{graphicx}
\usepackage{bussproofs}
\usepackage[table]{xcolor}
\usepackage{gensymb}
\usepackage{hyperref}
\graphicspath{{./images}}
\geometry{verbose,a3paper,left=3cm,top=2cm,right=3cm,bottom=3cm}
\title{Algoritmos e Estruturas de Dados 2 - EP2}
\author{Lourenço Henrique Moinheiro Martins Sborz Bogo}
\date{}
\linespread{1.5}
\newcommand{\real}{\mathbb{R}}
\newcommand{\product}[3]{\displaystyle\prod_{#1}^#2 #3}
\newcommand{\gsum}[3]{\displaystyle\sum_{#1}^#2 #3}
\newcommand{\mytitle}[1]{\textbf{\underline{#1}}}
\newcommand{\ring}[1]{\langle #1 \rangle}
\newcommand{\code}[1]{\mbox{\texttt{#1}}}
\usepackage{listings}
\usepackage{color}

\definecolor{dkgreen}{rgb}{0,0.6,0}
\definecolor{gray}{rgb}{0.5,0.5,0.5}
\definecolor{mauve}{rgb}{0.58,0,0.82}

\lstset{frame=tb,
  language=C++,
  aboveskip=3mm,
  belowskip=3mm,
  showstringspaces=false,
  columns=flexible,
  basicstyle={\small\ttfamily},
  numbers=none,
  numberstyle=\tiny\color{gray},
  keywordstyle=\color{blue},
  commentstyle=\color{dkgreen},
  stringstyle=\color{mauve},
  breaklines=true,
  breakatwhitespace=true,
  tabsize=4
}

\begin{document}

\maketitle

\section{Escolhas de Implementação}

Primeiro problema que encontrei ao começar a implementar o EP foi: como eu saberia em que vértice uma certa palavra está, de maneira rápida? Para resolver isso, eu decidi usar uma HashTable, onde as chaves são as strings que contêm as palavras e os valores são os nodos nos quais elas estão.

Para implementar a HashTable, decidi fazer minha própria biblioteca de lista ligada, que transformei em uma biblioteca de fila para que pudesse ser usada na BFS também.

Depois disso, implentar quase todas as funções do EP foi muito simples, com exceção das \texttt{emCiclo}, que demorei um pouco para pensar como fazer. Decidi implementar quase tudo usando DFS, menos o cálculo da distância onde optei por usar BFS.

Tive que fazer duas funções DFS auxiliares:

\begin{description}
\item[\texttt{int dfs(int)}] Roda uma dfs para o nodo passado como parâmetro e devolve o tamanho de sua componente.
  
\item[\texttt{void dfs(int, int, int, bool, int)}] Essa dfs foi a parte mais complicada de implementar do EP. Caso sejam passados só os 3 primeiros parâmetros para ela, ela irá procurar se existe ou não um ciclo que contém o primeiro parâmetro. Caso os 5 parâmetros sejam passados, a função irá fazer quase a mesma coisa, com a condição de que ao invés de buscar um ciclo no primeiro parâmetro, ela irá buscar um ciclo no primeiro parâmetro, que contenha o último parâmetro.
  
\end{description}

Além dessas funções, implementei mais algumas auxliares para descobrir a maior componente, a menor componente e o tamanho médio da componente, que são calculadas junto com o número de componentes, ou seja, quando um desses dados é pedido, todos são calculados para ganhar tempo.

Para achar a distância média entre vértices, preferi não usar a função que acha a distãncia entre dois vértices, pois ficaria muito lerdo. Preferi fazer uma função separada que roda uma BFS para cada nodo e depois tira a média de todas as distâncias encontradas.

\section{Como usar}

Para usar o EP, deve-se passar como argumento na linha de comando o \texttt{k} descrito no enunciado.

Depois disso, basta digitar \texttt{help} na prompt, e aparecerá explicações para todos os comandos possíveis.

\newpage

\section{Experimentos}

Vamos testar como o tipo do texto, sua língua e seu tamanho influenciam os resultados.

\subsection{Romances}

\subsubsection{Português}

\noindent\mytitle{k=1}
\begin{center}
\begin{tabular}{||c | c | c | c | c | c | c | c | c||}
\hline
Livro & Vertices & Arestas & Componentes & Maior Comp. & Menor Comp. & Média Comp. & Dist. Média & Denso \\ [0.5ex]
\hline\hline
cidade-e-as-serras.txt & 13405 & 9598 & 7444 & 3248 & 1 & 1.799 & 8.861 & False \\
\hline
domCasmurro.txt & 9475 & 7825 & 5115 & 2393 & 1 & 1.850 & 8.490 & False \\
\hline
memoriasPostumas.txt & 10976 & 8293 & 6307 & 2715 & 1 & 1.739 & 8.711 & False \\
\hline
quincasBorba.txt & 11335 & 9486 & 6195 & 2829 & 1 & 1.828 & 8.740 & False \\
\hline
reliquia.txt & 14655 & 10646 & 7991 & 3586 & 1 & 1.833 & 9.083 & False \\
\hline
\end{tabular}
\end{center}

\noindent\mytitle{k=3}
\begin{center}
\begin{tabular}{||c | c | c | c | c | c | c | c | c||}
\hline
Livro & Vertices & Arestas & Componentes & Maior Comp. & Menor Comp. & Média Comp. & Dist. Média & Denso \\ [0.5ex]
\hline\hline
cidade-e-as-serras.txt & 13311 & 8915 & 7448 & 3151 & 1 & 1.786 & 9.003 & False \\
\hline
domCasmurro.txt & 9370 & 6917 & 5116 & 2289 & 1 & 1.830 & 8.692 & False \\
\hline
memoriasPostumas.txt & 10873 & 7434 & 6310 & 2602 & 1 & 1.722 & 8.954 & False \\
\hline
quincasBorba.txt & 11232 & 8505 & 6200 & 2721 & 1 & 1.810 & 8.947 & False \\
\hline
reliquia.txt & 14567 & 9970 & 7995 & 3492 & 1 & 1.821 & 9.212 & False \\
\hline
\end{tabular}
\end{center}

\noindent\mytitle{k=5}
\begin{center}
\begin{tabular}{||c | c | c | c | c | c | c | c | c||}
\hline
Livro & Vertices & Arestas & Componentes & Maior Comp. & Menor Comp. & Média Comp. & Dist. Média & Denso \\ [0.5ex]
\hline\hline
cidade-e-as-serras.txt & 12527 & 6582 & 7488 & 2195 & 1 & 1.672 & 10.942 & False \\
\hline
domCasmurro.txt & 8626 & 4553 & 5163 & 1237 & 1 & 1.669 & 10.958 & False \\
\hline
memoriasPostumas.txt & 10061 & 4912 & 6377 & 1463 & 1 & 1.576 & 11.977 & False \\
\hline
quincasBorba.txt & 10408 & 5652 & 6262 & 1433 & 1 & 1.660 & 11.708 & False \\
\hline
reliquia.txt & 13738 & 7452 & 8048 & 2420 & 1 & 1.706 & 11.399 & False \\
\hline
\end{tabular}
\end{center}

\subsubsection{Inglês}

\noindent\mytitle{k=1}
\begin{center}
\begin{tabular}{||c | c | c | c | c | c | c | c | c||}
\hline
Livro & Vertices & Arestas & Componentes & Maior Comp. & Menor Comp. & Média Comp. & Dist. Média & Denso \\ [0.5ex]
\hline\hline
adventuresHuckleberry.txt & 7301 & 8218 & 3647 & 2647 & 1 & 1.999 & 7.237 & False \\
\hline
adventuresTomSawyer.txt & 8128 & 6730 & 4695 & 2305 & 1 & 1.729 & 7.297 & False \\
\hline
frankstein.txt & 7173 & 4512 & 4479 & 1460 & 1 & 1.599 & 7.409 & False \\
\hline
ladySusan.txt & 2978 & 1435 & 2058 & 487 & 1 & 1.442 & 7.560 & False \\
\hline
lastMan.txt & 13902 & 10078 & 8397 & 3196 & 1 & 1.654 & 8.804 & False \\
\hline
prideAndPrejudice.txt & 6799 & 4023 & 4485 & 1177 & 1 & 1.514 & 7.268 & False \\
\hline
\end{tabular}
\end{center}

\noindent\mytitle{k=3}
\begin{center}
\begin{tabular}{||c | c | c | c | c | c | c | c | c||}
\hline
Livro & Vertices & Arestas & Componentes & Maior Comp. & Menor Comp. & Média Comp. & Dist. Média & Denso \\ [0.5ex]
\hline\hline
adventuresHuckleberry.txt & 7203 & 7294 & 3651 & 2542 & 1 & 1.970 & 7.433 & False \\
\hline
adventuresTomSawyer.txt & 8073 & 6327 & 4699 & 2238 & 1 & 1.716 & 7.392 & False \\
\hline
frankstein.txt & 7107 & 4118 & 4483 & 1388 & 1 & 1.583 & 7.674 & False \\
\hline
ladySusan.txt & 2932 & 1187 & 2066 & 431 & 1 & 1.414 & 8.779 & False \\
\hline
lastMan.txt & 13835 & 9536 & 8401 & 3125 & 1 & 1.646 & 8.960 & False \\
\hline
prideAndPrejudice.txt & 6698 & 3253 & 4490 & 1069 & 1 & 1.490 & 7.560 & False \\
\hline
\end{tabular}
\end{center}

\noindent\mytitle{k=5}
\begin{center}
\begin{tabular}{||c | c | c | c | c | c | c | c | c||}
\hline
Livro & Vertices & Arestas & Componentes & Maior Comp. & Menor Comp. & Média Comp. & Dist. Média & Denso \\ [0.5ex]
\hline\hline
adventuresHuckleberry.txt & 5963 & 2630 & 3812 & 627 & 1 & 1.562 & 16.407 & False \\
\hline
adventuresTomSawyer.txt & 6978 & 2600 & 4848 & 565 & 1 & 1.437 & 12.323 & False \\
\hline
frankstein.txt & 6315 & 2036 & 4578 & 150 & 1 & 1.377 & 7.342 & False \\
\hline
ladySusan.txt & 2531 & 511 & 2069 & 19 & 1 & 1.218 & 1.963 & False \\
\hline
lastMan.txt & 12606 & 5180 & 8504 & 1586 & 1 & 1.481 & 14.729 & False \\
\hline
prideAndPrejudice.txt & 6041 & 1680 & 4562 & 99 & 1 & 1.322 & 7.099 & False \\
\hline
\end{tabular}
\end{center}


\subsection{Textos Filosóficos}

\subsubsection{Alemão}

\noindent\mytitle{k=1}
\begin{center}
\begin{tabular}{||c | c | c | c | c | c | c | c | c||}
\hline
Livro & Vertices & Arestas & Componentes & Maior Comp. & Menor Comp. & Média Comp. & Dist. Média & Denso \\ [0.5ex]
\hline\hline
Aphorismen.txt & 11099 & 8924 & 6577 & 2270 & 1 & 1.686 & 8.291 & False \\
\hline
thePrince\_Al.txt & 6240 & 3729 & 3717 & 1028 & 1 & 1.676 & 9.765 & False \\
\hline
\end{tabular}
\end{center}

\noindent\mytitle{k=3}
\begin{center}
\begin{tabular}{||c | c | c | c | c | c | c | c | c||}
\hline
Livro & Vertices & Arestas & Componentes & Maior Comp. & Menor Comp. & Média Comp. & Dist. Média & Denso \\ [0.5ex]
\hline\hline
Aphorismen.txt & 10910 & 6699 & 6595 & 2024 & 1 & 1.653 & 8.830 & False \\
\hline
thePrince\_Al.txt & 6182 & 3431 & 3717 & 957 & 1 & 1.660 & 10.056 & False \\
\hline
\end{tabular}
\end{center}

\noindent\mytitle{k=5}
\begin{center}
\begin{tabular}{||c | c | c | c | c | c | c | c | c||}
\hline
Livro & Vertices & Arestas & Componentes & Maior Comp. & Menor Comp. & Média Comp. & Dist. Média & Denso \\ [0.5ex]
\hline\hline
Aphorismen.txt & 10032 & 4734 & 6572 & 1090 & 1 & 1.525 & 10.259 & False \\
\hline
thePrince\_Al.txt & 5820 & 2814 & 3708 & 617 & 1 & 1.567 & 10.598 & False \\
\hline
\end{tabular}
\end{center}


\subsubsection{Inglês}

\noindent\mytitle{k=1}
\begin{center}
\begin{tabular}{||c | c | c | c | c | c | c | c | c||}
\hline
Livro & Vertices & Arestas & Componentes & Maior Comp. & Menor Comp. & Média Comp. & Dist. Média & Denso \\ [0.5ex]
\hline\hline
critiquePureReason.txt & 7995 & 4593 & 5461 & 952 & 1 & 1.462 & 7.227 & False \\
\hline
discourseMethod.txt & 2891 & 1151 & 2040 & 372 & 1 & 1.412 & 7.520 & False \\
\hline
manifestoCommunist.txt & 2257 & 865 & 1653 & 286 & 1 & 1.359 & 7.617 & False \\
\hline
thePrince.txt & 4774 & 2788 & 3107 & 862 & 1 & 1.533 & 7.114 & False \\
\hline
\end{tabular}
\end{center}

\noindent\mytitle{k=3}
\begin{center}
\begin{tabular}{||c | c | c | c | c | c | c | c | c||}
\hline
Livro & Vertices & Arestas & Componentes & Maior Comp. & Menor Comp. & Média Comp. & Dist. Média & Denso \\ [0.5ex]
\hline\hline
critiquePureReason.txt & 7859 & 3285 & 5467 & 805 & 1 & 1.436 & 8.171 & False \\
\hline
discourseMethod.txt & 2861 & 1042 & 2047 & 327 & 1 & 1.393 & 9.040 & False \\
\hline
manifestoCommunist.txt & 2218 & 687 & 1664 & 198 & 1 & 1.327 & 8.825 & False \\
\hline
thePrince.txt & 4708 & 2357 & 3113 & 784 & 1 & 1.509 & 7.813 & False \\
\hline
\end{tabular}
\end{center}

\noindent\mytitle{k=5}
\begin{center}
\begin{tabular}{||c | c | c | c | c | c | c | c | c||}
\hline
Livro & Vertices & Arestas & Componentes & Maior Comp. & Menor Comp. & Média Comp. & Dist. Média & Denso \\ [0.5ex]
\hline\hline
critiquePureReason.txt & 7272 & 2099 & 5513 & 20 & 1 & 1.317 & 1.951 & False \\
\hline
discourseMethod.txt & 2500 & 508 & 2037 & 15 & 1 & 1.222 & 1.679 & False \\
\hline
manifestoCommunist.txt & 1917 & 305 & 1636 & 7 & 1 & 1.166 & 1.375 & False \\
\hline
thePrince.txt & 4066 & 1066 & 3128 & 37 & 1 & 1.297 & 3.496 & False \\
\hline
\end{tabular}
\end{center}


\subsection{Manual}

\subsubsection{Inglês}

\noindent\mytitle{k=1}
\begin{center}
\begin{tabular}{||c | c | c | c | c | c | c | c | c||}
\hline
Livro & Vertices & Arestas & Componentes & Maior Comp. & Menor Comp. & Média Comp. & Dist. Média & Denso \\ [0.5ex]
\hline\hline
cat.txt & 133 & 129 & 95 & 32 & 1 & 1.295 & 2.423 & False \\
\hline
gcc.txt & 11781 & 11984 & 7083 & 2267 & 1 & 1.662 & 6.543 & False \\
\hline
ghc.txt & 1702 & 797 & 1307 & 131 & 1 & 1.295 & 4.380 & False \\
\hline
grep.txt & 962 & 860 & 672 & 132 & 1 & 1.417 & 4.108 & False \\
\hline
sort.txt & 317 & 368 & 234 & 52 & 1 & 1.312 & 2.235 & False \\
\hline
wget.txt & 2400 & 1591 & 1669 & 374 & 1 & 1.432 & 6.545 & False \\
\hline
\end{tabular}
\end{center}

\noindent\mytitle{k=3}
\begin{center}
\begin{tabular}{||c | c | c | c | c | c | c | c | c||}
\hline
Livro & Vertices & Arestas & Componentes & Maior Comp. & Menor Comp. & Média Comp. & Dist. Média & Denso \\ [0.5ex]
\hline\hline
cat.txt & 106 & 9 & 97 & 3 & 1 & 0.990 & 1.182 & False \\
\hline
gcc.txt & 11472 & 7512 & 7107 & 1917 & 1 & 1.613 & 8.532 & False \\
\hline
ghc.txt & 1638 & 383 & 1322 & 16 & 1 & 1.231 & 2.241 & False \\
\hline
grep.txt & 886 & 229 & 682 & 21 & 1 & 1.284 & 2.611 & False \\
\hline
sort.txt & 275 & 42 & 239 & 4 & 1 & 1.109 & 1.212 & False \\
\hline
wget.txt & 2308 & 759 & 1683 & 241 & 1 & 1.365 & 10.290 & False \\
\hline
\end{tabular}
\end{center}

\noindent\mytitle{k=5}
\begin{center}
\begin{tabular}{||c | c | c | c | c | c | c | c | c||}
\hline
Livro & Vertices & Arestas & Componentes & Maior Comp. & Menor Comp. & Média Comp. & Dist. Média & Denso \\ [0.5ex]
\hline\hline
cat.txt & 65 & 0 & 65 & 1 & 1 & 0.846 & -nan & False \\
\hline
gcc.txt & 9951 & 4165 & 7171 & 79 & 1 & 1.386 & 3.049 & False \\
\hline
ghc.txt & 1405 & 175 & 1235 & 4 & 1 & 1.130 & 1.069 & False \\
\hline
grep.txt & 713 & 107 & 610 & 7 & 1 & 1.152 & 1.297 & False \\
\hline
sort.txt & 194 & 12 & 182 & 2 & 1 & 1.011 & 1.000 & False \\
\hline
wget.txt & 1926 & 354 & 1606 & 9 & 1 & 1.193 & 1.312 & False \\
\hline
\end{tabular}
\end{center}

\section{Conclusão}

Com todos os testes, pude perceber o seguinte:

\begin{itemize}
  
\item O tamanho faz uma diferença muito grande entre os textos, o que era de se esperar. É mais fácil usar palavras diferentes em textos pequenos do que em textos grandes e isso faz com que quanto maior um texto fica, mais difícil é aumentar o seu número de vértices, e quando uma palavra diferente é adcionada, mais possível é que ela se conecte em diversas partes do grafo. Isso fica be claro ao ver que o número de arestas em textos grandes é bem mais próximo do número de vértices, e em textos pequenos, não chega nem perto.
  
\item O autor faz uma certa diferença nos textos também, mas não é tão grande. Apesar do fato de diferentes autores terem diferentes vocabulários, o número de conexões que acontecem entre as palavras em seus textos acabam ficam bem parecidos, caso as outras condições também sejam parecidas.
  
\item O gênero influencia os textos também, por dois motivos. Textos de diferentes gêneros tendem a ter tamanhos diferentes e, obviamente, o jeito de se expressar neles têm que ser diferente.
  
\item A língua faz uma certa diferença também, apesar de não ser tão grande.
  
\item A última coisa que pude notar é que variar o k influenciou os diversos textos da mesma maneira: a maioria dos dados diminuiu um pouco, com exceção do tamanho da maior componente e do número de arestas, que diminui \mytitle{drasticamente}.
  
\end{itemize}

\end{document}