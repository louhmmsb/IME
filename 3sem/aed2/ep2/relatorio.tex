\documentclass[12pt]{article}
\usepackage[utf8]{inputenc}
\usepackage{amsfonts}
\usepackage{amsmath}
\usepackage{siunitx}
\usepackage{amssymb}
\usepackage{enumitem}
\usepackage{mathtools}
\usepackage[brazil]{babel}
\usepackage{geometry}
\usepackage{graphicx}
\usepackage{bussproofs}
\usepackage[table]{xcolor}
\usepackage{gensymb}
\usepackage{hyperref}
\graphicspath{{./images}}
\geometry{verbose,a4paper,left=1cm,top=2cm,right=3cm,bottom=3cm}
\title{Algoritmos e Estruturas de Dados 2 - EP2}
\author{Lourenço Henrique Moinheiro Martins Sborz Bogo}
\date{}
\linespread{1.5}
\newcommand{\real}{\mathbb{R}}
\newcommand{\product}[3]{\displaystyle\prod_{#1}^#2 #3}
\newcommand{\gsum}[3]{\displaystyle\sum_{#1}^#2 #3}
\newcommand{\mytitle}[1]{\textbf{\underline{#1}}}
\newcommand{\ring}[1]{\langle #1 \rangle}
\newcommand{\code}[1]{\mbox{\texttt{#1}}}


\begin{document}

\maketitle

\section{Escolhas de Implementação}

Primeiro problema que encontrei ao começar a implementar o EP foi: como eu saberia em que vértice uma certa palavra está, de maneira rápida? Para resolver isso, eu decidi usar uma HashTable, onde as chave é a string que contém a palavra e o valor é o nodo no qual ela está.

Para implementar a HashTable, decidi fazer minha própria biblioteca de lista ligada, que transformei em uma biblioteca de fila para que pudesse ser usada na BFS também.

Depois disso, implentar quase todas as funções do EP foi muito simples, com exceção das \texttt{emCiclo}, que demorei um pouco para pensar como fazer. Decidi implementar quase tudo usando DFS, menos o cálculo da distância onde optei por usar BFS.

Tive que fazer duas funções DFS auxliares:

\begin{description}
\item[\texttt{int dfs(int)}] Roda uma dfs para o nodo passado como parâmetro e devolve o tamanho de sua componente.
  
\item[\texttt{void dfs(int, int, int, bool, int)}] Essa dfs foi a parte mais complicada de implementar do EP. Ela pode caso sejam passados só os 3 primeiros parâmetros para ela, ela irá procurar se existe ou não um ciclo que contém o primeiro parâmetro. Caso os 5 parâmetros sejam passados, a função irá fazer quase a mesma coisa, com a condição de que ao invés de buscar um ciclo no primeiro parâmetro, ela irá buscar um ciclo no primeiro parâmetro, que contenha o último.
  
\end{description}

\newpage

\section{Como usar}

Para usar o EP, deve-se passar como argumento na linha de comando o \texttt{k} descrito no enunciado.

Depois disso, basta digitar \texttt{help} na prompt, e aparecerá explicações para todos os comandos possíveis.

\section{Experimentos}

Primeiro experimento que decidi fazer é como textos de diferentes tipos se comportam:

\begin{center}
  \begin{tabular}{||c | c | c | c | c | c | c | c||} 
    \hline
    Livro & Vertices & Arestas & Componentes & Maior Comp. & Menor Comp. & Conexo & Denso \\ [0.5ex] 
    \hline\hline
    Memórias & 10976 & 8293 & 6307 & 2715 & 1 & False & False \\
    \hline
    Dom Casmurro & 9475 & 7825 & 5115 & 2393 & 1 & False & False \\ 
    \hline
    Quincas & 11335 & 9486 & 6195 & 2829 & 1 & False & False \\ 
    \hline
    The Raven & 445 & 197 & 309 & 51 & 1 & False & False \\ 
    \hline
    Discourse & 2819 & 1151 & 2040 & 372 & 1 & False & False \\ 
    \hline
    Critique & 7995 & 4593 & 5461 & 952 & 1 & False & False \\ 
    \hline
    Dom Casmurro & 9475 & 7825 & 5115 & 2393 & 1 & False & False \\ [1ex]
    \hline
  \end{tabular}
\end{center}

\mytitle{OBS} Todos os testes feitos na tabela a cima foram feitos com k = 1;

\end{document}