% Created 2020-06-29 seg 15:51
% Intended LaTeX compiler: pdflatex
\documentclass[11pt]{article}
\usepackage[utf8]{inputenc}
\usepackage[T1]{fontenc}
\usepackage{graphicx}
\usepackage{grffile}
\usepackage{longtable}
\usepackage{wrapfig}
\usepackage{rotating}
\usepackage[normalem]{ulem}
\usepackage{amsmath}
\usepackage{textcomp}
\usepackage{amssymb}
\usepackage{capt-of}
\usepackage{hyperref}
\usepackage{minted}
\usepackage[hyperref, x11names]{xcolor}
\hypersetup{colorlinks = true, urlcolor = SteelBlue4, linkcolor = black}
\usepackage[brazilian]{babel}
\author{Lourenço Henrique Moinheiro Martins Sborz Bogo - 11208005}
\date{\today}
\title{Exercícios do Capítulo 15}
\hypersetup{
 pdfauthor={Lourenço Henrique Moinheiro Martins Sborz Bogo - 11208005},
 pdftitle={Exercícios do Capítulo 15},
 pdfkeywords={},
 pdfsubject={},
 pdfcreator={Emacs 26.3 (Org mode 9.3.7)}, 
 pdflang={Brazilian}}
\begin{document}

\maketitle
\tableofcontents

\newpage

\section{Exercício 3}
\label{sec:orgb0d3c52}
\paragraph{}Precisamos deduzir a fórmula do erro da \emph{Midpoint Rule}.
Para isso, primeiro vamos expandir a função \(f(x)\), que
queremos, integrar usando taylor, até o termo de segunda ordem, 
ao redor do ponto \(\frac{a+b}{2}\) (a e b são nossas bordas de
integração):

\(f(x) = f(\frac{a+b}{2}) + (x-\frac{a+b}{2})f'(\frac{a+b}{2}) + (x-\frac{a+b}{2})^2\frac{f''(\xi_x)}{2}\) \\

Agora, iremos calcular o erro em si:

\noindent\(\displaystyle\int_a^b f(x)dx - \displaystyle\int_a^b f(\frac{a+b}{2})dx = \\
  = \displaystyle\int_a^b f(\frac{a+b}{2}) + (x-\frac{a+b}{2})f'(\frac{a+b}{2}) + (x-\frac{a+b}{2})^2\frac{f''(\xi_x)}{2} - f(\frac{a+b}{2}) = \\
  = \displaystyle\int_a^b(x-(\frac{a+b}{2}))f'(\frac{a+b}{2}) + \frac{(x-\frac{a+b}{2})^2}{2}f''(\xi_x)dx = \\
  = \displaystyle\int_a^b(x-(\frac{a+b}{2}))f'(\frac{a+b}{2}) + \displaystyle\int_a^b\frac{(x-\frac{a+b}{2})^2}{2}f''(\xi_x)dx = \\
  = 0 + \frac{f''(\eta)}{2}\displaystyle\int_a^b(x-\frac{a+b}{2})^2dx\)  (Pelo T.V.M. para integrais) \(= \\ 
  = \frac{(b-a)^3}{24}f''(\eta)\), \(\eta \in [a, b]\).

\newpage

\section{Exercício 4}
\label{sec:org8a40601}
\subsection{(a)}
\label{sec:org4ac7550}
Sabemos que o erro da \emph{Hermite Cubic Interpolation} é:\\

\(\displaystyle\frac{f^4(\eta)(x-a)^2(x-b)^2}{4!}\) \\

\noindent Portanto podemos escrever:\\

\(\displaystyle f(x) - p(x) = \frac{f^4(\eta)(x-a)^2(x-b)^2}{4!}\) \\

\noindent Depois, integramos os dois lados de a até b, conseguindo:\\

\(\displaystyle\int_a^b f(x) - p(x) 
   = \displaystyle\int_a^b \frac{f^4(\eta)(x-a)^2(x-b)^2}{4!} \rightarrow \\
   \rightarrow \displaystyle\int_a^b f(x) - \int_a^b p(x) =
   \frac{f^4(\eta)}{4!}\int_a^b(x-a)^2(x-b)^2\) \\

\noindent Ou seja, o erro de usarmos esse método para calcular a integral é \\
\(\displaystyle\frac{f^4(\eta)}{4!}\displaystyle\int_a^b(x-a)^2(x-b)^2\).\\

\noindent Agora precisamos simplificar essa expressão:\\

\(\displaystyle\frac{f^4(\eta)}{4!}\displaystyle\int_a^b(x-a)^2(x-b)^2 = 
   \frac{f^4(\eta)}{4!}\displaystyle\int_a^b(x^2-2ax+a^2)(x^2-2bx+b^2) = \\
   = \frac{f^4(\eta)}{4!}\frac{6b^5-6a^5+10b^5+40ab^4+10a^2b^3-10a^5-10a^3b^2-40a^4b+15a^2+45a^4b-15b^5-45ab^4}{30} = \\
   = \frac{f^4(\eta)(b-a)^5}{720}\), \(\eta \in [a, b]\).

\newpage

\subsection{(b)}
\label{sec:org4759d0e}

\begin{minted}[]{py}
import numpy as np

'''Função que calcula a regra pedida no enunciado para a 
função f de derivada df'''
def trap(f, df, a, b):
    c = b-a
    return (c/2)*(f(a)+f(b))+((c*c)/12)*(df(a)-df(b))

'''Função dada no enunciado (e^x)'''
def g(x):
    return np.exp(x)


'''Apenas cálculo do erro'''
res1 = np.e - 1
res2 = np.e - np.e**(.9)

aprox1 = trap(g, g, 0, 1)
aprox2 = trap(g, g, 0.9, 1)

err1 = abs(res1 - aprox1)
err2 = abs(res2 - aprox2)

print(err1)
print(err2)
\end{minted}

\begin{center}
\begin{tabular}{lllll}
\hline
Intervalo & Trapezoid & Simpson & Midpoint & Corrected\\
\hline
\hline
[0, 1] & 0.1408 & \(6\cdot 10^{-4}\) & \(6.96\cdot 10^{-2}\) & \(2\cdot 10^{-3}\)\\
\hline
[0.9, 1] & \(2.2\cdot 10^{-4}\) & \(9\cdot 10^{-9}\) & \(1.1\cdot 10^{-4}\) & \(3.6\cdot 10^{-8}\)\\
\hline
\hline
\end{tabular}
\end{center}

Podemos perceber que Corrigido aproximou melhor que a maioria dos outros métodos, com exceção do de Simpson,
o que já era esperado, já que ele usa um polinômio de segundo grau, ao invés de um polinômio de primeiro.

\newpage

\section{Exercício 6}
\label{sec:org83bf0d2}
\subsection{(a)}
\label{sec:org483ce4d}
Para conseguir a fórmula é muito simples. Ao invés de aplicar a fórmula direto
em \([a, b]\), vamos quebrar o intervalo em vários intervalos menores, equidistantes,
e aplicar a fórmula nesses intervalos.
Começaremos com a notação:

\begin{itemize}
\item \(h = (a-b)/n\), com \(n\) sendo o número de sub-intervalos.
\item \(x_i = a + ih\)
\end{itemize}

Temos então que a fórmula seria construída aplicando a fórmula do trapezío corrigida
para os intervalos \([x_{i}, x_{i+1}]\): \\

\(\displaystyle I_{cctr} = \sum_{i=0}^{n-1}(\frac{x_{i+1}-x_i}{2}(f(x_{i+1})+f(x_i))+\frac{(x_{i+1}-x_i)^2}{12}(f'(x_{i+1})-f'(x_i))) = \\
   = \sum_{i=0}^{n-1}(\frac{h}{2}(f(x_{i+1})+f(x_i))+\frac{h^2}{12}(f'(x_{i+1})-f'(x_i))) = \\
   = \frac{h}{2}\sum_{i=0}^{n-1}(f(x_{i+1})+f(x_i)) + \frac{h^2}{12}\sum_{i=0}^{n-1}(f'(x_{i+1})-f'(x_i))\)

Agora, podemos perceber que a primeira parcela dessa soma: \\
\(\displaystyle\frac{h}{2}\sum_{i=0}^{n-1}(f(x_{i+1})+f(x_i))\),
é a regra do trapézio composto, e, que a segunda parcela vai se cancelar quase que inteira, sobrando apenas 
\(f'(x_0)\) e \(f'(x_n)\), que são \(f'(a)\) e \(f'(b)\). 

Fazendo, então, todas as simplificações necessárias, chegamos na seguinte fórmula: \\

\(\displaystyle h(\frac{f(a)+f(b)}{2}+\sum_{i=1}^{n-1}f(x_i)) + h^2\frac{f'(a)-f'(b)}{12} = I_{tr} + h^2\frac{f'(a)-f'(b)}{12}\)

\newpage

Código: 
\begin{minted}[]{python}
import numpy as np

n = 10
a = 0;
b = 1;
h = (b-a)/n

def nCorr(f, h, a, b):
    fS = 0
    for i in np.arange(a+h, b, h):
	fS += f(i)

    return h*(fS + (f(a)+f(b))/2)

def corr(f, df, h, a, b):
    resp = nCorr(f, h, a, b)
    resp += (h*h*(df(a)-df(b)))/12
    return resp

def g(x):
    return np.exp(-(x*x))

def dg(x):
    return -2*x*np.exp(-(x*x))

print(f"Valor dado    = 0.746824133")
print(f"Não corrigida = {nCorr(g, h, a, b)}")
print(f"Corrigida     = {corr(g, dg, h, a, b)}")
\end{minted}

\begin{center}
\begin{tabular}{rrr}
Valor Dado no Enunciado & Não corrigida & Corrigida\\
\hline
0.746824133 & 0.746210796 & 0.746823928\\
 &  & \\
\end{tabular}
\end{center}

Podemos perceber que o erro na versão não corrigida é da ordem de \(10^{-4} = h^4\) e
na versão corrigida é da ordem de \(10^{-6} = h^6\), ou seja, uma melhora da ordem de \(10^{-2} = h^2\) nos experimentos.
\newpage
\subsection{(b)}
\label{sec:org8fc3a7d}
Primeiro, vamos mostrar que o erro do trapézio corrigido composto 
é \(O(h^4)\):

\(\displaystyle I-I_{tcc} = \sum_{i=0}^{n-1}\frac{f^4(\eta_i)h^5}{720} = 
   \frac{h^5}{720}\sum_{i=0}^{n-1}f^4(\eta_i)\).

Agora vamos multiplicar a expressão por \(\frac{a-b}{nh}\). Podemos fazer
isso pois essa fração é igual a 1.

\(\displaystyle\frac{a-b}{nh}\frac{h^5}{720}\sum_{i=0}^{n-1}f^4(\eta_i)\).

Agora usando o Teorema do Valor Intermediário, 
\(\displaystyle \frac{\sum_{i=0}^{n-1}f^4(\eta_i)}{n} = f^4(\eta)\), \(\eta \in [a, b]\), onde \([a, b]\) é o nosso intervalo de integração.\\

Simplificando, temos: 

\(\displaystyle I-I_{tcc} = \frac{h^4(a-b)f^4(\eta)}{720} = O(h^4)\).\\

Com essa informação, podemos provar o que o exercício pede:

\(I-I_{tcc} = O(h^4) \rightarrow I-I_{tc}-h^2\frac{f'(a)-f'(b)}{12} = O(h^4) \rightarrow \\
   \rightarrow I-I_{tc} = h^2\frac{f'(a)-f'(b)}{12}+O(h^4)\).
\end{document}