\documentclass[12pt]{article}
\usepackage[utf8]{inputenc}
\usepackage{amsfonts}
\usepackage{amsmath}
\usepackage{siunitx}
\usepackage{amssymb}
\usepackage{enumitem}
\usepackage{mathtools}
\usepackage[brazil]{babel}
\usepackage{geometry}
\usepackage{graphicx}
\usepackage{bussproofs}
\usepackage[table]{xcolor}
\usepackage{gensymb}
\usepackage{hyperref}
\graphicspath{{./images}}
\geometry{verbose,a4paper,left=2cm,top=2cm,right=3cm,bottom=3cm}
\title{Capítulo 11 Exercício 5}
\author{Lourenço Henrique Moinheiro Martins Sborz Bogo}
\date{}
\linespread{1.5}
\newcommand{\real}{\mathbb{R}}
\newcommand{\product}[3]{\displaystyle\prod_{#1}^#2 #3}
\newcommand{\gsum}[3]{\displaystyle\sum_{#1}^#2 #3}
\newcommand{\mytitle}[1]{\textbf{\underline{#1}}}
\newcommand{\ring}[1]{\langle #1 \rangle}
\newcommand{\code}[1]{\mbox{\texttt{#1}}}


\begin{document}

\maketitle

\mytitle{Notação}

\begin{itemize}
  
\item $S_i$ é o polinômio que interpola os pontos $i$ e $i+1$.
  
\item $z_i = S''(x_i)$
  
\end{itemize}



Para que nosso spline cúbico satisfaça a condição not a knot, é necessário que nossas condições livres sejam:

\begin{itemize}
  
\item $S_0(x) \equiv S_1(x)$
  
\item $S_{n-2}(x) \equiv S_{n-1}(x)$
  
\item $h_i = x_{i+1}-x_i$
  
\item $b_i = \frac{y_{i+1}-y_i}{h_i}$
  
\end{itemize}

Ou seja, o primeiro polinômio deve ser exatamente igual ao segundo e o penúltimo deve ser exatamente igual ao último.

Isso implica que $S_0'''(x) = S_1'''(x)$ e que $S_{n-2}'''(x) = S_{n-1}'''(x)$.

Logo, temos que:
$\frac{z_1-z_0}{h_0} = \frac{z_2-z_1}{h_1}$ e $\frac{z_n-z_{n-1}}{h_{n-1}} = \frac{z_{n-1}-z_{n-2}}{h_{n-2}}$.

Então, a primeira e a última equação do sistema linear são:

\begin{itemize}
  
\item $-h_1z_0 + (h_0+h_1)z_1 - h_0z_2 = 0$
  
\item $-h_{n-1}z_{n-2} + (h_{n-2}+h_{n-1})z_{n-1} - h_{n-2}z_n = 0$.
  
\end{itemize}

Além dessas duas, temos as equações que vêm da interpolação cubica independente da condição que estamos usando:

$h_{i-1}z_{i-1}+2(h_{i-1}+h_i)z_i+h_{}z_{i+1} = 6(b_i-b_{i-1}),\: i=1,2,\dots,n-1$.

Desse modo, podemos escrever nosso sistema linear na forma de matrizes:

$H\cdot\vec{z} = \vec{v}$,

$\vec{z} = 
\begin{bmatrix}
  
  z_0 \\
  z_1 \\
  z_2 \\
  \vdots \\
  z_{n-2} \\
  z_{n-1} \\
  z_n \\
  
\end{bmatrix}
$, 
$\vec{v} =
\begin{bmatrix}

  0 \\
  6(b_1-b_0) \\
  6(b_2-b_1) \\
  \vdots \\
  6(b_{n-2}-b_{n-3}) \\
  6(b_{n-1}-b_{n-2}) \\
  0 \\
  
\end{bmatrix}
$
E a nossa matriz de coficientes (que é o que queremos para o problema) é:

$H =
\begin{bmatrix}

  -h_1 & h_0+h_1 & -h_0 \\
  h_0 & 2(h_0+h_1) & h_1 \\
  & h_1 & 2(h_1+h_2) & h_2 \\
  & \ddots & \ddots & \ddots \\
  & & h_{n-3} & 2(h_{n-3}+h_{n-2}) & h_{n-2} \\
  & & & h_{n-2} & 2(h_{n-2}+h_{n-1}) & h_n \\
  & & & -h_{n-1} & h_{n-2}+h_{n-1} & -h_{n-2}
  
\end{bmatrix}
$

Podemos ver que ela é tridiagonal caso desconsideremos a primeira e a última linha e que ela é dominante diagonalmente (os espaços em branco são 0).

\end{document}