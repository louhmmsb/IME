\documentclass[12pt]{article}
\usepackage[utf8]{inputenc}
\usepackage{amsfonts}
\usepackage{amsmath}
\usepackage{siunitx}
\usepackage{amssymb}
\usepackage{enumitem}
\usepackage{mathtools}
\usepackage[brazil]{babel}
\usepackage{geometry}
\usepackage{graphicx}
\usepackage{bussproofs}
\usepackage[table]{xcolor}
\usepackage{gensymb}
\usepackage{hyperref}
\graphicspath{{./images}}
\geometry{verbose,a4paper,left=2cm,top=2cm,right=3cm,bottom=3cm}
\title{Algoritmo do Spline Quadrático}
\author{}
\date{}
\linespread{1.5}
\newcommand{\real}{\mathbb{R}}
\newcommand{\product}[3]{\displaystyle\prod_{#1}^#2 #3}
\newcommand{\gsum}[3]{\displaystyle\sum_{#1}^#2 #3}
\newcommand{\mytitle}[1]{\textbf{\underline{#1}}}
\newcommand{\ring}[1]{\langle #1 \rangle}
\newcommand{\code}[1]{\mbox{\texttt{#1}}}


\begin{document}

\maketitle

Dado um intervalo $[a,b]$ e uma divisão desse intervalo em n+1 pontos equidistantes, conseguimos fazer o spline quadrático de uma certa função, nesse intervalo. Além dessas condições iniciais, iremos precisar também do valor da função $f$ nos pontoss que queremos interpolar (óbvio) e também, da derivada da $f$ no ponto inicial do intervalo.

Notação que usarei:

\begin{itemize}
  
\item O intervalo $[a,b]$ será dividido em $n+1$ pontos que chamarei de $x_1, x_2, \dots, x_n$.
  
\item O valor da função $f$ em aplicada no ponto $x_k$ será chamado $y_k$.
  
\item O valor da derivada da função $f$ no ponto $x_k$ será chamado de $z_k$.
  
\item O polinômio que interpola os pontos $k$ e $k+1$ será chamado de $Q_k$.
  
\end{itemize}

\mytitle{Explicação do Algoritmo}

Inicialmente, iremos achar a derivada em todos os outros pontos recursivamente, a partir da derivada inicial com a fórmula:

$z_{i+1} = -z_i + 2(\frac{y_{i+1}-y_i}{x_{i+1}-x_i})$.

Com todas as derivadas em mãos, agora podemos criar cada um dos polinômios que irão interpolar o intervalo dois a dois pontos consecutivos com a seguinte fórmula:

$Q_i(x) = \frac{z_{i+1}-z_{i}}{2(x_{i+1}-x_i)}(x-x_i)^2+z_i(x-x_i)+y_i$.

\mytitle{Justificativa}

Primeiramente iremos precisar de $n-1$ polinômios de segundo grau (ou menos) contínuos, cujas derivadas também são contínuas. Ou seja necessitamos que:

\begin{itemize}
  
\item $Q_i(x_i) = y_i$ para $i = 0, 1, \dots, n-1$.
  
\item $Q_i(x_{i+1}) = y_{i+1}$ para $i = 0, 1, \dots, n-1$.
  
\item $Q'_i(x_i) = Q'_{i+1}(x_i)$ para $i = 1, 2, \dots, n-1$.
  
\end{itemize}

Como $Q(x)$ é um spline quadrático de nossa $f$, sua derivada seŕa um spline linear. Desse modo, surgem 4 novas condições:

\begin{itemize}
  
\item $Q_i(x_i) = y_i$ (1).
  
\item $Q_i(x_{i+1}) = y_{i+1}$ (2).
  
\item $Q'_i(x_i) = z_i$ (3).
  
\item $Q'_i(x_{i+1}) = z_{i+1}$ (4).
  
\end{itemize}

De (1), (3) e (4):

$Q_i(x) = \frac{z_{i+1}-z_{i}}{2(x_{i+1}-x_i)}(x-x_i)^2+z_i(x-x_i)+y_i$ (5).

Agora para achar os valores de $z_i$ iremos usar (2):

$z_{i+1} = -z_i + 2(\frac{y_{i+1}-y_i}{x_{i+1}-x_i})$ (6).

Com (6), conseguiremos computar $z_k$ dado $z_{k-1}$, logo precisaremos do $f'(a)$ para usar essa algoritmo.

\end{document}