\documentclass[12pt]{article}
\usepackage[utf8]{inputenc}
\usepackage{amsfonts}
\usepackage{amsmath}
\usepackage{siunitx}
\usepackage{amssymb}
\usepackage{enumitem}
\usepackage{mathtools}
\usepackage[brazil]{babel}
\usepackage{geometry}
\usepackage{graphicx}
\usepackage{bussproofs}
\usepackage[table]{xcolor}
\usepackage{gensymb}
\usepackage{hyperref}
\graphicspath{{./images}}
\geometry{verbose,a4paper,left=1cm,top=2cm,right=3cm,bottom=3cm}
\title{Cap11 Ex5}
\author{Lourenço Henrique Moinheiro Martins Sborz Bogo - 11208005}
\date{}
\linespread{1.5}
\newcommand{\real}{\mathbb{R}}
\newcommand{\product}[3]{\displaystyle\prod_{#1}^#2 #3}
\newcommand{\gsum}[3]{\displaystyle\sum_{#1}^#2 #3}
\newcommand{\mytitle}[1]{\textbf{\underline{#1}}}
\newcommand{\ring}[1]{\langle #1 \rangle}
\newcommand{\code}[1]{\mbox{\texttt{#1}}}
\setcounter{MaxMatrixCols}{20}

\begin{document}

\maketitle

Começaremos com um spline cúbico padrão, ou seja, temos as seguintes condições:

\begin{enumerate}
  
\item $s_i(x_i) = f(x_i)$, $i = 0,\dots,n-1$
  
\item $s_i(x_{i+1}) = f(x_{i+1})$, $i = 0,\dots,n-1$
  
\item $s_i'(x_{i+1}) = s_{i+1}'(x_{i+1})$, $i = 0,\dots,n-2$
  
\item $s_i''(x_{i+1}) = s_{i+1}''(x_{i+1})$, $i = 0,\dots,n-2$
  
\end{enumerate}

Onde:

\begin{enumerate}
  
\item $s_i(x) = a_i+b_i(x-x_i)+c_i{(x-x_i)}^2+d_i{(x-x_i)}^3$
  
\item $s_i'(x) = b_i+2c_i{(x-x_i)}+3d_i{(x-x_i)}^2$
  
\item $s_i''(x) = 2c_i+6d_i{(x-x_i)}$
  
\end{enumerate}

O que nós queremos inicialmente é determinar os coeficientes desses polinômios para $i = 0,1,\dots,n-1$

Notação:
\begin{itemize}

\item $a_i = f(x_i)$, $i = 0,1,\dots,n-1$  
  
\item $h_i = x_{i+1}-x_i$, $i = 0,1,\dots,n-1$
  
\item $f[x_i, x_{i+1}] = \frac{f(x_{i+1})-f(x_{i})}{x_{i+1}-x_i}$
  
\end{itemize}

Então, aplicando todas as condições impostas para splines cúbicos, chegaremos em:

$h_{i-1}c_{i-1}+2(h_{i-1}+h_i)c_i+h_i c_{i+1} = 3(f[x_i, x_{i+1}]-f[x_{i-1}, x_i])$, $i = 1,\dots,n-1$.

(Os passos dessa parte foram pulados pois estão todos no livro e o exercício em si começa agora).

O sistema acima nos dará $n-1$ equações lineares, e queremos achar $n+1$ incógnitas. Ou seja, agora para fazer o not a knot, precisaremos apenas usar como nossas duas condições livres:

\begin{itemize}
  
\item $d_0 = d_1$
  
\item $d_{n-1} = d_{n-2}$
  
\end{itemize}

Peguemos do livro a equação para achar $d_i$:

$d_i = \frac{c_{i+1}-c_i}{3h_i}$, $i = 0,\dots,n-1$.

Vamos agora usar as condições impostas pelo knot a not:

\begin{itemize}
  
\item $d_0 = d_1 \rightarrow \frac{c_1-c_0}{3h_0} = \frac{c_2-c_1}{3h_1} \rightarrow c_1-c_0 = \frac{h_0}{h_1}(c_2-c_1) \rightarrow c_0 = c_1 - \frac{h_0}{h_1}(c_2-c_1)$
  
\item $d_{n-2} = d_{n-1} \rightarrow \frac{c_{n-1}-c_{n-2}}{3h_{n-2}} = \frac{c_n-c_{n-1}}{3h_{n-1}} \rightarrow \frac{h_{n-1}}{h_{n-2}}(c_{n-1}-c_{n-2}) = c_n-c_{n-1} \rightarrow \\ \rightarrow c_n = c_{n-1} + \frac{h_{n-1}}{h_{n-2}}(c_{n-1}-c_{n-2})$
  
\end{itemize}

Agora, pegaremos o o sistema que tínhamos antes e nos casos $i=1$ e $i=n-1$ iremos substituir $c_0$ e $c_{n}$ respecetivamente, resultando nesse novo sistema:

$\begin{cases}
  
h_{i-1}c_{i-1}+2(h_{i-1}+h_i)c_i+h_i c_{i+1} = 3(f[x_i, x_{i+1}]-f[x_{i-1}, x_i])$, $i = 2,\dots,n-2 \\

h_{0}(c_1 - \frac{h_0}{h_1}(c_2-c_1))+2(h_{0}+h_1)c_1+h_1 c_{2} = 3(f[x_1, x_{2}]-f[x_{0}, x_1]) \\

h_{n-2}c_{n-2}+2(h_{n-2}+h_{n-1})c_{n-1}+h_{n-1} (c_{n-1} + \frac{h_{n-1}}{h_{n-2}}(c_{n-1}-c_{n-2})) = 3(f[x_{n-}, x_{n}]-f[x_{n-2}, x_{n-1}])
  
\end{cases}$

Ou seja,

$\begin{cases}
  
h_{i-1}c_{i-1}+2(h_{i-1}+h_i)c_i+h_i c_{i+1} = 3(f[x_i, x_{i+1}]-f[x_{i-1}, x_i])$, $i = 2,\dots,n-2 \\

c_1(3h_0+2h_1+\frac{h_0^2}{h_1})+c_2(h_1-\frac{h_0^2}{h_1}) = 3(f[x_1, x_{2}]-f[x_{0}, x_1]) \\

c_{n-1}(3h_{n-1}+2h_{n-2}+\frac{h_{n-1}^2}{h_{n-2}})+c_{n-2}(h_{n-2}-\frac{h_{n-1}^2}{h_{n-2}}) = 3(f[x_{n-}, x_{n}]-f[x_{n-2}, x_{n-1}])
  
\end{cases}$

\newpage

Agora, escrevendo esse sistema na forma matricial, temos:

$H\vec{c} = \vec{f}$, com

$
H = \begin{bmatrix}

  3h_0+2h_1+\frac{h_0^2}{h_1} && h_1-\frac{h_0^2}{h_1} \\
  h_1 && 2(h_1+h_2) && h_2 \\
  && h_2 && 2(h_2+h_3) && h_3 \\
  && && \ddots && \ddots && \ddots \\
  && && && h_{n-3} && 2(h_{n-3}+h_{n-2}) && h_{n-2} \\
  && && && && h_{n-2}-\frac{h_{n-1}^2}{h_{n-2}} && 3h_{n-1}+2h_{n-2}+\frac{h_{n-1}^2}{h_{n-2}}
   
\end{bmatrix}$

$\vec{c} = \begin{bmatrix}
  
  c_1 \\ c_2 \\ \vdots \\ c_{n-2} \\ c_{n-1}

\end{bmatrix}$

$\vec{f} = \begin{bmatrix}

3(f[x_1, x_2]-f[x_0, x_1]) \\ 3(f[x_2, x_3]-f[x_1, x_2]) \\ \vdots \\ 3(f[x_{n-2}, x_{n-1}]-f[x_{n-3}, x_{n-2}]) \\ 3(f[x_{n-1}, x_{n}]-f[x_{n-2}, x_{n-1}])

\end{bmatrix}$

A matrix que estamos interessados para responder o exercício é a $H$, e podemos ver que ela é estritamente diagonal dominante e tridiagonal.

\end{document}