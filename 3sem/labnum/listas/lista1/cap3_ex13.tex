\documentclass[12pt]{article}
\usepackage[utf8]{inputenc}
\usepackage{amsfonts}
\usepackage{amsmath}
\usepackage{siunitx}
\usepackage{amssymb}
\usepackage{enumitem}
\usepackage{mathtools}
\usepackage[brazil]{babel}
\usepackage{geometry}
\usepackage{graphicx}
\usepackage{bussproofs}
\usepackage[table]{xcolor}
\usepackage{gensymb}
\graphicspath{{./images}}
\geometry{verbose,a4paper,left=2cm,top=2cm,right=3cm,bottom=3cm}
\title{Lista}
\author{Lourenço Henrique Moinheiro Martins Sborz Bogo}
\date{}
\linespread{1.5}
\newcommand{\real}{\mathbb{R}}
\newcommand{\product}[3]{\displaystyle\prod_{#1}^#2 #3}
\newcommand{\gsum}[3]{\displaystyle\sum_{#1}^#2 #3}
\newcommand{\mytitle}[1]{\textbf{\underline{#1}}}
\newcommand{\ring}[1]{\langle #1 \rangle}
\newcommand{\glim}[1]{\displaystyle\lim_{#1}}

\begin{document}

\maketitle

Para provar que perto do ponto $0$ a função pode ser apoximada por $\frac{-x}{3}$ iremos usar Taylor.

$f(x) = \frac{xcos(x) - sin(x)}{x^2}$.
$f'(x) = \frac{-x^3sin(x) - 2x(xcos(x)-sin(x))}{x^4}$.

Denominando $t(x)$ a expansão de Taylor de primeira ordem da função $f(x)$ usando o ponto $x_0 = 0$ como base, temos:

$t(x) = f(x_0) + f'(x_0)(x - 0) = f(x_0) + f'(x_0)(x)$.

Como $f'(x)$ e $f(x)$ não estão bem definidas no ponto $x = 0$, precisamos calcular o limite quando $x \to 0$. Temos então:
\\

$\displaystyle\lim_{x \to 0}f'(x) = \displaystyle\lim_{x \to 0}\frac{-x^3sin(x) - 2x(xcos(x)-sin(x))}{x^4} = \displaystyle\lim_{x \to 0}\frac{-x^2sin(x) - 2(xcos(x)-sin(x))}{x^3} \rightarrow L'Hopital \rightarrow \displaystyle\lim_{x \to 0}\frac{-2xsin(x)-2x^2cos(x)+2xsen(x)}{3x^2} = \displaystyle\lim_{x \to 0}\frac{-2xsin(x)-2x^2cos(x)+2xsen(x)}{3x^2} = \displaystyle\lim_{x \to 0}\frac{-x^2cos(x)}{3x^2} = \frac{-1}{3}$.
\\

$\displaystyle\lim_{x \to 0}f(x) = \displaystyle\lim_{x \to 0}\frac{xcos(x)-sen(x)}{x^2} \rightarrow L'Hopital \rightarrow \displaystyle\lim_{x \to 0}\frac{-xsen(x)}{2x} = 0$
\\

Agora, por fim, temos:
\\

$\displaystyle\lim_{x_0 \to 0}t(x) = \displaystyle\lim_{x_0 \to 0}f(x_0) + f'(x_0)(x) = 0 + \frac{-x}{3} = \frac{-x}{3}$
.

\end{document}