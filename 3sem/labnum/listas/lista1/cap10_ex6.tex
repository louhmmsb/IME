\documentclass[12pt]{article}
\usepackage[utf8]{inputenc}
\usepackage{amsfonts}
\usepackage{amsmath}
\usepackage{siunitx}
\usepackage{amssymb}
\usepackage{enumitem}
\usepackage{mathtools}
\usepackage[brazil]{babel}
\usepackage{geometry}
\usepackage{graphicx}
\usepackage{bussproofs}
\usepackage[table]{xcolor}
\usepackage{gensymb}
\graphicspath{{./images}}
\geometry{verbose,a4paper,left=2cm,top=2cm,right=3cm,bottom=3cm}
\title{Lista}
\author{Lourenço Henrique Moinheiro Martins Sborz Bogo}
\date{}
\linespread{1.5}
\newcommand{\real}{\mathbb{R}}
\newcommand{\product}[3]{\displaystyle\prod_{#1}^#2 #3}
\newcommand{\gsum}[3]{\displaystyle\sum_{#1}^#2 #3}
\newcommand{\mytitle}[1]{\textbf{\underline{#1}}}
\newcommand{\ring}[1]{\langle #1 \rangle}


\begin{document}

\maketitle

\mytitle{Monomial Basis:}
Primeiro temos que montar um sistema linear com os pontos que nos foram dados: $(-1, 1), (0, 1), (1, 2), (2, 0)$

$
\begin{cases}
  -a+b-c+d = 1\\
  0a+0b+0c+d = 1\\
  a+b+c+d = 2\\\
  8a+4b+2c+1 = 0
\end{cases}
$


Resolvendo esse sistema linear temos que:

$
\begin{cases}
  a = \frac{-2}{3}\\
  b = \frac{1}{2}\\
  c = \frac{7}{6}\\
  d = 1
\end{cases}
$
\\

Portanto o polinômio pela base monomial é: $\frac{4x^3-3x^2-7x-6}{-6}$


\newpage

\mytitle{Lagrange Basis:} Temos que achar os polinômios da base de Lagrange:

$L(x) = \gsum{j=0}{k}{y_jl_j(x)}$,

com $l_i(x) = \product{j=0, j \neq i}{k}{\frac{x-x_i}{x_j-x_i}}$.

Para o nosso exemplo, temos:

$l_0(x) = \product{i=0, 0 \neq i}{k}{\frac{x-x_i}{x_j-x_i}} = \frac{x}{-1} \cdot \frac{x-1}{-2} \cdot \frac{x-2}{-3} = \frac{x^3-3x^2+2x}{-6}\\$

$l_1(x) = \product{i=0, 1 \neq i}{k}{\frac{x-x_i}{x_j-x_i}} = \frac{x+1}{1} \cdot \frac{x-1}{-1} \cdot \frac{x-2}{-2} = \frac{x^3-2x^2-x+2}{2}\\$

$l_2(x) = \product{i=0, 2 \neq i}{k}{\frac{x-x_i}{x_j-x_i}} = \frac{x+1}{2} \cdot \frac{x}{1} \cdot \frac{x-2}{-1} = \frac{x^3-x^2-2x}{-2}\\$

$l_3(x) = \product{i=0, 3 \neq i}{k}{\frac{x-x_i}{x_j-x_i}} = \frac{x+1}{3} \cdot \frac{x}{2} \cdot \frac{x-1}{1} = \frac{x^3-x}{6}\\$

Agora que temos os polinômios da base de lagrange, podemos construir finalmente o polinômio final utilizando a somatória $L(x)$.

$L(x) = \gsum{j=0}{k}{y_jl_j(x)} = \frac{x^3-3x^2+2x}{-6} + \frac{-3x^3+6x^2+3x-6}{-6} + \frac{6x^3-6x^2-12x}{-6} + 0 = \\ = \frac{4x^3-3x^2-7x-6}{-6}$.

\newpage

\mytitle{Newton Basis}: Para esse método, temos que achar $P_{n-1}(x)$, o polinômio que interpola n pontos, para depois construir $P_n$ recursivamente.

Primeiro, vamos construir $P_0(x)$, o polinômio que interpola 1 ponto.

$P_0(x) = a_0 \rightarrow P_0(x_0) = y_0 \rightarrow P_0(-1) = 1$.

Agora para $n = 1$, temos:

$P_1(x) = P_0(x) + a_1(x-x_0) \rightarrow a_1 = \frac{y_1-y_0}{x_1-x_0} = \frac{1-1}{0-(-1)} = 0 \rightarrow P_1(x) = P_0(x) = y_0 = 1$

$n=2$:

$P_2(x) = P_1(x) + a_2(x-x_0)(x-x_1) \rightarrow y_2 = P_1(x_2) + a_2(x_2-x_0)(x_2-x_1) \rightarrow a_2 = \frac{y_2-P_1(x)}{(x_2-x_0)(x_1-x_0)} = \frac{1}{2} \rightarrow P_2(x) = 1 + \frac{(x+1)(x)}{2}$

$n = 3$:

$P_3(x) = P_2(x) + a_3(x-x_0)(x-x_1)(x-x_2) \rightarrow y_3 = P_2(x_3) + a_3(x_3-x_0)(x_3-x_1)(x_3-x_2) \rightarrow a_3 = \frac{y_3-P_2(x_3)}{(x_3-x_0)(x_3-x_1)(x_3-x_2)} = \frac{-4}{6} \rightarrow P_3(x) = \frac{6}{6} + \frac{3x^2+3x}{6} + \frac{-4(x+1)(x)(x-1)}{6} = \frac{-4x^3+3x^2+7x+6}{6}$.

\end{document}