\documentclass[12pt]{article}
\usepackage[utf8]{inputenc}
\usepackage{amsfonts}
\usepackage{amsmath}
\usepackage{siunitx}
\usepackage{amssymb}
\usepackage{enumitem}
\usepackage{mathtools}
\usepackage[brazil]{babel}
\usepackage{geometry}
\usepackage{graphicx}
\usepackage{bussproofs}
\usepackage[table]{xcolor}
\usepackage{gensymb}
\usepackage{hyperref}
\graphicspath{{./images}}
\geometry{verbose,a4paper,left=2cm,top=2cm,right=3cm,bottom=3cm}
\title{LabNum - EP3}
\author{Lourenço Henrique Moinheiro Martins Sborz Bogo - 11208005}
\date{}
\linespread{1.5}
\newcommand{\real}{\mathbb{R}}
\newcommand{\product}[3]{\displaystyle\prod_{#1}^#2 #3}
\newcommand{\gsum}[3]{\displaystyle\sum_{#1}^#2 #3}
\newcommand{\mytitle}[1]{\textbf{\underline{#1}}}
\newcommand{\ring}[1]{\langle #1 \rangle}
\newcommand{\code}[1]{\mbox{\texttt{#1}}}
\usepackage{listings}
\usepackage{color}

\definecolor{dkgreen}{rgb}{0,0.6,0}
\definecolor{gray}{rgb}{0.5,0.5,0.5}
\definecolor{mauve}{rgb}{0.58,0,0.82}

\lstset{frame=tb,
  language=C++,
  aboveskip=3mm,
  belowskip=3mm,
  showstringspaces=false,
  columns=flexible,
  basicstyle={\small\ttfamily},
  numbers=none,
  numberstyle=\tiny\color{gray},
  keywordstyle=\color{blue},
  commentstyle=\color{dkgreen},
  stringstyle=\color{mauve},
  breaklines=true,
  breakatwhitespace=true,
  tabsize=4
}


\begin{document}

\maketitle

\tableofcontents

\newpage

\section{Parte 1}

\subsection{Implementação}

A primeira parte do EP pedia para que, usando os dados de uma tabela, interpolássemos um certa função usando um polinômio e depois integrássemos o polinômio numericamente usando a Regra do Trapezóide e a Regra de Simpson.

Para a parte da interpolação, optei por fazer usando o Polinômio de Newton, pois achei que seria mais fácil.

Implementei o método usando as seguintes funções:

\begin{description}
  
\item[\texttt{calcula\_den(int index)}] Essa função calcula a seguinte produtória $\product{i=0}{{index-1}}{x_{index}-x_i}$.
  
\item[\texttt{calcula\_p(int grau, double xp)}] Avalia o polinômio de ordem grau no ponto xp. Essa função só funciona caso os coeficientes necesssários para calcular esse polinômio já tenham sido calculados antes.
  
\item[\texttt{calcula\_a(int index)}] Calcula o coeficiente do termo de grau index do Polinômio de Newton.
  
\item[\texttt{calcula\_as()}] Calcula todos os coeficientes do polinômio.
  
\item[\texttt{trapezoid(int n)}] Pega $n$ pontos igualmente espaçados do polinômio de newton (já construído) e integra usando a Regra do Trapezóide.
  
\item[\texttt{simpsons(int n)}] Pega $2n$ pontos igualmente espaçados do polinômio de newton (já construído) e integra usando a Regra de Simpson.
  
\end{description}

\newpage

\subsection{Resultados}

Percebi com testes que, por mais que a Regra de Simpson demore mais pra rodar, ela é muito mais eficiente no quesito de conseguir um resultado muito melhor, particionando o polinômio menos vezes.

\begin{center}
  \begin{tabular}{||c | c | c ||}
    \hline
    Partições & Trapezóide & Simpson \\ [0.5ex]
    \hline\hline
    10 & 117.832290 & 117.130914 \\
    \hline
    100 & 117.138600 & 117.131621 \\
    \hline
    1000 & 117.131691 & 117.131621 \\
    \hline
    10000 & 117.131622 & 117.131621 \\
    \hline
    100000 & 117.131621 & 117.131621 \\
    \hline
    1000000 & 117.131621 & 117.131621 \\
    \hline
    10000000 & 117.131621 & 117.131621 \\
    \hline
    100000000 & 117.131621 & 117.131621 \\
    \hline
  \end{tabular}
\end{center}

Percebe-se na tabela que, particionando o conjunto $10^2$ vezes e usando a regra de simpson, consegue-se o mesmo resultado até a sexta casa decimal que particionando o conjunto $10^5$ vezes e usando a Regra do Trapezóide.

\newpage

\section{Parte 2}

\subsection{Implementação}

Fiz dois programas nessa parte:

\begin{description}
  
\item[\texttt{pt2.c}] Esse programa resolve as integrais dos itens 1, 2 e 3 do enunciado, usando o Método de Monte Carlo. O programa recebe como parâmetro o número de variáveis aleatórias que serão usadas para aplicar o método e o número de sementes que serão testadas (caso esse número seja 100, serão testadas as sementes de 0 a 100).
  Sua saída será o menor erro encontrado (e sua semente), o maior erro encontrado (e sua semente) e o erro médio, para cada um dos métodos.

  As integrais que calculei nesse programa e as substituições usadas foram as seguintes:
  
  $\displaystyle\int_0^1\sin{(x)}dx$
  
  $\displaystyle\int_3^7x^3dx = 4\displaystyle\int_0^1(4t+3)^3dt$

  $\displaystyle\int_0^{\infty}e^{-x}dx = \displaystyle\int_0^1e^{-x}dx + \displaystyle\int_1^{\infty}e^{-x}dx \rightarrow x = \frac{1}{u} \quad dx = \frac{-1}{u^2}du \rightarrow \displaystyle\int_0^1e^{-x}dx + \displaystyle\int_1^0\frac{-e^{\frac{-1}{u}}}{u^2}du = \displaystyle\int_0^1e^{-x}dx + \displaystyle\int_0^1\frac{e^{\frac{-1}{u}}}{u^2}du = \displaystyle\int_0^1(e^{-x}+\frac{e^{\frac{-1}{x}}}{x^2})dx$

  Depois de achar as substituições, foi muito simples. Apenas gerei valores aleatórios entre 0 e 1, apliquei na função que quero integrar e tirei a média dos resultados. Discutirei os resultados encontrados na seção de resultados.
  
\item[\texttt{circ.c}] Aqui eu resolvi o item 4 do enunciado usando o Método de Monte Carlo. Foi feita basicamente a mesma coisa que no programa descrito no item anterior, apenas com uma alteração para tratar o fato de que estamos em duas dimensões: ao invés de gerar 1 valor aleatório, eu gerei um par de valores aleatórios entre 0 e 1 e passei para a função que $g(x, y)$ dada no enunciado.

  Tirei a média dos resultados, multipliquei por 4, e voilá! $\pi$ (pelo menos perto o suficiente).
  
\end{description}

\newpage

\subsection{Resultados}

Colocarei aqui tabelas com os erros que obtive ao comparar as funções integradas analiticamente com as integradas por Monte Carlo:

\begin{center}
  \begin{tabular}{||c | c | c | c | c ||}
    \hline
    Função & Número de V.A. & Menor Erro & Maior Erro & Erro Médio \\ [0.5ex]
    \hline\hline
    $\sin{x}$ & 10 & 3.610268e-05 & 2.328637e-01 & 6.322206e-02 \\
    \hline
    $x^3$ & 10 & 1.113892e-01 & 3.824100e+02 & 9.294158e+01 \\
    \hline
    $e^{-x}$ & 10 & 2.390419e-05 & 1.184496e-01 & 3.206498e-02 \\
    \hline
    $\sin{x}$ & 100 & 4.442498e-06 & 8.549161e-02 & 1.889440e-02 \\
    \hline
    $x^3$ & 100 & 1.299647e-01 & 1.218060e+02 & 2.797397e+01 \\
    \hline
    $e^{-x}$ & 100 & 2.870791e-05 & 4.890248e-02 & 1.003721e-02 \\
    \hline
    $\sin{x}$ & 1000 & 6.023429e-06 & 2.526216e-02 & 6.077690e-03 \\
    \hline
    $x^3$ & 1000 & 3.933176e-02 & 3.620151e+01 & 8.935365e+00 \\
    \hline
    $e^{-x}$ & 1000 & 1.481880e-06 & 1.434760e-02 & 3.167431e-03 \\
    \hline
    $\sin{x}$ & 10000 & 3.760302e-06 & 8.087467e-03 & 1.978368e-03 \\
    \hline
    $x^3$ & 10000 & 7.316047e-03 & 1.157567e+01 & 2.853298e+00 \\
    \hline
    $e^{-x}$ & 10000 & 3.004806e-06 & 4.082099e-03 & 9.929621e-04 \\
    \hline
    $\sin{x}$ & 100000 & 1.330556e-06 & 2.676081e-03 & 6.605803e-04 \\
    \hline
    $x^3$ & 100000 & 3.156454e-03 & 3.921654e+00 & 9.587524e-01 \\
    \hline
    $e^{-x}$ & 100000 & 2.250376e-06 & 1.508875e-03 & 3.319549e-04 \\
    \hline
    $\sin{x}$ & 1000000 & 6.454527e-08 & 8.011643e-04 & 1.959469e-04 \\
    \hline
    $x^3$ & 1000000 & 1.348677e-03 & 1.181796e+00 & 2.842198e-01 \\
    \hline
    $e^{-x}$ & 1000000 & 3.699904e-07 & 4.150512e-04 & 1.036093e-04 \\
    \hline
  \end{tabular}
\end{center}

Podemos perceber que a função $x^3$ é a que se comporta da pior maneira em relação ao método, tendo erro médio $10^3$ vezes mais alto que os outros métodos. A outras duas funções convegem muito bem, com o erro sendo da ordem de $10^{-4}$ para $10^{-6}$ variáveis aleatórias. O verdadeiro problema desse método é o tempo que ele demora. Para por exemplo, conseguir um erro médio de da ordem de $10^{-10}$, precisaríamos de uma quantidade gigantesca de variáveis aleatórias, o que consumiria muito tempo.

\newpage

Agora para o círculo:

\begin{center}
  \begin{tabular}{||c | c | c | c ||}
    \hline
    Número de V.A. & Menor Erro & Maior Erro & Erro Médio \\ [0.5ex]
    \hline\hline
    10 & 0.458407 & 3.98126 & 2.62904 \\
    \hline
    100 & 0.000515128 & 0.706994 & 0.196243 \\
    \hline
    1000 & 0.000488647 & 0.178986 & 0.0444891 \\
    \hline
    10000 & 5.07252e-05 & 0.0575342 & 0.013288 \\
    \hline
    100000 & 1.34666e-05 & 0.0183472 & 0.00404787 \\
    \hline
    1000000 & 7.78275e-08 & 0.00525208 & 0.00126605 \\
    \hline
  \end{tabular}
\end{center}

Nesse caso, podemos perceber que o erro médio converge devagar. Mesmo com $10^6$ variáveis aleatórias, em média, o erro foi da ordem de $10^{-3}$.
O melhor caso é muito bom, mas isso não significa muita coisa, pois os testes são aleatórios.

\mytitle{OBS:} Foram testadas 1000 sementes para cada teste acima.

O que posso concluir desse método é que para ele ser utilizável na prática, é necessário 

\end{document}