\documentclass[12pt]{article}
\usepackage[utf8]{inputenc}
\usepackage{amsfonts}
\usepackage{amsmath}
\usepackage{siunitx}
\usepackage{amssymb}
\usepackage{enumitem}
\usepackage{mathtools}
\usepackage[brazil]{babel}
\usepackage{geometry}
\usepackage{graphicx}
\usepackage{bussproofs}
\usepackage[table]{xcolor}
\usepackage{gensymb}
\usepackage{hyperref}
\graphicspath{{./images}}
\geometry{verbose,a4paper,left=2cm,top=2cm,right=3cm,bottom=3cm}
\title{Relatorio}
\author{Lourenço Henrique Moinheiro Martins Sborz Bogo - 11208005}
\date{}
\linespread{1.5}
\newcommand{\real}{\mathbb{R}}
\newcommand{\product}[3]{\displaystyle\prod_{#1}^#2 #3}
\newcommand{\gsum}[3]{\displaystyle\sum_{#1}^#2 #3}
\newcommand{\mytitle}[1]{\textbf{\underline{#1}}}
\newcommand{\ring}[1]{\langle #1 \rangle}


\begin{document}

\maketitle

\mytitle{Parte 1 do EP: Método do Ponto Fixo}

Para a primeira parte do EP, eu escolhi a função de ponto fixo $g(x) = x - \frac{f(x)}{f'(x)}$. Como essa é a função do Método de Newton, se usarmos um ponto inicial "próximo" o suficiente da raiz, dentro de um certo intervalo, iremos achá-la.

Para achar cada uma das raízes foi usada precisão de $10^{-8}$ e os seguintes pontos iniciais:

$
\begin{cases}
  
  X_0 = 1.5 \to 1.487962 \\
  X_0 = -0.5 \to -0.539835 \\
  X_0 = 2.5 \to 2.617867 \\
  
\end{cases}\\
$

Essa parte do EP é implementada em apenas um arquivo, o fixed.c, que contém tudo que foi pedido e está devidamente comentado.

\mytitle{Detalhes de Implementação:}

\begin{itemize}

\item Optei também por usar ponteiros de funções pois isso facilitaria o resto da implementação.

\end{itemize}

\mytitle{Parte 2 do EP: Método de Newton}

Fiz essa parte do EP de uma maneira um pouco diferente da que foi pedida. Fiz uma pequena interface para que seja mais fácil utilizar o EP sem ter que mexer no código fonte. Ele pede primeiro para que seja passado quantos pixeis serão calculados em cada direção, depois, pede os pontos u e l, respectivamente, e por último são oferecidas 6 funções para plotar.

O programa usa precisão de $10^{-8}$ para aplicar o método de newton e $10^{-5}$ para diferenciar as raízes. A segunda precisão maior pois caso isso não seja feito, acharíamos várias vezes a mesma raiz, em consequência das aproximações de double feitas pela computador.

Escolhi algumas funções que estavam em artigos que li sobre fractais de Newton, coma a minha principal fonte sendo a \underline{\href{https://en.wikipedia.org/wiki/Newton\_fractal}{wikipedia}}.

As funções implementadas no EP são:


\begin{itemize}
  
\item  $x^4 - 1$
\item  $x^2 + 1$
\item  $x^3 - 1$
\item  $x^3 - x$
\item  $x^8 + 15x^4 - 16$
\item  $sin(x) - 1$
  
\end{itemize}


Os resultados dos experimentos podem ser vistos ao rodar o programa para uma das funções. O programa irá gerar um output chamado grafico.png, que é a imagem das bacias de convergência da função escolhida. O outro modo de utilizar o EP é rodar o script rodatudo.sh, que irá criar 7 bacias de convergência de uma vez e deixar suas imagens na pasta (paciência ao rodar esse script, ele demora um pouco).

\mytitle{Implementação:} O EP é dividido em vários arquivos, cada um deles com suas respectivas funções a seguir:

\begin{description}
  
\item[newton.c] É o arquivo principal, que calcula todos os pontos, os guarda em um arquivo chamado out.txt e depois chama o script printa.sh. O arquivo é dividido em várias funções, todas devidamente comentadas.
\item[printa.sh] É o arquivo que chama o gnuplot e plota os gráficos a partir do arquivo out.txt, deletando todos os arquivos auxiliares após.
\item[grafico.png] É a imagem da bacia de convergência.
  
\end{description}

\mytitle{Detalhes de Implementação:}

\begin{itemize}
  
\item Decidi usar uma matriz de ponteiros de funções para armazernar as funções matemáticas e suas derivadas, pois isso facilitaria o resto da implementação.
\item Decidi usar 30 como limite de iterações pois como a convergência é quadrática (rápida), é muito provável que isso seja suficiente para achar a raiz. Colocar mais iterações deixaria o programa mais lerdo, e poderia causar problemas ao achar as raízes, como por exemplo, achar algumas que não existem.
\item Decidi economizar o máximo possível no número de prints durante a execução do programa pois isso deixaria o programa muito mais lerdo também.
\item O script rodatudo.sh calcula 1000x1000 pontos, pois isso é o suficiente para conseguir uma ótima resolução do fractal.
\item Optei por deixar as imagens dentro do tar, pois é possível que quem for rodar o EP não tenha algum interpretador de bash, impedindo o uso dos scripts.
  
\end{itemize}

\end{document}