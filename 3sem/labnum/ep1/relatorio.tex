\documentclass[12pt]{article}
\usepackage[utf8]{inputenc}
\usepackage{amsfonts}
\usepackage{amsmath}
\usepackage{siunitx}
\usepackage{amssymb}
\usepackage{enumitem}
\usepackage{mathtools}
\usepackage[brazil]{babel}
\usepackage{geometry}
\usepackage{graphicx}
\usepackage{bussproofs}
\usepackage[table]{xcolor}
\usepackage{gensymb}
\graphicspath{{./images}}
\geometry{verbose,a4paper,left=2cm,top=2cm,right=3cm,bottom=3cm}
\title{Relatorio}
\author{Lourenço Henrique Moinheiro Martins Sborz Bogo}
\date{}
\linespread{1.5}
\newcommand{\real}{\mathbb{R}}
\newcommand{\product}[3]{\displaystyle\prod_{#1}^#2 #3}
\newcommand{\gsum}[3]{\displaystyle\sum_{#1}^#2 #3}
\newcommand{\mytitle}[1]{\textbf{\underline{#1}}}
\newcommand{\ring}[1]{\langle #1 \rangle}


\begin{document}

\maketitle

\mytitle{Parte 1 do EP: Método do Ponto Fixo}

Para o método do ponto fixo, eu escolhi a função de ponto fixo $g(x) = x - \frac{f(x)}{f'(x)}$. Essa, por ser também, a função do Método de Newton, achará todas as raízes dependendo apenas do ponto inicial estar dentro de um certo intervalo que contenha a raiz.

Para achar cada uma das raízes foi usada precisão de $10^{-8}$ e os seguintes pontos iniciais:

$
\begin{cases}
  
  X_0 = 1.5 \to 1.487962 \\
  X_0 = -0.5 \to -0.539835 \\
  X_0 = y2.5 \to 2.617867 
  
\end{cases}\\
$

\mytitle{Parte 2 do EP: Método de Newton}

Fiz essa parte do EP com uma aproximação um pouco diferente. Fiz uma pequena interface para que seja mais fácil utilizar o EP sem ter que mexer no código fonte. Ele pede primeiro para que seja passadddo quantos pixeis serão calculados em cada direção, depois ele pede os pontos u e l, respectivamente, depois é pedida a precisão e por último
são oferecidas 5 funções para plotar. Explicarei como adcionar outras funções mais a frente.


\end{document}