% Created 2020-07-18 sáb 18:07
% Intended LaTeX compiler: pdflatex
\documentclass[11pt]{article}
\usepackage[utf8]{inputenc}
\usepackage[T1]{fontenc}
\usepackage{graphicx}
\usepackage{grffile}
\usepackage{longtable}
\usepackage{wrapfig}
\usepackage{rotating}
\usepackage[normalem]{ulem}
\usepackage{amsmath}
\usepackage{textcomp}
\usepackage{amssymb}
\usepackage{capt-of}
\usepackage{hyperref}
\usepackage{minted}
\usepackage[hyperref, x11names]{xcolor}
\hypersetup{colorlinks = true, urlcolor = SteelBlue4, linkcolor = black}
\usepackage[brazilian]{babel}
\usepackage{geometry}
\geometry{verbose,a4paper,left=2cm,top=2cm,right=3cm,bottom=3cm}
\author{Lourenço Henrique Moinheiro Martins Sborz Bogo - 11208005}
\date{\today}
\title{Modelagem e Simulação - Auto-avaliação}
\hypersetup{
 pdfauthor={Lourenço Henrique Moinheiro Martins Sborz Bogo - 11208005},
 pdftitle={Modelagem e Simulação - Auto-avaliação},
 pdfkeywords={},
 pdfsubject={},
 pdfcreator={Emacs 26.3 (Org mode 9.3.7)}, 
 pdflang={Brazilian}}
\begin{document}

\maketitle
\tableofcontents

\newpage
\section{Questão 1}
\label{sec:org22293ae}
Eu acho que faltei em uma ou duas aulas presenciais, mas não tenho certeza.
Digo isso pois, antes do carnaval tive que fazer uma operação além de ficar
internado alguns dias após o procedimento, tive que ficar em casa por um tempo
depois de receber alta.

\section{Questão 2}
\label{sec:org3f209ba}
Acho que faltei em algumas também, pois tive que ajudar minha família com algumas coisas.

\section{Questão 3}
\label{sec:org74f1d9b}
Entendi a matéria muito bem. O breve conhecimento que eu tinha em física e
simulações básicas me ajudou a acompanhar a matéria com certa facilidade.

\section{Questão 4}
\label{sec:org0eb7ab1}
Essa pergunta tem duas respostas. Em relação ao conteúdo mais importante 
que eu aprendi, eu diria o método de Euler. Agora, essa matéria me ensinou
outras coisas muito importantes, indiretamente, como por exemplo: 
trabalhar em grupo e me deixou mais confortável para modelar certos
problemas. Considerando tudo isso, eu diria que a coisa mais importante
que eu aprendi foi como trabalhar em grupo. Ficou claro ao decorrer da
discliplina que organizar as tarefas, fazer cronogramas e aprender 
a utilizar ferramentas que auxiliem o trabalho em grupo é de extrema
importância para conseguir fazer um trabalho bem feito e de maneira
eficiente.

\section{Questão 5}
\label{sec:org7b4428c}
Essa questão é muito difícil pois acho que qualquer tipo de conhecimento
tem sua utilidade. Dito isso, eu diria que a coisa menos importante que
eu aprendi foi o Gantt Chart. Pessoalmente, não achei que ele serve bem
o seu propósito.

\section{Questão 6}
\label{sec:orga17ad45}
Sim, eu participei da primeira tarefa em grupo. Acredito que participei da
melhor maneira que pude, tentei cronometrar o mais precisamento possível
os tempos. Gostaria de ter corrido, mas graças à minha operação, não podia
fazer exercício físico.

\section{Questão 7}
\label{sec:org9a2c211}
Igual à primeira pergunta, não lembro muito bem. Porém é possível que eu
tenha faltado alguma aula no CEC, pois estava de cama graças à minha
operação. Fora isso, participei da melhor maneira que pude no CEC,
fazendo o que era pedido e perguntando quando não entendia algo.

\section{Questão 8}
\label{sec:org497f596}
Aprofundei o método de Euler um pouco mais. Acredito que aprendi bastante
sobre o método e consigo imaginar várias situações além das mostradas em aula
para a qual ele poderia ser útil.

\section{Questão 9}
\label{sec:org47fce4d}
Graças à quarentena, eu tenho mais horas do que o normal, diria que algo
por volta de 30 horas por semana. Dessas, eu devo ter utilizado algo em
torno de duas ou três horas para essa matéria.

\section{Questão 10}
\label{sec:org205fec0}
De certa maneira sim. Gostava de participar das aulas, pois semrpe surgia
uma dúvida, minha ou dos meus colegas, muito interessante e que de fato eu
gostaria de conhecer a resposta. Uma coisa que provavelmente aumentaria
minha motivação para participar seria se fossem discutidos outros métodos.
Sinto que ficamos muito tempo preso no método de Euler, mas entendo que 
seria muito difícil encaixar mais conteúdo na disciplina.

\section{Questão 11}
\label{sec:org612fbe0}
Como já foi dito no item acima, gostaria de ter aprendido mais métodos.
Porém, a matéria superou minhas expectativas em outros aspectos.
Os métodos que foram ensinados, eu aprendi muito bem e como já falei
antes, acho que nos foi ensinado a trabalhar em grupo, o que é muito
importante.
Em relação aos EPS:
\begin{itemize}
\item EP1: Achei um tanto quanto cansativo ter quem implementar vários 
movimentos de duas maneiras diferentes. Produzir o EP foi relativamente
monótono. Dito isso, os resultados foram muito legais de visualizar e
compensou o nosso esforço. Talvez esse EP tivesse sido mais 
interessante se eu não tivesse aprendido sobre os movimentos tão antes
de ter que fazê-lo. Isso tirou um pouco da graça, pois foi
relativamente fácil modelá-los.
\item EP2: Não posso falar o mesmo desse EP. Preciso falar muito melhor.
Esse EP foi incrivelmente entusiasmante de fazer, desde pesquisar os
dados no artigo do Sonnino, até implementar os métodos. O método
estocástico que fizemos no EP foi extremamente divertido, tanto de
codar, quanto visualizar. Aprendi muito sobre o funcionamento dos vírus
e sobre modelagem em si.
\end{itemize}

\section{Questão 12}
\label{sec:org5b8cadc}
Como já disse antes, gosstaria de ter aprendido outros métodos de 
modelagem, como por exemplo métodos estocásticos. Mas entendo que isso
faria a matéria ficar muito pesada.

\section{Questão 13}
\label{sec:org7cfa431}
Imagino que por tarefas práticas, estamos falando dos EPS. Já expliquei
como me senti em relação a cada um dos EPS, mas vou fazer um breve resumo.

O EP1 foi um pouco cansativo e não tive que pesquisar muito para fazê-lo
o que tirou um pouco a graça e fez com que não agregasse muito ao meu
conhecimento.

O EP2 foi muito interessante em todos os aspectos. Desde ter que
pesquisar certas informações no artigo do Sonnino até implementar os
métodos. Foi um dos EPS com o resultado mais legal de se visualizar.

Sinto-me perfeitamente capaz de continuar estudando o conteúdo sozinho,
e tenho quase certeza que o farei pois parece que fica cada vez mais 
interessante.

\section{Questão 14}
\label{sec:orgc4e02c6}
Eu achei a matéria tranquila de acompanhar, mas eu já tinha um certo
conhecimento prévio em alguns dos assuntos tratados. Participei de quase
todas as aulas (ou todas, como já disse, não me lembro se tive que faltar
em alguma) e dei o melhor que pude para resolver minhas dúvidas e fazer
os trabalhos da melhor maneira possível. Entendi bem o conteúdo dado 
e estudei por fora outros métodos pois achei o conceito da matéria muito
interessante.

Me esforcei ao máximo nos EPS, e sinto que mereço nota máxima por eles,
porém não me dediquei tanto à fazer os exercícios opcionais. Dito isso,
não me sinto muito confortável dando-me uma nota.
\end{document}