% Created 2020-07-21 ter 08:10
% Intended LaTeX compiler: pdflatex
\documentclass[11pt]{article}
\usepackage[utf8]{inputenc}
\usepackage[T1]{fontenc}
\usepackage{graphicx}
\usepackage{grffile}
\usepackage{longtable}
\usepackage{wrapfig}
\usepackage{rotating}
\usepackage[normalem]{ulem}
\usepackage{amsmath}
\usepackage{textcomp}
\usepackage{amssymb}
\usepackage{capt-of}
\usepackage{hyperref}
\usepackage{minted}
\usepackage[hyperref, x11names]{xcolor}
\hypersetup{colorlinks = true, urlcolor = SteelBlue4, linkcolor = black}
\usepackage[brazilian]{babel}
\usepackage{geometry}
\geometry{verbose,a4paper,left=2cm,top=2cm,right=3cm,bottom=3cm}
\author{Lourenço Henrique Moinheiro Martins Sborz Bogo - 11208005}
\date{\today}
\title{P3 - Funções Diferenciáveis e Séries}
\hypersetup{
 pdfauthor={Lourenço Henrique Moinheiro Martins Sborz Bogo - 11208005},
 pdftitle={P3 - Funções Diferenciáveis e Séries},
 pdfkeywords={},
 pdfsubject={},
 pdfcreator={Emacs 26.3 (Org mode 9.3.7)}, 
 pdflang={Brazilian}}
\begin{document}

\maketitle
\tableofcontents

\newpage

\section{Questão 1 (Lista 6 Q2)}
\label{sec:org56982f8}
\paragraph{}Primeiro vamos derivar em x e em y a função dada:

\(\displaystyle\\
  \frac{\partial{f}}{\partial{x}} = \frac{x^2}{(x^3+y^3)^{\frac{2}{3}}}\\
  \frac{\partial{f}}{\partial{y}} = \frac{y^2}{(x^3+y^3)^{\frac{2}{3}}}\)


Como as derivadas parciais são um polinômio sobre a
raiz de um outro polinômio, sabemos que elas são contínuas.
O único ponto que pode nos causar alguma dúvida é
o ponto \(x = -y\), pois isso faria com que o denominador
fosse 0, logo vamos tratar esse caso a parte, derivando
pela definição.

\(\displaystyle
  \frac{\partial{f}}{\partial{x}}(x_0, y_0) = 
  \lim_{x \to x_0}{\frac{f(x, y_0)-f(x_0, y_0)}{x-x_0}} =
  \lim_{x \to x_0}{\frac{(x^3+y_0^3)^{\frac{1}{3}}-(x_0^3+y_0^3)^{\frac{1}{3}}}{x-x_0}} = 
  \lim_{x \to x_0}{\frac{(x^3+y_0^3)^{\frac{1}{3}}}{x-x_0}} = \\
  = \lim_{x \to x_0}{\left(\frac{x^3-x_0^3}{(x-x_0)^3}\right)^{\frac{1}{3}}} = 
  \left(\lim_{x \to x_0}{\frac{x^3-x_0^3}{(x-x_0)^3}}\right)^{\frac{1}{3}} \xrightarrow{\text{L'Hopital}}
  \left(\lim_{x \to x_0}{\frac{3x^2}{3(x-x_0)^2}}\right)^{\frac{1}{3}} = +\infty\)

Com isso, provamos que a função é diferenciável em
qualquer ponto onde \(x \neq -y\) e que o gradiente é:

\(\displaystyle\nabla{f(x_0, y_0)} = \left(\frac{x_0^2}{(x_0^3+y_0^3)^{\frac{2}{3}}}, \frac{y_0^2}{(x_0^3+y_0^3)^{\frac{2}{3}}}\right)\)
\end{document}