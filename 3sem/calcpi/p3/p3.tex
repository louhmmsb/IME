% Created 2020-07-23 qui 11:40
% Intended LaTeX compiler: pdflatex
\documentclass[11pt]{article}
\usepackage[utf8]{inputenc}
\usepackage[T1]{fontenc}
\usepackage{graphicx}
\usepackage{grffile}
\usepackage{longtable}
\usepackage{wrapfig}
\usepackage{rotating}
\usepackage[normalem]{ulem}
\usepackage{amsmath}
\usepackage{textcomp}
\usepackage{amssymb}
\usepackage{capt-of}
\usepackage{hyperref}
\usepackage{minted}
\usepackage[hyperref, x11names]{xcolor}
\hypersetup{colorlinks = true, urlcolor = SteelBlue4, linkcolor = black}
\usepackage[brazilian]{babel}
\usepackage{geometry}
\usepackage{amsmath}
\geometry{verbose,a4paper,left=2cm,top=2cm,right=3cm,bottom=3cm}
\author{Lourenço Henrique Moinheiro Martins Sborz Bogo - 11208005}
\date{\today}
\title{P3 - Funções Diferenciáveis e Séries}
\hypersetup{
 pdfauthor={Lourenço Henrique Moinheiro Martins Sborz Bogo - 11208005},
 pdftitle={P3 - Funções Diferenciáveis e Séries},
 pdfkeywords={},
 pdfsubject={},
 pdfcreator={Emacs 26.3 (Org mode 9.3.7)}, 
 pdflang={Brazilian}}
\begin{document}

\maketitle
\tableofcontents

\newpage

\section{Questão 1 (Lista 6 Q2)}
\label{sec:orgf8a6e15}
\paragraph{}Primeiro vamos derivar em x e em y a função dada:

\(\displaystyle\\
  \frac{\partial{f}}{\partial{x}} = \frac{x^2}{(x^3+y^3)^{\frac{2}{3}}}\\
  \frac{\partial{f}}{\partial{y}} = \frac{y^2}{(x^3+y^3)^{\frac{2}{3}}}\)


Como as derivadas parciais são um polinômio sobre a
raiz de um outro polinômio, sabemos que elas são contínuas.
O único ponto que pode nos causar alguma dúvida é
o ponto \(x = -y\), pois isso faria com que o denominador
fosse 0, logo vamos tratar esse caso a parte, derivando
pela definição. Portanto, assumindo que \(x_0 = -y_0\), temos: 

\(\displaystyle
  \frac{\partial{f}}{\partial{x}}(x_0, y_0) = 
  \lim_{x \to x_0}{\frac{f(x, y_0)-f(x_0, y_0)}{x-x_0}} =
  \lim_{x \to x_0}{\frac{(x^3+y_0^3)^{\frac{1}{3}}-(x_0^3+y_0^3)^{\frac{1}{3}}}{x-x_0}} = 
  \lim_{x \to x_0}{\frac{(x^3+y_0^3)^{\frac{1}{3}}}{x-x_0}} = \\
  = \lim_{x \to x_0}{\left(\frac{x^3-x_0^3}{(x-x_0)^3}\right)^{\frac{1}{3}}} = 
  \left(\lim_{x \to x_0}{\frac{x^3-x_0^3}{(x-x_0)^3}}\right)^{\frac{1}{3}} \xrightarrow{\text{L'Hopital}}
  \left(\lim_{x \to x_0}{\frac{3x^2}{3(x-x_0)^2}}\right)^{\frac{1}{3}} = +\infty\)

Só podemos fazer esse último passo se \(x_0 \neq 0\).
Nesse caso, teríamos:

\(\displaystyle\left(\lim_{x \to x_0}{\frac{3x^2}{3(x-x_0)^2}}\right)^{\frac{1}{3}} \xrightarrow{x_0 = 0} 
  \left(\lim_{x \to 0}{\frac{3x^2}{3x^2}}\right) = 1\)

Com isso, falta tratar apenas o caso onde \(x_0 = y_0 = 0\),
para isso, iremos usar o limite:

\(\displaystyle\lim_{(x, y) \to (0, 0)}{\frac{f(x, y)-f(0, 0)-\frac{\partial{f}}{\partial{x}}(0, 0)(x)-\frac{\partial{f}}{\partial{y}}(0, 0)(y)}{\sqrt[2]{(x)^2+(y)^2}}}\).

Se esse limite der 0, temos que a função é diferenciável em \((0, 0)\).
Vamos simplificar o limite usando nossas observações.
Sabemos que no ponto \((0, 0)\) a derivada parcial em x
é 1, e como a derivada parcial em y é análoga, ela também é.
Sabemos também, que \(f(0, 0) = 0\) e com isso em mente nosso
limite se simplifica para:

\(\displaystyle\lim_{(x, y) \to (0, 0)}{\frac{\sqrt[3]{x^3+y^3}-x-y}{\sqrt[2]{x^2+y^2}}}\)

Vamos mostrar que esse limite não existe, mostrando que
seu valor difere caso usemos duas curvas diferentes para
aproximar o ponto:

\begin{itemize}
\item Usando a curva \((0, t)\), temos:

\(\displaystyle\lim_{t \to 0}{\frac{\sqrt[3]{0^3+t^3}-0-t}{\sqrt[2]{0^2+t^2}}} =
    \lim_{t \to 0}{\frac{t-t}{|t|}} = 0\)

\item Usando a curva \((t, t)\), temos:

\(\displaystyle\lim_{t \to 0}{\frac{\sqrt[3]{t^3+t^3}-t-t}{\sqrt[2]{t^2+t^2}}} =
    \lim_{t \to 0}{\frac{\sqrt[3]{2}t-2t}{|t|}} \neq 0\).
\end{itemize}

Provamos então que a função não é diferenciável em \((0, 0)\).

Com isso, terminamos a prova do exercício,
já que demonstramos que a função é diferenciável em
todo ponto onde \(x \neq y\) e, seu gradiente nesses pontos é:  

\(\displaystyle\nabla{f(x_0, y_0)} = \left(\frac{x_0^2}{(x_0^3+y_0^3)^{\frac{2}{3}}}, \frac{y_0^2}{(x_0^3+y_0^3)^{\frac{2}{3}}}\right)\)

\newpage
\section{Questão 2 (Lista 6 Q15)}
\label{sec:org1b33fa6}
\paragraph{}Primeiro vamos montar a Jacobiana da função T.\\

\(\displaystyle\begin{bmatrix}
  \frac{\partial{u}}{\partial{x}} & \frac{\partial{u}}{\partial{y}} \\
  \frac{\partial{v}}{\partial{x}} & \frac{\partial{v}}{\partial{y}}
  \end{bmatrix} = 
  \begin{bmatrix}
  \cos{(x+y)} & \cos{(x+y)}\\
  -\sin{(x+y)} & -\sin{(x+y)}
  \end{bmatrix} = J(T)\)



Tirando o determinante da Jacobiana, conseguimos o
Jacobiano:

\(det(J(T)) = -\sin{(x+y)\cos{(x+y)}} + \sin{(x+y)}\cos{(x+y)} = 0\)

Agora, precisamos achar em que pontos a 
função é injetora localmente.

Para isso vamos supor uma bola aberta de raio \(R > 0\)
ao redor de um ponto genérico \(P_0 = (x_0, y_0)\). Agora,
escolhemos um número \(h\) tal que \(0< h <R\). Agora
suponhamos o ponto \(P_1 = (x_0+h, y_0-h)\).
Os dois pontos, \(P_0, P_1\), levam ao mesmo valor da função T.
Desse modo, provamos que para qualquer ponto, não existe
uma bola aberta ao seu redor, tal que a função seja
injetora dentro dessa bola, logo, a função nunca é injetora localmente.
\section{Questão 3 (Lista 7 Q17)}
\label{sec:orgd620d4e}
Seja \(f\) uma função de \(\mathbb{R}^6\to\mathbb{R}^3\), 
tal que:\\
\(f(x, y, z, u, v, w) = (3x+2y+z^2+u+v^2, 4x+3y+z+u^2+w+2, x+z+w+u^2+2) = (f_1, f_2, f_3)\)
Vamos escrever a parte da Jacobiana de \(f\) que é necessária
para usarmos o Teorema da Função Implícita:

\(\displaystyle M = 
  \begin{bmatrix}
  \frac{\partial{f_1}}{\partial{u}} & \frac{\partial{f_1}}{\partial{v}} & \frac{\partial{f_1}}{\partial{w}} \\
  \frac{\partial{f_2}}{\partial{u}} & \frac{\partial{f_2}}{\partial{v}} & \frac{\partial{f_2}}{\partial{w}} \\
  \frac{\partial{f_3}}{\partial{u}} & \frac{\partial{f_3}}{\partial{v}} & \frac{\partial{f_3}}{\partial{w}}
  \end{bmatrix} = 
  \begin{bmatrix}
  1 & 2v & 0 \\
  2u & 1 & 1 \\
  2u & 0 & 0 
  \end{bmatrix}\)

Tirando seu determinante, temos:

\(\det{M} = 1 + 4uv + 0 - 0 - 4uv - 0 = 1\)

Já que o resultado foi diferente de 0, agora precisamos
apenas mostrar que as expressões dão 0 nas condições
dadas, ou seja, 
\((x, y, z) = (0, 0, 0)\), \((u, v) = (0, 0)\) e \(w = -2\).

\begin{itemize}
\item \(3x+2y+z^2+u+v^2 = 0\)
\item \(4x+3y+z+u^2+w+2 = -2+2 = 0\)
\item \(x+z+w+u^2+2 = -2+2 = 0\)
\end{itemize}

Com isso, podemos responder o a pergunta feita no
enunciado. Sim, é possível solucionar as equações da
maneira pedida.

\newpage
\section{Questão 4 (Lista 7 Q18b)}
\label{sec:org539618f}
\paragraph{}Primeiro, vamos conseguir o gradiente da função \(f\):

\(\displaystyle\nabla{f(x, y, z)} = (2xy^2z^2, 2x^2yz^2, 2x^2y^2z)\)

Agora, seja \(g(x, y, z)\) uma função tal que
\(g(x, y, z) = x^2+y^2+z^2-1\). Podemos perceber que sua
curva de nível \(C = 0\) é a restrição dada. Seu gradiente
é:

\(\displaystyle\nabla{g(x, y, z)} = (2x, 2y, 2z)\)

Agora, usando os mutiplicadores de Lagrange, temos que:

\((2xy^2z^2, 2x^2yz^2, 2x^2y^2z) = \lambda (2x, 2y, 2z)\) \(\rightarrow\) \\
\(\rightarrow\)
\(\begin{cases}
  2x\lambda = 2xy^2z^2 \\
  2y\lambda = 2x^2yz^2 \\
  2z\lambda = 2x^2y^2z 
  \end{cases}\)

Quando \(x\), \(y\) e \(z\) são diferentes de 0, podemos escrever:

\(\begin{cases}
  \lambda = y^2z^2 \\
  \lambda = x^2z^2 \\
  \lambda = x^2y^2 
  \end{cases}\) \(\rightarrow\) \(x^2=y^2=z^2\)

Aplicando isso na condição inicial:

\(x^2+y^2+z^2 = 3x^2 = 1 \rightarrow 
  x = \pm \frac{\sqrt{3}}{3}, 
  y = \pm \frac{\sqrt{3}}{3}, 
  z = \pm \frac{\sqrt{3}}{3}\)

Qualquer combinação dos possíveis valores de \(x\), \(y\) e \(z\)
aplicada na função \(f\) dará o mesmo resultado:

\(\displaystyle f(\frac{\sqrt{3}}{3}, \frac{\sqrt{3}}{3}, \frac{\sqrt{3}}{3}) = \frac{1}{27}\)

Esse é o máximo da função nas restrições dadas.

Agora para o mínimo, precisamos perceber duas coisas.

\begin{enumerate}
\item A função \(f\) é sempre positiva, seu valor mínimo
possível é 0.
\item Se uma das cordenadas for 0, o valor da função é 0.
\end{enumerate}

Com isso em mente, fica muito fácil perceber que o mínimo
na restrição dada é 0, já que, por exemplo, o ponto \((0, 0, 1)\)
está na restrição.
\end{document}