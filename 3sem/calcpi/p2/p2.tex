\documentclass[12pt]{article}
\usepackage[utf8]{inputenc}
\usepackage{amsfonts}
\usepackage{amsmath}
\usepackage{siunitx}
\usepackage{amssymb}
\usepackage{enumitem}
\usepackage{mathtools}
\usepackage[brazil]{babel}
\usepackage{geometry}
\usepackage{graphicx}
\usepackage{bussproofs}
\usepackage[table]{xcolor}
\usepackage{gensymb}
\usepackage{hyperref}
\graphicspath{{./images}}
\geometry{verbose,a4paper,left=2cm,top=2cm,right=3cm,bottom=3cm}
\title{P2 - Funções diferenciáveis e séries}
\author{Lourenço Henrique Moinheiro Martins Sborz Bogo - 11208005}
\date{}
\linespread{1.5}
\newcommand{\real}{\mathbb{R}}
\newcommand{\product}[3]{\displaystyle\prod_{#1}^#2 #3}
\newcommand{\gsum}[3]{\displaystyle\sum_{#1}^#2 #3}
\newcommand{\mytitle}[1]{\textbf{\underline{#1}}}
\newcommand{\ring}[1]{\langle #1 \rangle}
\newcommand{\code}[1]{\mbox{\texttt{#1}}}

\begin{document}

\maketitle

\mytitle{Questão 1}

Preciso obter a expressão da série $\frac{x^4}{4} + \frac{x^8}{8} + \frac{x^{12}}{12} + \dots$.

Vamos chamar a série de $f(x)$. Agora, precisaremos derivar essa série:

$f(x) = \frac{x^4}{4} + \frac{x^8}{8} + \frac{x^{12}}{12} + \dots$

$f'(x) = x^3 + x^7 + x^{11} + \dots$

É fácil notar que essa $f'(x)$ é uma P.G. de termo inicial $x^3$ e razão $x^4$. Logo podemos escrever ela da seguinte maneira:

$f'(X) = \frac{x^3}{1-x^4}$, com convergência para $-1<x<1$.

Queremos a expressão de $f(x)$, logo precisamos integrar $f'(x)$.

Iremos fazer a seguinte substituição:

$u = 1-x^4 \rightarrow du = -4x^3 \rightarrow x^3 = \frac{-du}{4}$.

Nossa integral então fica:

$f(x) = \frac{-1}{x^4}\int\frac{1}{u}du = \frac{-1}{4}\ln{u}+c = \frac{-1}{4}\ln{(1-x^4)}+c$, com convergência para $-1<x<1$.

Agora precisamos descobrir a constante. Para isso iremos usar a fórmula para calcular $\frac{-1}{4}\ln{1}$ e para isso queremos $x = 0$:

$\frac{-1}{4}\ln{1 - 0^4}+c= \frac{-1}{4}\ln{1}+c= \frac{0^4}{4} + \frac{0^8}{8} + \frac{0^{12}}{12} + \dots \rightarrow 0+c = 0 \rightarrow c = 0$.

Nossa resposta então, é $\frac{x^4}{4} + \frac{x^8}{8} + \frac{x^{12}}{12} + \dots = \frac{-1}{4}\ln{(1-x^4)}$.

\mytitle{Questão 2}

Precisamos achar o valor de $\int_0^1x^2e^{-x^2}dx$ com um erro menor que $0.01$.

Começaremos primeiro com a série de taylor de $e^z$ ao redor de $x=0$.

Todas as derivadas serão $1$, logo temos a seguinte série:

$e^z = 1+z+\frac{z^2}{2!}+\frac{z^3}{3!} + \dots$. Vamos mostrar que converge para qualquer z:

$\displaystyle\lim_{n \to \infty}\frac{a_{n+1}}{a_n} = \lim_{n \to \infty}\frac{z^{n+1}}{(n+1)!}\frac{n!}{z^n} = \lim_{n \to \infty}\frac{z}{n+1} = 0$, $\forall z \in \real$.

Como isso vale para qualquer $z$, podemos trocá-lo por $-x^2$, conseguindo a série de taylor de $e^{-x^2}$.

Multiplicamos então a série inteira por $x^2$, conseguindo então a série de $x^2e^{-x^2}$:

$x^2e^{-x^2} = x^2-x^4+\frac{x^6}{2!}-\frac{x^8}{3!} \dots = t(x)$

Queremos então, a integral dessa série, de 0 até 1, com erro menor que 0.01, e como a série é alternada, para conseguir esse erro, precisamos simplesmente parar no termo anterior ao que fica menor que 0.01.

$\int_0^1 t(x)dx = \frac{1^3}{3} - \frac{1^5}{5} + \frac{1^7}{14} - \frac{1^9}{54}$, paramos aqui, pois o próximo termo é $\frac{1^{11}}{269} < \frac{1}{100}$. Nossa soma é aproximadamente $0.18$.

\mytitle{Questão 3}

Queremos mostrar, usando séries de taylor, que:

$\displaystyle\lim_{x \to 1}x^{\frac{1}{1-x}} = \frac{1}{e}$.

Primeiro, iremos fazer a seguinte substituição $h = 1-x$, e também iremos escrever a expressão na forma de $\ln$:

$\displaystyle\lim_{x \to 1}x^{\frac{1}{1-x}} = \lim_{h \to 0}(1-h)^{\frac{1}{h}} = \lim_{h \to 0}e^{\ln{(1-h)^{\frac{1}{h}}}} = \frac{1}{e}$.

Portanto, temos que:

$\displaystyle\lim_{h \to 0}\ln{(1-h)^{\frac{1}{h}}} = -1 \rightarrow \lim_{h \to 0}\frac{1}{h}\ln(1-h)$.

Agora iremos usar a série de taylor de $\ln{(1-h)}$:

$f(x) = \ln{(1-x)} \rightarrow f'(x) = \frac{-1}{1-x} = \gsum{n=1}{\infty}{-x^{n-1}} \rightarrow f(x) = \int f'(x)dx = \gsum{n=1}{\infty}{\frac{=x^{n-1}}{n}}+c$, $-1<x<1$. Para provar que a constante é 0, é só calcular $\ln{1}$, ou seja, a série para $x = 0$.

Logo, voltando para nosso limite, temos que:

$\displaystyle\lim_{h \to 0}\frac{1}{h}\ln(1-h) = \lim_{h \to 0}\frac{1}{h}\gsum{n=1}{\infty}{\frac{-h^n}{n}} = \lim_{h \to 0}\gsum{n=1}{\infty}{\frac{-h^{n-1}}{n}} = \lim_{h \to 0}-h^0 = -1$.

Concluímos então que:

$\displaystyle\lim_{h \to 0}(1-h)^{\frac{1}{h}} = \lim_{h \to 0}e^{\ln{(1-h)^{\frac{1}{h}}}} = e^{-1} = \frac{1}{e}$.

\mytitle{Questão 4}

Primeiro vamos provar que a função $f_n$ converge uniformemente usando o Critério "M" de Weierstrass. Isso não é muito difícil de mostrar, pois:

Como nossos $"M_n"$, usaremos a sequência $c_n$. Então, temos que mostrar que nossa $f_n(x) <= c_n$ e que $\gsum{n=0}{\infty}{c_n} < \infty$. Nos foi dado que $\gsum{n=0}{\infty}{c_n} < \infty$, logo precisamos apenas provar que $f_n(x) <= c_n$:

$f_n(x) = \gsum{i=1}{n}{c_iI(x-x_i)}$, e como $I(x-x_i) <= 1$, temos que $f_n(x) <= c_n$.

Agora precisamos provar que a função $f(x)$ é contínua. Vamos separar isso em dois casos.

\begin{enumerate}
  
\item A sequência $x_n$ não converge para um certo ponto $z \in [a, b]$

  Nesse caso, existe um certa vizinhança ao redor de $z$ na qual, para qualquer $y$ o seguinte vale:

  $f(z) = f(y)$. Desse modo, temos que $f(z) - f(y) < \epsilon$, $\forall \epsilon \in \real-{0}$. Logo $f$ é contínua em $[a, b]$.

  
\item Agora temos que a sequência $x_n$  converge para um $z \in [a, b]$. Sabemos por convergência uniforme que $(\forall \epsilon > 0) (\forall x \in [a, b]) (\exists n_0 \in \mathbb{N}) (\forall n > n_0) |f_n(x) - f(x)| < \epsilon$. Vamos pegar então o menor $n$ possível, ou seja $n_0+1$. Agora vamos definir $k_i = z - x_i$, $i = 0,\dots ,n_0+1 = n$. Vamos pegar o mínimo dos $k_i$ e chamar de $k$. Essa era a vizinhança que precisávamos. De $x-k$ até $x+k$, $f_n(z) = f_n(y),\: \forall y \in ]z-k, z+k[$. Sabemos também que se $f_n$  converge uniformemente para $f$ e $f_n$ é contínua, $f$ também é, que é exatamente a situação que temos. Logo $f$ é contínua.

\end{enumerate}

Provamos então que a $f$ é contínua para os dois casos, mostrando que a f é contínua em $[a, b]$.

\end{document}