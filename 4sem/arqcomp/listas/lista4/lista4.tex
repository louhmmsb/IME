% Created 2020-11-10 ter 15:58
% Intended LaTeX compiler: pdflatex
\documentclass[11pt]{article}
\usepackage[utf8]{inputenc}
\usepackage[T1]{fontenc}
\usepackage{graphicx}
\usepackage{grffile}
\usepackage{longtable}
\usepackage{wrapfig}
\usepackage{rotating}
\usepackage[normalem]{ulem}
\usepackage{amsmath}
\usepackage{textcomp}
\usepackage{amssymb}
\usepackage{capt-of}
\usepackage{hyperref}
\usepackage{minted}
\usepackage[hyperref, x11names]{xcolor}
\hypersetup{colorlinks = true, urlcolor = SteelBlue4, linkcolor = black}
\usepackage[brazilian]{babel}
\usepackage{geometry}
\geometry{verbose,a4paper,left=2cm,top=2cm,right=3cm,bottom=3cm}
\author{Lourenço Bogo - 11208005}
\date{\today}
\title{Arquitetura de Computadores Lista 4}
\hypersetup{
 pdfauthor={Lourenço Bogo - 11208005},
 pdftitle={Arquitetura de Computadores Lista 4},
 pdfkeywords={},
 pdfsubject={},
 pdfcreator={Emacs 27.1 (Org mode 9.4)}, 
 pdflang={Brazilian}}
\begin{document}

\maketitle

\section{Questão 1}
\label{sec:orgb668271}
$x_{1}x_{2}x_{3}x_{4}x_{5}x_{6}x_{7}x_{7}x_{9}x_{10}x_{11} = x_{1}x_{2}1x_{4}100x_{8}101$
\begin{center}
\begin{tabular}{c | c c c c}
1 a 11 em binário & 8 & 4 & 2 & 1\\
\hline
1 & 0 & 0 & 0 & 1\\
2 & 0 & 0 & 1 & 0\\
3 & 0 & 0 & 1 & 1\\
4 & 0 & 1 & 0 & 0\\
5 & 0 & 1 & 0 & 1\\
6 & 0 & 1 & 1 & 0\\
7 & 0 & 1 & 1 & 1\\
8 & 1 & 0 & 0 & 0\\
9 & 1 & 0 & 0 & 1\\
10 & 1 & 0 & 1 & 0\\
11 & 1 & 0 & 1 & 1\\
\end{tabular}
\end{center}

\noindent$x_{8} = 1 \oplus 0 \oplus 1 = 0$\newline
$x_{4} = 1 \oplus 0 \oplus 0 = 1$\newline
$x_{2} = 1 \oplus 0 \oplus 0 \oplus 0 \oplus 1 = 0$\newline
$x_{1} = 1 \oplus 1 \oplus 0 \oplus 1 \oplus 1 = 0$\newline
Portanto, temos
$00111000101$

\section{Questão 2}
\label{sec:orge00684b}
Para detectar algum erro, vamos calcular os bits adicionais:\\
$x_{1} \oplus x_{3} \oplus x_{5} \oplus x_{7} \oplus x_{9} \oplus x_{11} = 1$ \\
$x_{2} \oplus x_{3} \oplus x_{6} \oplus x_{7} \oplus x_{10} \oplus x_{11} = 0$ \\
$x_{4} \oplus x_{5} \oplus x_{6} \oplus x_{7} = 1$ \\
$x_{8} \oplus x_{9} \oplus x_{10} \oplus x_{11} = 0$ \\
Temos paridade ímpar em dois casos
As posições cujo bit é 1 são a 3, a 4, a 9 e a 11. Se tirarmos o `xor` de todas, temos:

$0011 \oplus 0100 \oplus 1001 \oplus 1011 = 0101$

Isso significa quee o erro está na posição \(0101\), ou seja, posição 5. Na mensagem que recebemos
o Bit 5 está 0, logo, na mensagem correta o Bit 5 é um 1.
\end{document}