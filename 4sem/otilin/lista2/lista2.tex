% Created 2020-12-08 ter 16:38
% Intended LaTeX compiler: pdflatex
\documentclass[11pt]{article}
\usepackage[utf8]{inputenc}
\usepackage[T1]{fontenc}
\usepackage{graphicx}
\usepackage{grffile}
\usepackage{longtable}
\usepackage{wrapfig}
\usepackage{rotating}
\usepackage[normalem]{ulem}
\usepackage{amsmath}
\usepackage{textcomp}
\usepackage{amssymb}
\usepackage{capt-of}
\usepackage{hyperref}
\usepackage{minted}
\usepackage[hyperref, x11names]{xcolor}
\hypersetup{colorlinks = true, urlcolor = SteelBlue4, linkcolor = black}
\usepackage[brazilian]{babel}
\usepackage{geometry}
\geometry{verbose,a4paper,left=2cm,top=2cm,right=3cm,bottom=3cm}
\AtBeginEnvironment{align*}{\setcounter{equation}{0}}
\usepackage{pgfplots}
\usepackage{tikz}
\pgfplotsset{width=10cm, compat=1.9}
\definecolor{marsela}{RGB}{165, 42, 42}
\author{Lourenço Bogo - 11208005}
\date{\today}
\title{Otimização Linear Lista 2}
\hypersetup{
 pdfauthor={Lourenço Bogo - 11208005},
 pdftitle={Otimização Linear Lista 2},
 pdfkeywords={},
 pdfsubject={},
 pdfcreator={Emacs 27.1 (Org mode 9.5)}, 
 pdflang={Brazilian}}
\begin{document}

\maketitle

\section{Questão 1}
\label{sec:orgc836cd8}
\paragraph{} Primeiro vamos colocar o problema na forma canônica e adicionar as variáveis artificiais:
\begin{align*}
  \max Z = -A_{1} -A_{2}\\
  -1x_{1}+1x_{2}-1S_{1}+0S_{2}+0S_{3}+1A_{1}+0A_{2} = 1\\
  1x_{1}+1x_{2}+0S_{1}-1S_{2}+0S_{3}+0A_{1}+1A_{2} = 3\\
  2x_{1}+1x_{2}+0S_{1}+0S_{2}+1S_{3}+0A_{1}+0A_{2} = 4\\
  x_{1},x_{2},S_{1},S_{2},S_{3},A_{1},A_{2} \geq 0
\end{align*}
Em forma de tableaux temos:

\begin{center}
\begin{tabular}{c c c c c c c c c c c}
Iteração 1 &  & \(C_j\) & 0 & 0 & 0 & 0 & 0 & -1 & -1 & Razão\\
B & \(C_b\) & \(X_b\) & \(x_1\) & \(x_2\) & \(S_1\) & \(S_2\) & \(S_3\) & \(A_1\) & \(A_2\) & \(\frac{X_b}{x_2}\)\\
\(A_1\) & -1 & 1 & -1 & 1 & -1 & 0 & 0 & 1 & 0 & 1\\
\(A_2\) & -1 & 3 & 1 & 1 & 0 & -1 & 0 & 0 & 1 & 3\\
\(S_3\) & 0 & 4 & 2 & 1 & 0 & 0 & 1 & 0 & 0 & 4\\
Z = -4 &  & \(Z_j\) & 0 & -2 & 1 & 1 & 0 & -1 & -1 & \\
 &  & \(Z_j-C_j\) & 0 & -2 & 1 & 1 & 0 & 0 & 0 & \\
\end{tabular}
\end{center}

Podemos perceber que a variável que vai entrar é \(x_2\). Tirando a razão, vemos que a variável que entra é \(A_1\), ou seja, podemos excluí-la:

Agora precisamos montar a próxima iteração, como nosso pivô era 1, nossa linha 1 ficará igual e as outras serão subtraídas pelo seu valor, multiplicadas pelo valor da variável que está
entrando na sua linha.

\begin{center}
\begin{tabular}{c c c c c c c c c c}
Iteração 2 &  & \(C_j\) & 0 & 0 & 0 & 0 & 0 & -1 & Razão\\
B & \(C_b\) & \(X_b\) & \(x_1\) & \(x_2\) & \(S_1\) & \(S_2\) & \(S_3\) & \(A_2\) & \(\frac{X_b}{x_1}\)\\
\(x_2\) & 0 & 1 & -1 & 1 & -1 & 0 & 0 & 0 & \\
\(A_2\) & -1 & 2 & 2 & 0 & 1 & -1 & 0 & 1 & 1\\
\(S_3\) & 0 & 3 & 3 & 0 & 1 & 0 & 1 & 0 & 1\\
Z = -2 &  & \(Z_j\) & -2 & 0 & -1 & 1 & 0 & -1 & \\
 &  & \(Z_j-C_j\) & -2 & 0 & -1 & 1 & 0 & 0 & \\
\end{tabular}
\end{center}

A variável que entra agora é \(x_1\) e a que sairá será \(A_2\):
\begin{center}
\begin{tabular}{c c c c c c c c}
Iteração 3 &  & \(C_j\) & 0 & 0 & 0 & 0 & 0\\
B & \(C_b\) & \(X_b\) & \(x_1\) & \(x_2\) & \(S_1\) & \(S_2\) & \(S_3\)\\
\(x_2\) & 0 & 2 & 0 & 1 & -0.5 & -0.5 & 0\\
\(x_1\) & 0 & 1 & 1 & 0 & 0.5 & -0.5 & 0\\
\(S_3\) & 0 & 0 & 0 & 0 & -0.5 & 1.5 & 1\\
Z = 0 &  & \(Z_j\) & 0 & 0 & 0 & 0 & 0\\
 &  & \(Z_j-C_j\) & 0 & 0 & 0 & 0 & 0\\
\end{tabular}
\end{center}

Conseguimos fazer todos os \(Z_j-C_j\) serem maiores ou iguais à 0, logo a solução ótima foi atingida com \(x_1=1,x_2=2\).

Agora vamos para a fase 2, com nossa solução inicial sendo a solução que conseguimos na fase 1. Voltaremos a usar a função objetiva original.

\begin{center}
\begin{tabular}{c c c c c c c c c}
Iteração 1 &  & \(C_j\) & 3 & 1 & 0 & 0 & 0 & Razão\\
B & \(C_b\) & \(X_b\) & \(x_1\) & \(x_2\) & \(S_1\) & \(S_2\) & \(S_3\) & \(\frac{X_b}{S_2}\)\\
\(x_2\) & 1 & 2 & 0 & 1 & -0.5 & -0.5 & 0 & \\
\(x_1\) & 3 & 1 & 1 & 0 & 0.5 & -0.5 & 0 & \\
\(S_3\) & 0 & 0 & 0 & 0 & -0.5 & 1.5 & 1 & \(\frac{0}{1.5} = 0\)\\
Z = 5 &  & \(Z_j\) & 3 & 1 & 1 & -2 & 0 & \\
 &  & \(Z_j-C_j\) & 0 & 0 & 1 & -2 & 0 & \\
\end{tabular}
\end{center}

A variável que entra é \(S_2\) e a que sai é \(S_3\), nos dando:

\begin{center}
\begin{tabular}{c c c c c c c c}
Iteração 2 &  & \(C_j\) & 3 & 1 & 0 & 0 & 0\\
B & \(C_b\) & \(X_b\) & \(x_1\) & \(x_2\) & \(S_1\) & \(S_2\) & \(S_3\)\\
\(x_2\) & 1 & 2 & 0 & 1 & \(-\frac{2}{3}\) & 0 & \(\frac{1}{3}\)\\
\(x_1\) & 3 & 1 & 1 & 0 & \(\frac{1}{3}\) & 0 & \(\frac{1}{3}\)\\
\(S_2\) & 0 & 0 & 0 & 0 & \(-\frac{1}{3}\) & 1 & \(\frac{2}{3}\)\\
Z = 5 &  & \(Z_j\) & 3 & 1 & \(\frac{1}{3}\) & 0 & \(\frac{4}{3}\)\\
 &  & \(Z_j-C_j\) & 0 & 0 & \(\frac{1}{3}\) & 0 & \(\frac{4}{3}\)\\
\end{tabular}
\end{center}

Como todas as diferenças são novamente positivas, chegamos no ponto ótimo que é \(x_1=1, x_2=2\), com \(\max Z = 5\)

\section{Questao 2}
\label{sec:org87e2167}
Primeiro, colocando o problema na forma canônica e adicionando as variáveis artificiais temos:

\begin{align*}
  \max Z = -A_{1} -A_{2}\\
  -1x_{1}+1x_{2}-1S_{1}+0S_{2}+0S_{3}+1A_{1}+0A_{2} = 1\\
  1x_{1}+1x_{2}+0S_{1}-1S_{2}+0S_{3}+0A_{1}+1A_{2} = 3\\
  2x_{1}+1x_{2}+0S_{1}+0S_{2}+1S_{3}+0A_{1}+0A_{2} = 2\\
  x_{1},x_{2},S_{1},S_{2},S_{3},A_{1},A_{2} \geq 0
\end{align*}

Na forma de tableaux:

\begin{center}
\begin{tabular}{c c c c c c c c c c c}
Iteração 1 &  & \(C_j\) & 0 & 0 & 0 & 0 & 0 & -1 & -1 & Razão\\
B & \(C_b\) & \(X_b\) & \(x_1\) & \(x_2\) & \(S_1\) & \(S_2\) & \(S_3\) & \(A_1\) & \(A_2\) & \(\frac{X_b}{x_2}\)\\
\(A_1\) & -1 & 1 & -1 & 1 & -1 & 0 & 0 & 1 & 0 & 1\\
\(A_2\) & -1 & 3 & 1 & 1 & 0 & -1 & 0 & 0 & 1 & 3\\
\(S_3\) & 0 & 2 & 2 & 1 & 0 & 0 & 1 & 0 & 0 & 2\\
Z = -4 &  & \(Z_j\) & 0 & -2 & 1 & 1 & 0 & -1 & -1 & \\
 &  & \(Z_j-C_j\) & 0 & -2 & 1 & 1 & 0 & 0 & 0 & \\
\end{tabular}
\end{center}

A variável que entra é a \(x_2\) e a que sai é a \(A_1\), nos dando:

\begin{center}
\begin{tabular}{c c c c c c c c c c}
Iteração 2 &  & \(C_j\) & 0 & 0 & 0 & 0 & 0 & -1 & Razão\\
B & \(C_b\) & \(X_b\) & \(x_1\) & \(x_2\) & \(S_1\) & \(S_2\) & \(S_3\) & \(A_2\) & \(\frac{X_b}{x_2}\)\\
\(x_2\) & 0 & 1 & -1 & 1 & -1 & 0 & 0 & 0 & \\
\(A_2\) & -1 & 2 & 2 & 0 & 1 & -1 & 0 & 1 & 1\\
\(S_3\) & 0 & 1 & 3 & 0 & 1 & 0 & 1 & 0 & \(\frac{1}{3}\)\\
Z = -2 &  & \(Z_j\) & -2 & 0 & -1 & 1 & 0 & -1 & \\
 &  & \(Z_j-C_j\) & -2 & 0 & -1 & 1 & 0 & 0 & \\
\end{tabular}
\end{center}

Agora, a variável que irá entrar é a \(x_1\) e a que irá sair é a \(S_3\):


\begin{center}
\begin{tabular}{c c c c c c c c c c}
Iteração 3 &  & \(C_j\) & 0 & 0 & 0 & 0 & 0 & -1 & Razão\\
B & \(C_b\) & \(X_b\) & \(x_1\) & \(x_2\) & \(S_1\) & \(S_2\) & \(S_3\) & \(A_2\) & \(\frac{X_b}{S_1}\)\\
\(x_2\) & 0 & \(\frac{4}{3}\) & 0 & 1 & \(-\frac{2}{3}\) & 0 & \(\frac{1}{3}\) & 0 & \\
\(A_2\) & -1 & \(\frac{4}{3}\) & 0 & 0 & \(\frac{1}{3}\) & -1 & \(-\frac{2}{3}\) & 1 & 4\\
\(x_1\) & 0 & \(\frac{1}{3}\) & 1 & 0 & \(\frac{1}{3}\) & 0 & \(\frac{1}{3}\) & 0 & 1\\
Z = \(-\frac{4}{3}\) &  & \(Z_j\) & 0 & 0 & \(-\frac{1}{3}\) & 1 & \(\frac{2}{3}\) & -1 & \\
 &  & \(Z_j-C_j\) & 0 & 0 & \(-\frac{1}{3}\) & 1 & \(\frac{2}{3}\) & 0 & \\
\end{tabular}
\end{center}

A variável que irá entrar agora é a \(S_1\) e a que irá sair é a \(x_1\), nos dando:

\begin{center}
\begin{tabular}{c c c c c c c c c}
Iteração 4 &  & \(C_j\) & 0 & 0 & 0 & 0 & 0 & -1\\
B & \(C_b\) & \(X_b\) & \(x_1\) & \(x_2\) & \(S_1\) & \(S_2\) & \(S_3\) & \(A_2\)\\
\(x_2\) & 0 & 2 & 2 & 1 & 0 & 0 & 1 & 0\\
\(A_2\) & -1 & 1 & -1 & 0 & 0 & -1 & -1 & 1\\
\(x_1\) & 0 & 1 & 3 & 0 & 1 & 0 & 1 & 0\\
Z = -1 &  & \(Z_j\) & 1 & 0 & 0 & 1 & 1 & -1\\
 &  & \(Z_j-C_j\) & 1 & 0 & 0 & 1 & 1 & 0\\
\end{tabular}
\end{center}

Como todas as diferenças estão positivas, a solução ótima é \(x_1=0, x_2=2\), porém essa solução não é factível, pois viola a restrição
\(x_1+x_2 \geq 3\) e a variável artificial \(A_2\) está na base. A fase 2, portanto, não é possível de ser feita.
\section{Questão 3}
\label{sec:org7d45a05}
\paragraph{} O problema dual é:
\begin{align*}
  \max \: &20y_{1} + 10y_{2} - 30y_{3} \\
  \text{suj. } &2y_{1} + y_{2} + 3y_{3} \geq 5\\
       &4y_{1} - y_{2} - 2y_{3} \leq 6 \\
  &7y_{1} - y_{2} - 3y_{3} = 5
    y_{1} \geq 0, y_{2} \leq 0, y_{3}\text{ irrestrito}
\end{align*}
\section{Questão 4}
\label{sec:orgaeb6648}
\paragraph{}Teorema Fraco da Dualidade: Se x é uma solução factível de um problema primal de
maximização e y é uma solução factível do dual, então
$c^{t}x \leq b^{t}y$

Demonstração:

Por hipótese, temos que \(Ax_0 \leq b\) e \(A^ty_0 \geq c\). Sejam \(u = b - Ax_0\) e \(v = A^ty_0 - c\) as folgas das restrições.

Então: \(u \geq 0\), \(v \geq 0\), \(b = u + Ax_0\) e \(A^ty_0 = v + c\).

Temos:

\begin{align*}
   b\cdot y_{0} &= y_{0}^{t}b = y_{0}^{t}(u+Ax_{0}) = y_{0}^{t}u+y_{0}^{t}Ax_{0}\\
   &= y_{0}\cdot u + {(y_{0}^{t}Ax_{0})}^{t} = y_{0}\cdot u + x_{0}^{t}(A^{t}y_{0}) \\
   &= y_{0}\cdot u + x_{0}^{t}(v+c) = y_{0}\cdot u + x_{0}\cdot v + x_{0}\cdot c
 \end{align*}

Ou seja:

$b\cdot y_{0} = y_{0}\cdot u + x_{0}\cdot v + x_{0}\cdot c$

Como \(y_0\cdot u + x_0\cdot v\) é posivito, temos que \(c\cdot x_0 \leq b\cdot y_0\)

Teorema Forte da Dualidade: Se \(x^*\) é uma solução ótima do problema primal
\begin{align*}
  \max c^{t}x\\
  \text{suj }Ax &= b\\
  x &\geq 0
\end{align*}

então o problema dual
\begin{align*}
  \max b^{t}y\\
  \text{suj }A^{t}x &= c\\
  y &\text{ irrestrito}
\end{align*}

tem solução ótima \(y^*\) com \(c^tx^* = b^ty^*\).
\section{Questão 5}
\label{sec:orge8662c0}
Primeiro vamos adcionar as variáveis de folga, nos dando o seguinte problema:
\begin{align*}
  \max z = 3x_{1} + 2x_{2} + 4x_{3}\\
  x_{1} + x_{2} + 2x_{3} + x_{4} = 4\\
  2x_{1} + 3x_{3} + x_{5} = 5\\
  2x_{1} + x_{2} + 3x_{3} + x_{6} = 7
\end{align*}

Nossa base inicial é \(X_b = (x_4, x_5, x_6) = (4, 5, 7)\), nossa matriz B é a indentidade e nosso vetor y é \((0, 0, 0)\).

Calculando os valores \(c_j-z_j = c_j-yP_j\), temos:
\begin{align*}
  c_{1} - yP_{1} = 3 - (0, 0, 0)\cdot P_{1} = 3\\
  c_{2} - yP_{2} = 2 - (0, 0, 0)\cdot P_{2} = 2\\
  c_{3} - yP_{3} = 4 - (0, 0, 0)\cdot P_{3} = 4
\end{align*}

Logo, a variável que irá entrar é a variável \(x_3\).

Agora, vamos descobrir quem sai:

$\bar{P_{j}} = B^{-1}Pj =
\begin{bmatrix}
  2 \\
  3 \\
  3
\end{bmatrix}$

Fazendo as razões entre os elementos do vetor \(X_b\) e \(P_j\) e pegando o mínimo, temos que o elemento que irá sair é o \(x_5\).

Agora para a próxima iteração, temos o novo vetor \(X_b = (x_4, x_3, x_6) = (\frac{4}{3}, \frac{5}{3}, 2)\).

Para calcular y, precisamos da inversa da nova matriz B, logo vamos usar as propriedades de matrizes Eta para conseguir isso:

Vamos falar que nossa nova matriz \(B_1\) é igual à \(B\cdot H_1\) onde \(H_1\) é uma matriz theta cuja coluna não identidade é o vetor \(\bar{P_j}\) da
iteração anterior. Com isso, temos:

\begin{align*}
  H_{1} =
  \begin{bmatrix}
	1 && 2 && 0\\
    0 && 3 && 0\\
    0 && 3 && 1
  \end{bmatrix}\\
  y = C_{b}B_{1}^{-1} \rightarrow yB_{1} = C_{b} \rightarrow yBH_{1} = C_{b}
\end{align*}

Resolvendo esse sistema, encontramos \(y = (0, \frac{4}{3}, 0)\).

Agora, fazendo o mesmo processo que fizemos antes, calculamos todos os \(C_j-z_j\) e descobrimos que quem irá entrar é o elemento \(x_2\).

Agora, precisamos calcular \(\bar{P_j}\):

\begin{align*}
  \bar{P_{j}} = B_{1}^{-1} P_{j} \rightarrow B_{1}\bar{P_{j}} = P_{j}
\end{align*}

Resolvendo esse sistema, temos \(\bar{P_j} = (1, 0, 1)\) e isso indica que o elemento que irá sair é o \(x_4\).

Agora, para a próxima iteração, iremos repetir todo o processo:

\begin{itemize}
\item Calculamos nosso novo \(X_b\) usando o pivô: \(X_b = (\frac{2}{3}, \frac{5}{3}, \frac{4}{3})\)
\item Usaremos outra matriz eta para não termos que calcular \(B^{-1}\): \(B_2 = B_1\cdot H_2 = B\cdot H_1\cdot H_2\)
\item Achamos \(y\) resolvendo o sistema \(yB_2 = C_b\), nos dando \(y = (2, 0, 0)\)
\item Usamos o \(y\) que calculamos para fazer as contas \(C_j-z_j\), e descobrimos que quem irá entrar é o elemento \(x_2\)
\item Usamos os valores coeficientes de \(x_2\) nas equações para calcular \(\bar{P_j}\), nos dando \(\bar{P_j} = (1, 2, 2)\)
\item Usamos o vetor \(\bar{P_j}\) para descobrir quem irá sair (será o que tiver a menor razão), nesse caso é o elemento \(x_3\)
\end{itemize}

Agora, continuamos para a próxima iteração, onde repetiremos o processo mais uma vez:

\begin{itemize}
\item \(X_b = (x_2, x_1, x_6) = (\frac{3}{2}, \frac{5}{2}, \frac{1}{2})\)
\item \(B_3 = B_0\cdot H_1\cdot H_2\cdot H_3\)
\item \(y = (2, \frac{1}{2}, 0)\)
\item Agora, ao calcularmos as diferenças \(C_j-z_j\), todas dão negativas, o que significa que chegamos na solução ótima.
\end{itemize}

Temos, então, que a solução ótima é \((x_1, x_2, x_3) = (\frac{5}{2}, \frac{3}{2}, 0)\) com função objetiva no valor de 10.5.

Todas as contas e manipulações algébricas estarão em uma foto que enviarei como anexo, preferi deixar o espaço aqui o mais limpo possível.

\section{Questão 6}
\label{sec:org98809e0}
Vamos usar o método simplex para achar a melhor solução para o problema.

Primeior, vamos colocar o problema na forma canônica:

\begin{align}
  \max Z = 7x_{1}+6x_{2}+5x_{3}-2x_{4}+3x_{5}
  x_{1}+3x_{2}+5x_{3}-2x_{4}+2x_{5}+S_{1} = 4
  4x_{1}+2x_{2}-2x_{3}+x_{4}+x_{5}+S_{3} = 3
  2x_{1}+4x_{2}+4x_{3}-2x_{4}+5x_{5}+S_{4} = 5
  3x_{1}+x_{2}+2x_{3}-x_{4}-2x_{5}+S_{5} = 1
\end{align}

Em forma de tableaux:

\begin{center}
\begin{tabular}{c c c c c c c c c c c c c}
Iteration 1 &  & \(C_j\) & 7 & 6 & 5 & -2 & 3 & 0 & 0 & 0 & 0 & \\
B & \(C_b\) & \(X_b\) & \(x_1\) & \(x_2\) & \(x_3\) & \(x_4\) & \(x_5\) & \(S_1\) & \(S_2\) & \(S_3\) & \(S_4\) & Min ratio \(\frac{X_b}{x_1}\)\\
\(S_1\) & 0 & 4 & 1 & 3 & 5 & -2 & 2 & 1 & 0 & 0 & 0 & \(\frac{4}{1}\)\\
\(S_2\) & 0 & 3 & 4 & 2 & -2 & 1 & 1 & 0 & 1 & 0 & 0 & \(\frac{3}{4}\)\\
\(S_3\) & 0 & 5 & 2 & 4 & 4 & -2 & 5 & 0 & 0 & 1 & 0 & \(\frac{5}{2}\)\\
\(S_4\) & 0 & 1 & 3 & 1 & 2 & -1 & -2 & 0 & 0 & 0 & 1 & \(\frac{1}{3}\)\\
\(Z=0\) &  & \(Z_j\) & 0 & 0 & 0 & 0 & 0 & 0 & 0 & 0 & 0 & \\
 &  & \(Z_j-c_j\) & -7 & -6 & -5 & 2 & -3 & 0 & 0 & 0 & 0 & \\
\end{tabular}
\end{center}

O mínimo negativo é -7, logo a variável que entra é \(x_1\) e a menor razão é a de \(S_4\) logo, ela é a variável que irá sair. O pivô é 3, logo
a linha de da variável que sai será dividida por 3, e as outras serão subtraídas do novo valor dessa linha multiplicado pelo valor da variável que está entrando, na respectiva linha.

\begin{center}
\begin{tabular}{c c c c c c c c c c c c c}
Iteration 2 &  & \(C_j\) & 7 & 6 & 5 & -2 & 3 & 0 & 0 & 0 & 0 & \\
B & \(C_b\) & \(X_b\) & \(x_1\) & \(x_2\) & \(x_3\) & \(x_4\) & \(x_5\) & \(S_1\) & \(S_2\) & \(S_3\) & \(S_4\) & Min ratio \(\frac{X_b}{x_1}\)\\
\(S_1\) & 0 & \(\frac{11}{3}\) & 0 & \(\frac{8}{3}\) & \(\frac{13}{3}\) & \(-\frac{5}{3}\) & \(\frac{8}{3}\) & 1 & 0 & 0 & \(-\frac{1}{3}\) & \(\frac{11}{8}\)\\
\(S_2\) & 0 & \(\frac{5}{3}\) & 0 & \(\frac{2}{3}\) & \(-\frac{14}{3}\) & \(\frac{7}{3}\) & \(\frac{11}{3}\) & 0 & 1 & 0 & \(-\frac{4}{3}\) & \(\frac{5}{11}\)\\
\(S_3\) & 0 & \(\frac{13}{3}\) & 0 & \(\frac{10}{3}\) & \(\frac{8}{3}\) & \(-\frac{4}{3}\) & \(\frac{19}{3}\) & 0 & 0 & 1 & \(-\frac{2}{3}\) & \(\frac{13}{19}\)\\
\(x_1\) & 7 & \(\frac{1}{3}\) & 1 & \(\frac{1}{3}\) & \(\frac{2}{3}\) & \(-\frac{1}{3}\) & \(-\frac{2}{3}\) & 0 & 0 & 0 & 1 & \\
\(Z=\frac{7}{3}\) &  & \(Z_j\) & 7 & \(\frac{7}{3}\) & \(\frac{14}{3}\) & \(-\frac{7}{3}\) & \(-\frac{14}{3}\) & 0 & 0 & 0 & \(\frac{7}{3}\) & \\
 &  & \(Z_j-c_j\) & 0 & \(-\frac{11}{3}\) & \(-\frac{1}{3}\) & \(-\frac{1}{3}\) & \(-\frac{23}{3}\) & 0 & 0 & 0 & \(\frac{7}{3}\) & \\
\end{tabular}
\end{center}

Agora a variável que entra é a \(x_5\), e tirando as razões, a que sai é a variável \(S_2\). Nosso pivô é \(\frac{11}{3}\)

\begin{center}
\begin{tabular}{c c c c c c c c c c c c c}
Iteration 3 &  & \(C_j\) & 7 & 6 & 5 & -2 & 3 & 0 & 0 & 0 & 0 & \\
B & \(C_b\) & \(X_b\) & \(x_1\) & \(x_2\) & \(x_3\) & \(x_4\) & \(x_5\) & \(S_1\) & \(S_2\) & \(S_3\) & \(S_4\) & Min ratio \(\frac{X_b}{x_1}\)\\
\(S_1\) & 0 & \(\frac{27}{11}\) & 0 & \(\frac{24}{11}\) & \(\frac{85}{11}\) & \(-\frac{37}{11}\) & 0 & 1 & \(-\frac{8}{11}\) & 0 & \(\frac{7}{11}\) & \(\frac{27}{85}\)\\
\(x_5\) & 3 & \(\frac{5}{11}\) & 0 & \(\frac{2}{11}\) & \(-\frac{14}{11}\) & \(\frac{7}{11}\) & 1 & 0 & \(\frac{3}{11}\) & 0 & \(-\frac{4}{11}\) & \\
\(S_3\) & 0 & \(\frac{16}{11}\) & 0 & \(\frac{24}{11}\) & \(\frac{118}{11}\) & \(-\frac{59}{11}\) & 0 & 0 & \(\frac{-19}{11}\) & 1 & \(\frac{18}{11}\) & \(\frac{8}{59}\)\\
\(x_1\) & 7 & \(\frac{7}{11}\) & 1 & \(\frac{5}{11}\) & \(-\frac{2}{11}\) & \(\frac{1}{11}\) & 0 & 0 & \(\frac{2}{11}\) & 0 & \(\frac{1}{11}\) & \\
\(Z=\frac{64}{11}\) &  & \(Z_j\) & 7 & \(\frac{41}{11}\) & \(-\frac{56}{11}\) & \(\frac{28}{11}\) & 3 & 0 & \(\frac{23}{11}\) & 0 & \(\frac{-5}{11}\) & \\
 &  & \(Z_j-c_j\) & 0 & \(-\frac{25}{11}\) & \(-\frac{111}{11}\) & \(\frac{50}{11}\) & 0 & 0 & \(\frac{23}{11}\) & 0 & \(-\frac{5}{11}\) & \\
\end{tabular}
\end{center}

Agora entra \(x_3\) e sai \(S_3\) com pivô \(\frac{118}{11}\):

\begin{center}
\begin{tabular}{c c c c c c c c c c c c c}
Iteration 4 &  & \(C_j\) & 7 & 6 & 5 & -2 & 3 & 0 & 0 & 0 & 0 & \\
B & \(C_b\) & \(X_b\) & \(x_1\) & \(x_2\) & \(x_3\) & \(x_4\) & \(x_5\) & \(S_1\) & \(S_2\) & \(S_3\) & \(S_4\) & Min ratio \(\frac{X_b}{x_1}\)\\
\(S_1\) & 0 & \(\frac{83}{59}\) & 0 & \(\frac{36}{59}\) & 0 & \(\frac{1}{2}\) & 0 & 1 & \(\frac{61}{118}\) & \(-\frac{85}{118}\) & \(-\frac{32}{59}\) & \(\frac{166}{59}\)\\
\(x_5\) & 3 & \(\frac{37}{59}\) & 0 & \(\frac{26}{59}\) & 0 & 0 & 1 & 0 & \(\frac{4}{59}\) & \(\frac{7}{59}\) & \(-\frac{10}{59}\) & \\
\(x_3\) & 5 & \(\frac{8}{59}\) & 0 & \(\frac{12}{59}\) & 1 & \(-\frac{1}{2}\) & 0 & 0 & \(-\frac{-19}{118}\) & \(\frac{11}{118}\) & \(\frac{9}{59}\) & \\
\(x_1\) & 7 & \(\frac{39}{59}\) & 1 & \(\frac{29}{59}\) & 0 & 0 & 0 & 0 & \(\frac{9}{59}\) & \(\frac{1}{59}\) & \(\frac{7}{59}\) & \\
\(Z=\frac{424}{59}\) &  & \(Z_j\) & 7 & \(\frac{341}{59}\) & 5 & \(-\frac{5}{2}\) & 3 & 0 & \(\frac{55}{118}\) & \(\frac{111}{118}\) & \(\frac{64}{59}\) & \\
 &  & \(Z_j-c_j\) & 0 & \(-\frac{13}{59}\) & 0 & \(-\frac{1}{2}\) & 0 & 0 & \(\frac{55}{118}\) & \(\frac{111}{118}\) & \(\frac{64}{59}\) & \\
\end{tabular}
\end{center}

Agora entra \(x_4\) e sai \(S_1\) com pivô \(\frac{1}{2}\)
\begin{center}
\begin{tabular}{c c c c c c c c c c c c c}
Iteration 5 &  & \(C_j\) & 7 & 6 & 5 & -2 & 3 & 0 & 0 & 0 & 0 & \\
B & \(C_b\) & \(X_b\) & \(x_1\) & \(x_2\) & \(x_3\) & \(x_4\) & \(x_5\) & \(S_1\) & \(S_2\) & \(S_3\) & \(S_4\) & Min ratio \(\frac{X_b}{x_1}\)\\
\(x_4\) & -2 & \(\frac{166}{59}\) & 0 & \(\frac{72}{59}\) & 0 & 1 & 0 & 2 & \(\frac{122}{118}\) & \(-\frac{85}{59}\) & \(-\frac{64}{59}\) & \\
\(x_5\) & 3 & \(\frac{37}{59}\) & 0 & \(\frac{26}{59}\) & 0 & 0 & 1 & 0 & \(\frac{4}{59}\) & \(\frac{7}{59}\) & \(-\frac{10}{59}\) & \\
\(x_3\) & 5 & \(\frac{91}{59}\) & 0 & \(\frac{48}{59}\) & 1 & 0 & 0 & 1 & \(\frac{42}{118}\) & \(-\frac{37}{59}\) & \(-\frac{23}{59}\) & \\
\(x_1\) & 7 & \(\frac{39}{59}\) & 1 & \(\frac{29}{59}\) & 0 & 0 & 0 & 0 & \(\frac{9}{59}\) & \(\frac{1}{59}\) & \(\frac{7}{59}\) & \\
\(Z=\frac{509}{59}\) &  & \(Z_j\) & 7 & \(\frac{377}{59}\) & 5 & 2 & 3 & 1 & 0.9831\(\dots\) & 0.2203\(\dots\) & 0.5424\(\dots\) & \\
 &  & \(Z_j-c_j\) & 0 & \(-\frac{23}{59}\) & 0 & 0 & 0 & 1 & 0.9831\(\dots\) & 0.2203\(\dots\) & 0.5424\(\dots\) & \\
\end{tabular}
\end{center}

Nosso simplex acaba por aqui já que todas as diferenças são positivas. Nossa função objetiva tem valor maior do que a que foi dada no enunciado, logo a solução dada no enunciado não é
a ótima.

\section{Questão 7}
\label{sec:orga334fc4}
Como temos apenas duas variáveis, vamos resolver o problema geometricamente. Primeiro vamos plotar as restrições:

\begin{tikzpicture}
  \begin{axis}[axis lines = left, xlabel = $x_{1}$, ylabel = $x_{2}$]

  \addplot[domain=0:10, samples=100, color=red,]{3*x - 1};
  \addlegendentry{$-3x_{1}+x_{2} = -1$}

  \addplot[domain=0:10, samples=100, color=green,]{x - 1};
  \addlegendentry{$x_{1}-x_{2} = 1$}

  \addplot[domain=0:10, samples=100, color=blue,]{(2*x + 6)/7};
  \addlegendentry{$-2x_{1}+7x_{2} = 6$}

  \addplot[domain=0:10, samples=100, color=marsela,]{(9*x - 6)/4};
  \addlegendentry{$9x_{1}-4x_{2} = 6$}

  \addplot[domain=0:10, samples=100, color=yellow,]{(5*x - 3)/2};
  \addlegendentry{$-5x_{1}+2x_{2} = -3$}

  \addplot[domain=0:10, samples=100, color=purple,]{(7*x - 6)/3};
  \addlegendentry{$7x_{1}-3x_{2} = 6$}
  \end{axis}
\end{tikzpicture}

Nossa região factível fica delimitada por 4 pontos:

\begin{itemize}
\item O primeiro é definido pelas restrições \(-5x_1+2x_2 \leq -3\) e \(x_2 \geq 0\). O ponto é o \((0.6, 0)\)
\item O segundo é definido pelas restrições \(9x_1-4x_2 \leq 6\) e \(x_2 \geq 0\). O ponto é o \((\frac{2}{3}, 0)\)
\item O terceiro é definido pelas restrições \(-2x_1+7x_2 \leq 6\) e \(9x_1-4x_2 \leq 6\). O ponto é o \((1.2, 1.2)\)
\item O quarto é definido pelas restrições \(-2x_1+7x_2 \leq 6\) e \(-5x_1+2x_2 \leq -3\). O ponto é o \((\frac{165}{155}, \frac{36}{31})\)
\end{itemize}

Desses pontos, o que nos da o maior valor da função objetiva é o ponto \((0.6, 0)\), ou seja, o valor máximo da função objetiva é \(-0.6\).
\end{document}