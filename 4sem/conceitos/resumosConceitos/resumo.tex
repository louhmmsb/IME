% Created 2021-01-17 Sun 16:52
% Intended LaTeX compiler: pdflatex
\documentclass[11pt]{article}
\usepackage[utf8]{inputenc}
\usepackage[T1]{fontenc}
\usepackage{graphicx}
\usepackage{grffile}
\usepackage{longtable}
\usepackage{wrapfig}
\usepackage{rotating}
\usepackage[normalem]{ulem}
\usepackage{amsmath}
\usepackage{textcomp}
\usepackage{amssymb}
\usepackage{capt-of}
\usepackage{hyperref}
\usepackage{minted}
\usepackage[hyperref, x11names]{xcolor}
\hypersetup{colorlinks = true, urlcolor = SteelBlue4, linkcolor = black}
\usepackage[brazilian]{babel}
\usepackage{geometry}
\geometry{verbose,a4paper,left=2cm,top=2cm,right=3cm,bottom=3cm}
\author{Lourenço Henrique Moinheiro Martins Sborz Bogo - 11208005}
\date{\today}
\title{Resumo Conceitos 2020}
\hypersetup{
 pdfauthor={Lourenço Henrique Moinheiro Martins Sborz Bogo - 11208005},
 pdftitle={Resumo Conceitos 2020},
 pdfkeywords={},
 pdfsubject={},
 pdfcreator={Emacs 27.1 (Org mode 9.5)}, 
 pdflang={Brazilian}}
\begin{document}

\maketitle
\tableofcontents

\newpage
\section{Funcional}
\label{sec:orgfe613b7}
\subsection{Introdução}
\label{sec:org9f7801e}

\paragraph{} Programação funcional foi o primeiro tópico que vimos na matéria.
\paragraph{} Trata-se de um paradigma onde as funções são valores, ou seja, podemos passá-las como argumentos e
atribuí-las a variáveis. Além disso, o paradigma possui outras propriedades relevantes que irei abordar em seguida.

\subsection{Propriedades Principais}
\label{sec:orga89e575}
\paragraph{} Uma parte extremamente importante de linguagens funcionais é que, em geral, os procedimentos (funções) não têm efeitos colaterais:
uma função apenas recebe os seus argumentos, executa algum tipo de operação com eles e no final retorna um novo valor.
O procedimento não altera variáveis de outro escopo, inclusive os seus argumentos, que permanecerão inalterados no ambiente do escopo no
qual o procedimento foi chamado.

\paragraph{} As funções podem ser argumentos de outras funções, o que nos conduz a um ponto crucial do paradigma funcional:
as funções de ordem superior. Essas são funções cujos parâmetros são outras funções e, quando combinadas com certas
propriedades do paradigma, nos permitem atingir um nível de abstração muito interessante nas linguagens. Com base nesse conceito
podemos criar alguns dos procedimentos mais importantes e conhecidos da programação funcional, como o \texttt{map}, o \texttt{filter} e o \texttt{reduce}.

\subsection{Lambdas e Closures}
\label{sec:org3ef84b2}
\paragraph{} Sendo as nossas funções valores, não é necessário que tenham nomes e, por essa razão, são definidas de maneira anônima. Denominamos funções anônimas como lambdas. Se quisermos dar-lhes um nome, podemos simplesmente atribuí-las a alguma variável, já que
elas são valores comuns iguais a \texttt{1} ou \texttt{"foo"}.

\paragraph{} Outro conceito é o fechamento (closure), que é um par \texttt{(lambda, ambiente)}. Quando criamos uma nova função
usando \texttt{lambda}, montamos um fechamendo com a lambda previamente definida e o ambiente atual é estendido pela variável dessa lambda.
Isso é feito por dois motivos principais:
\begin{enumerate}
\item No momento de fazer a aplicação da função definida, vamos ao seu ambiente e alteramos o valor da variável para o valor passado como argumento na aplicação.
\item Como todo ambiente de fechamento é uma extensão do ambiente no qual esse fechamento foi criado, as funções mais internas poderão acessar os argumentos das funções externas.
\end{enumerate}

\paragraph{} Vale relembrar aqui a diferença entre escopo léxico e escopo dinâmico. O que usamos em nossos trabalhos foi o escopo léxico, onde uma
variável terá o valor que lhe foi dado localmente, pois expandimos os ambientes da maneira que foi explicada acima. Já no escopo dinâmico,
todo identificador terá associada uma pilha global de valores. Quando criamos uma variável cujo nome é esse identificador, inserimos o
valor atribuído à variável na pilha do identificador. Ao avaliarmos certa variável, o valor resultante será o que estiver no
topo da pilha. O desempilhamento é feito seguindo os escopos.

\subsection{Currying}
\label{sec:org8ae8eaa}
\paragraph{} Outro ponto interessante abordado foi o conceito de \texttt{curry} que, de forma sintética e simplista, significa transformar funções
de múltiplos parâmetros em cadeias de funções com apenas um parâmetro. Exemplificando:
\begin{minted}[]{scheme}
;; Função definida sem currying
(define somaDois
  (lambda (x y)
    (+ x y)))

;; Função definida com currying
(define somaDoisCurry
  (lambda (x)
    (lambda (y)
      (+ x y))))
\end{minted}

\paragraph{} A primeira função só pode ser chamada quando passamos dois argumentos e ela somará os dois. A segunda função recebe um
argumento, devolve um fechamento, que também recebe um argumento, e só então soma os dois. Isso é útil, pois torna o código mais
reutilizável e sustentável. Por exemplo, se quisermos fazer um procedimento que recebe um argumento e acrescenta 1 a esse argumento,
poderíamos reutilizar \texttt{somaDoisCurry}, e definir a nova função mais facilmente:

\begin{minted}[]{scheme}
(define somaUm
  (somaDoisCurry 1)) ;; Retorna uma lambda de um argumento que soma 1 a esse argumento
\end{minted}

\subsection{Continuações}
\label{sec:orgb813025}
\paragraph{} Aprendemos também sobre continuações. Ao invés de termos uma função que recebe e retorna valores, adcionamos a essa função outro
parâmetro que será um procedimento. O argumento desse procedimento será o valor que a função estava retornando anteriormente. Ou seja:
\begin{itemize}
\item Fazíamos \(f(x) \rightarrow y\)
\item Passamos a fazer \(f(x, g) \rightarrow g(y)\). y aqui é o que a função retornava no caso anterior. Nesse exemplo, g é denominada a continuação de f.
\end{itemize}

\paragraph{} A grande utilidade disso é que nos permite controlar melhor o fluxo do programa e tratar mais facilmente situações não usuais, como por
exemplo, sair de recursões sem termos que voltar por todas as chamadas.

\color{red}

\subsection{Recursão}
\label{sec:orgce3d05e}
\paragraph{} Como nas linguagens funcionais não podemos ter mutações, não temos loops, portanto, fazemos ações repetidas com o auxílio de
recursão. Porém, recursão pode acabar sendo ineficiente em alguns casos, por exemplo, quando queremos calcular o fatorial de um número. O
jeito trivial de fazer isso seria:

\begin{minted}[]{scheme}
(define fatorial
  (lambda (x)
    (if (equal? 0 x) 1 (* x (fatorial (- x 1))))))
\end{minted}

Isso é ineficiente, pois podemos acabar enchendo a pilha com chamadas recursivas para fatorial, caso nossa entrada seja um inteiro muito
grande.

Podemos resolver isso usando recursão de cauda! Faríamos da seguinte maneira:

\begin{minted}[]{scheme}
(define fatorialaux
  (lambda (x acc)
    (if (equal x 0) acc (fatorial (- x 1) (* acc x)))))

(define fatorial
  (lambda (x)
    (fatorialaux x 1)))
\end{minted}

Como podemos ver, agora a nossa função \texttt{fatorialaux} não faz nada após as chamadas recursivas, ou seja, a função é uma recursão de cauda.
Isso é útil no quesito eficiência, pois os interpretadores/compiladores da grande maioria das linguagens otimizam recursões de cauda para
virarem loops, então esse código roda sem desperdiçar a pilha de execução.

\subsection{Polimorfismo}
\label{sec:orgd0824eb}
Em programação funcional temos também o conceito de polimorfismo, que é criar uma interface única para funções de vários tipos. Isso é
útil pois podemos fazer funções como o mapc que pode ser aplicada a uma lista de qualquer tipo (inclusive em linguagens tipadas como
Haskell). Isso dá uma quantidade imensa de expressividade para linguagens funcionais (fica mais evidente em linguagens fortemente
tipadas).
\subsection{Propósito}
\label{sec:org38756d0}
Todas essas propriedades fazem linguagens funcionais parecerem menos úteis que linguages procedurais, mas isso não é verdade. Programação
funcional funciona muito melhor que programação procedural/orientada a objetos para resolver problemas que têm uma modelagem matemática
precisa, pois basicamente o que fazemos nesse paradigma é encadear funções sem efeitos colaterais, assim como fazemos na matemática.

\color{black}
\section{Lazy Evaluation}
\label{sec:org01723f2}
\subsection{Introdução}
\label{sec:org40c6eb2}
\paragraph{} O segundo tópico abordado na matéria foi "Lazy Evaluation" (Avaliação por Demanda).

\paragraph{} Avaliação por demanda é uma estratégia de avaliação cujo princípio básico é: apenas calcular o valor de uma expressão quando
esse valor for necessário. É o contrário do que vínhamos aprendendo até agora, que era avaliação ansiosa, cujo princípio é avaliar uma
expressão na primeira vez que ela for encontrada.

\paragraph{} Para ilustrar a diferença entre esses dois métodos de avaliação, segue um exemplo simples:

\begin{minted}[]{scheme}
(cons (+ 1 2) '())
\end{minted}

\paragraph{} O código acima monta uma lista cujo primeiro elemento é a aplicação da função \texttt{+} nos elementos 1 e 2. Se estamos em uma linguagem onde temos
avaliação ansiosa, acontecerá o seguinte: ao encontrarmos a operação \texttt{(+ 1 2)}, iremos avaliá-la e teremos como resultado 3. Nosso código
então irá produzir uma lista com apenas 1 elemento, que será o número 3.

\paragraph{} Fica claro que não usamos o valor 3 para nada, ele não foi necessário para nenhuma operação. Para montarmos a lista, não precisávamos saber
que ao avaliarmos aquela expressão teríamos como resultado 3. Para isso que serve a avaliação por demanda. No código acima, se estivermos
em uma linguagem que implementa esse tipo de avaliação, não iremos calcular o valor da soma, iremos criar uma lista de 1 elemento, cujo
valor é uma \textbf{SUSPENSÃO}, que quando avaliada irá nos retornar o valor 3.

\subsection{Suspensões}
\label{sec:org4f2e1d0}
\paragraph{} Suspensões são uma estrutura semelhante a um fechamento sem argumentos, elas guardam uma expressão e um ambiente. Quando o
valor da suspensão for necessário, iremos interpretar a expressão dessa suspensão, com o ambiente contido nela.

\paragraph{} Por eficiência, sempre que avaliamos pela primeira vez uma suspensão específica, substituímos no ambiente global o seu valor
pelo valor retornado dessa avaliação. Assim, na próxima vez que precisarmos do valor dessa suspensão, poderemos utilizá-la sem ter que
recalcular o seu valor.

\paragraph{} Para todo esse sistema funcionar, necessitamos de um novo tipo de funções que serão chamadas \textbf{Funções Estritas}. Essas funções
irão avaliar seus argumentos imediatamente, ou seja, caso recebam uma suspensão como argumento, elas irão expandir essa suspensão.
Exemplificando:

\begin{minted}[]{scheme}
(if (equal? (+ 1 2) 3) (alguma_coisa) (outra_coisa))
\end{minted}

\paragraph{} Nesse caso, para podermos continuar o programa, é necessário que avaliemos o valor da condição do if imediatamente. Ou seja,
a condição do if é estrita, significando que ela sempre irá avaliar as suspensões dadas.

\paragraph{} Alguns outros exemplos de funções estritas são:

\begin{itemize}
\item Operações aritméticas, já que precisamos saber em quais valores estamos aplicando essa operação (não faz sentido somar duas suspensões)
\item Car e cdr, já que quando queremos um elemento de uma lista, queremos o elemento em si e não uma suspensão
\end{itemize}

\subsection{Vantages e Desvantagens}
\label{sec:org5cbc80f}
\paragraph{} Primeiro, a grande desvantagem da avaliação por demanda é a seguinte: já que uma suspensão guarda também um ambiente,
dependendo do jeito que implementarmos esse sistema, \color{red} pode ser que o custo de memória fique imprevisível, podendo ser muito alto, muito baixo ou até mesmo negativo, já que vamos evitar a avaliação de partes não usadas de estruturas recursivas \color{black}. Além disso, funções
com efeitos colaterais podem quebrar as coisas, já que podemos alterar suspensões antes de termos usado seu valor para o que queríamos.

\paragraph{} Agora, as vantagens principais desse método de avaliação são:

\begin{itemize}
\item Aumento na performance da linguagem, já que iremos evitar avaliações desnecessárias
\item Podemos ter estruturas de dados infinitas, pois só iremos calcular os elementos necessários dessa estrutura. Exemplo em haskell:
\begin{minted}[]{haskell}
fibs = 0 : 1 : zipWith (+) fibs (tail fibs)
      -- Código retirado de wiki.haskell.org/The_Fibonacci_sequence
      -- Nesse exemplo montamos uma lista que contém TODOS os números da
      -- sequência de Fibonacci. O único problema com isso é que se pedirmos algo que
      -- exige a lista toda (como o tamanho da lista), o programa irá quebrar.
\end{minted}
\end{itemize}
\section{POO}
\label{sec:orgacdf131}
\subsection{Introdução}
\label{sec:org990f2d6}
\paragraph{} O terceiro e último grande tópico abordado foi o paradigma de programação conhecido como Orientação a Objetos. O paradigma foi
inventado com o propósito de podermos abstrair nossos dados e esconder alguns detalhes de suas representações. Os códigos desse paradigma
tendem a ficar mais modularizados e intuitivos.

\paragraph{} Como já dito antes, o paradigma tem como objetivo principal a abstração dos dados, ou seja, podemos montar \texttt{classes} para
representar mais abstratamente o dado que temos. Isso é muito útil, pois nos permite organizar nossos dados com mais facilidade e de
maneira mais intuitiva.

\subsection{Objetos}
\label{sec:org92f3d6d}
\paragraph{} Tudo (ou quase tudo, dependendo da linguagem) nesse paradigma é representado por objetos. Um objeto é formado por um conjunto
de dados e um conjunto de procedimentos (métodos), que nos permite alterar esses dados e produzir valores.

\paragraph{} Isso nos conduz ao primeiro ponto principal do paradigma que é o encapsulamento. Só podemos acessar e alterar os dados de um
certo objeto através da sua interface de funções (os métodos), nos dando duas vantagens principais:

\begin{itemize}
\item Abstração, já que alteramos o estado do objeto através de métodos que podem ser extremamente complexos
\item Segurança, já que como o único jeito de alterar o estado do objeto é através dos métodos, se os métodos garantidamente sempre produzirem novos estados consistentes, não conseguiremos quebrar o programa.
\end{itemize}

\paragraph{} Outra coisa essencial do paradigma é que, como já mencionado, podemos (e devemos) alterar os estados dos objetos, ou seja,
temos efeitos colaterais, diferente do paradigma funcional, onde tínhamos idealmente apenas funções puras.

\subsection{Classes}
\label{sec:org12e6e45}
\paragraph{} Uma classe é como uma fábrica de objetos de uma certa categoria. Todos os objetos criados a partir de uma classe terão um
escopo interno com mesmos nomes e os mesmos métodos. Classes seguem uma estrutura hierárquica, ou seja, podemos definir uma classe B que é
'filha' de uma certa classe A. Nesse caso, dizemos que B está \textbf{herdando} de A, e temos as seguintes propriedades:

\begin{itemize}
\item B terá todas as variáveis de instância que A
\item B terá os mesmos métodos que A
\item B pode definir novas variáveis e métodos (incremento)
\item B pode \textbf{REDEFINIR} os métodos de A (Redefinição)
\end{itemize}

\paragraph{} Ao invés de herdar, podemos também delegar o trabalho para um objeto de outra classe, ou seja, podemos ter uma certa classe
D, que contém um objeto da classe C. Desse modo, podemos usar os métodos definidos em C usando o objeto dentro de D.

\paragraph{} Delegar é quase sempre melhor que herdar, a grande exceção pra isso é quando queremos usar a propriedade de redefinição da
herança. Nesses casos, herdar é vantajoso.

\subsection{Implementações de Herança}
\label{sec:org8925735}
\paragraph{} Em Java e Smalltalk, fazemos uma busca dinâmica pelos métodos, ou seja, quando queremos enviar uma mensagem (chamar um
método), o que fazemos é procurar pelo nome desse método na classe e nas superclasses do objeto para o qual a mensagem está sendo enviada.

\paragraph{} Já em C++, fazemos a execução direta do código, ou seja, cada objeto inclui ponteiros para cada um dos métodos de suas
classes. Isso torna C++ a linguagem orientada a objetos mais rápida que existe e que pode existir, porém tira um pouco de expressividade.

\subsection{Polimorfismo}
\label{sec:org9a7d48c}
\paragraph{} Herança nos permite o uso de uma propriedade chamada polimorfismo. Suponhamos que temos uma classe Animal. Dessa classe,
herdamos uma outra classe Cachorro e uma outra classe Gato. Cachorro e Gato podem ter implemetações diferentes para os métodos da classe
animal (usando redefinição), porém uma função que recebe um Animal irá funcionar igualmente bem em objetos da classe Cachorro e objetos da
classe Gato, já que os dois herdam da mesma classe Animal.

\paragraph{} Isso é polimorfismo e é uma das propriedades da programação orientada a objeto que dá mais expressividade para o paradigma.

\subsection{Diferentes tipos de OO}
\label{sec:org9e0b5be}
\paragraph{} Em C++, tipos são basicamente a mesma coisa que classe. Precisamos redefinir métodos explicitamente usando a palavra chave
\texttt{virtual} e não temos o conceito de interface, o mais próximos que podemos usar são classes abstratas puras.

\paragraph{} Já em Smalltalk, tipos são apenas conjuntos de métodos e como a verificação dos tipos é feita dinamicamente, temos
polimorfismo natural.

\paragraph{} Em Java, tipos é uma classe mais uma interface implementada por essa classe. O polimorfismo é garantido pelo pelo uso das
interfaces.

\subsection{Implementação}
\label{sec:org64ec4a0}
Para implementarmos POO, temos que ter algumas coisas em mente:

\begin{itemize}
\item Um objeto sempre armazena o conteúdo das suas variáveis de instância
\item O código dos métodos é compartilhado entre objetos de mesma classe, assim ocupamos menos espaço
\item Todos os métodos têm um parâmetro implícito que é o próprio objeto que está chamado esse método. Assim os métodos podem alterar o estado desse objeto.
\end{itemize}
\end{document}