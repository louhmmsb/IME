% Created 2020-12-26 Sat 12:25
% Intended LaTeX compiler: pdflatex
\documentclass[11pt]{article}
\usepackage[utf8]{inputenc}
\usepackage[T1]{fontenc}
\usepackage{graphicx}
\usepackage{grffile}
\usepackage{longtable}
\usepackage{wrapfig}
\usepackage{rotating}
\usepackage[normalem]{ulem}
\usepackage{amsmath}
\usepackage{textcomp}
\usepackage{amssymb}
\usepackage{capt-of}
\usepackage{hyperref}
\usepackage{minted}
\usepackage[hyperref, x11names]{xcolor}
\hypersetup{colorlinks = true, urlcolor = SteelBlue4, linkcolor = black}
\usepackage[brazilian]{babel}
\usepackage{geometry}
\geometry{verbose,a4paper,left=2cm,top=2cm,right=3cm,bottom=3cm}
\author{Lourenço Bogo}
\date{\today}
\title{Programação Funcional}
\hypersetup{
 pdfauthor={Lourenço Bogo},
 pdftitle={Programação Funcional},
 pdfkeywords={},
 pdfsubject={},
 pdfcreator={Emacs 27.1 (Org mode 9.5)}, 
 pdflang={Brazilian}}
\begin{document}

\maketitle
\tableofcontents

\newpage
\section{Introdução}
\label{sec:orgef3f1ba}

\paragraph{} Programação funcional foi o primeiro tópico que vimos na matéria.
\paragraph{} Trata-se de um paradigma onde as funções são valores, ou seja, podemos passá-las como argumentos e
atribuí-las a variáveis. Além disso, o paradigma possui outras propriedades relevantes que irei abordar em seguida.

\section{Propriedades Principais}
\label{sec:org8506400}
\paragraph{} Uma parte extremamente importante de linguagens funcionais é que, em geral, os procedimentos (funções) não têm efeitos colaterais:
uma função apenas recebe os seus argumentos, executa algum tipo de operação com eles e no final retorna um novo valor.
O procedimento não altera variáveis de outro escopo, inclusive os seus argumentos, que permanecerão inalterados no ambiente do escopo no
qual o procedimento foi chamado.

\paragraph{} As funções podem ser argumentos de outras funções, o que nos conduz a um ponto crucial do paradigma funcional:
as funções de ordem superior. Essas são funções cujos parâmetros são outras funções e, quando combinadas com certas
propriedades do paradigma, nos permitem atingir um nível de abstração muito interessante nas linguagens. Com base nesse conceito
podemos criar alguns dos procedimentos mais importantes e conhecidos da programação funcional, como o \texttt{map}, o \texttt{filter} e o \texttt{reduce}.

\section{Lambdas e Closures}
\label{sec:org83183b7}
\paragraph{} Sendo as nossas funções valores, não é necessário que tenham nomes e, por essa razão, são definidas de maneira anônima. Denominamos funções anônimas como lambdas. Se quisermos dar-lhes um nome, podemos simplesmente atribuí-las a alguma variável, já que
elas são valores comuns iguais a \texttt{1} ou \texttt{"foo"}.

\paragraph{} Outro conceito é o fechamento (closure), que é um par \texttt{(lambda, ambiente)}. Quando criamos uma nova função
usando \texttt{lambda}, montamos um fechamendo com a lambda previamente definida e o ambiente atual é estendido pela variável dessa lambda.
Isso é feito por dois motivos principais:
\begin{enumerate}
\item No momento de fazer a aplicação da função definida, vamos ao seu ambiente e alteramos o valor da variável para o valor passado como argumento na aplicação.
\item Como todo ambiente de fechamento é uma extensão do ambiente no qual esse fechamento foi criado, as funções mais internas poderão acessar os argumentos das funções externas.
\end{enumerate}

\paragraph{} Vale relembrar aqui a diferença entre escopo léxico e escopo dinâmico. O que usamos em nossos trabalhos foi o escopo léxico, onde uma
variável terá o valor que lhe foi dado localmente, pois expandimos os ambientes da maneira que foi explicada acima. Já no escopo dinâmico,
todo identificador terá associada uma pilha global de valores. Quando criamos uma variável cujo nome é esse identificador, inserimos o
valor atribuído à variável na pilha do identificador. Ao avaliarmos certa variável, o valor resultante será o que estiver no
topo da pilha. O desempilhamento é feito seguindo os escopos.

\section{Currying}
\label{sec:orgf69b16d}
\paragraph{} Outro ponto interessante abordado foi o conceito de \texttt{curry} que, de forma sintética e simplista, significa transformar funções
de múltiplos parâmetros em cadeias de funções com apenas um parâmetro. Exemplificando:
\begin{minted}[]{scheme}
;; Função definida sem currying
(define somaDois
  (lambda (x y)
    (+ x y)))

;; Função definida com currying
(define somaDoisCurry
  (lambda (x)
    (lambda (y)
      (+ x y))))
\end{minted}

\paragraph{} A primeira função só pode ser chamada quando passamos dois argumentos e ela somará os dois. A segunda função recebe um
argumento, devolve um fechamento, que também recebe um argumento, e só então soma os dois. Isso é útil, pois torna o código mais
reutilizável e sustentável. Por exemplo, se quisermos fazer um procedimento que recebe um argumento e acrescenta 1 a esse argumento,
poderíamos reutilizar \texttt{somaDoisCurry}, e definir a nova função mais facilmente:

\begin{minted}[]{scheme}
(define somaUm
  (somaDoisCurry 1)) ;; Retorna uma lambda de um argumento que soma 1 a esse argumento
\end{minted}

\section{Continuações}
\label{sec:org1fd27a0}
\paragraph{} Aprendemos também sobre continuações. Ao invés de termos uma função que recebe e retorna valores, adcionamos a essa função outro
parâmetro que será um procedimento. O argumento desse procedimento será o valor que a função estava retornando anteriormente. Ou seja:
\begin{itemize}
\item Fazíamos \(f(x) \rightarrow y\)
\item Passamos a fazer \(f(x, g) \rightarrow g(x)\)
\end{itemize}

\paragraph{} A grande utilidade disso é que nos permite controlar melhor o fluxo do programa e tratar mais facilmente situações não usuais, como por
exemplo, sair de recursões sem termos que voltar por todas as chamadas.
\end{document}