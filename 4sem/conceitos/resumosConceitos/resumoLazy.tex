% Created 2020-12-29 Tue 19:10
% Intended LaTeX compiler: pdflatex
\documentclass[11pt]{article}
\usepackage[utf8]{inputenc}
\usepackage[T1]{fontenc}
\usepackage{graphicx}
\usepackage{grffile}
\usepackage{longtable}
\usepackage{wrapfig}
\usepackage{rotating}
\usepackage[normalem]{ulem}
\usepackage{amsmath}
\usepackage{textcomp}
\usepackage{amssymb}
\usepackage{capt-of}
\usepackage{hyperref}
\usepackage{minted}
\usepackage[hyperref, x11names]{xcolor}
\hypersetup{colorlinks = true, urlcolor = SteelBlue4, linkcolor = black}
\usepackage[brazilian]{babel}
\usepackage{geometry}
\geometry{verbose,a4paper,left=2cm,top=2cm,right=3cm,bottom=3cm}
\author{Lourenço Bogo}
\date{\today}
\title{Resumo Avaliação por Demanda}
\hypersetup{
 pdfauthor={Lourenço Bogo},
 pdftitle={Resumo Avaliação por Demanda},
 pdfkeywords={},
 pdfsubject={},
 pdfcreator={Emacs 27.1 (Org mode 9.5)}, 
 pdflang={Brazilian}}
\begin{document}

\maketitle
\tableofcontents

\newpage
\section{Introdução}
\label{sec:org2f6528c}
\paragraph{} O segundo tópico abordado na matéria foi "Lazy Evaluation" (Avaliação por Demanda).

\paragraph{} Avaliação por demanda é uma estratégia de avaliação cujo princípio básico é: apenas calcular o valor de uma expressão quando
esse valor for necessário. É o contrário do que vínhamos aprendendo até agora, que era avaliação ansiosa, cujo princípio é avaliar uma
expressão na primeira vez que ela for encontrada.

\paragraph{} Para ilustrar a diferença entre esses dois métodos de avaliação, segue um exemplo simples:

\begin{minted}[]{scheme}
(cons (+ 1 2) '())
\end{minted}

\paragraph{} O código acima monta uma lista cujo primeiro elemento é a aplicação da função \texttt{+} nos elementos 1 e 2. Se estamos em uma linguagem onde temos
avaliação ansiosa, acontecerá o seguinte: ao encontrarmos a operação \texttt{(+ 1 2)}, iremos avaliá-la e teremos como resultado 3. Nosso código
então irá produzir uma lista com apenas 1 elemento, que será o número 3.

\paragraph{} Fica claro que não usamos o valor 3 para nada, ele não foi necessário para nenhuma operação. Para montarmos a lista, não precisávamos saber
que ao avaliarmos aquela expressão teríamos como resultado 3. Para isso que serve a avaliação por demanda. No código acima, se estivermos
em uma linguagem que implementa esse tipo de avaliação, não iremos calcular o valor da soma, iremos criar uma lista de 1 elemento, cujo
valor é uma \textbf{SUSPENSÃO}, que quando avaliada irá nos retornar o valor 3.

\section{Suspensões}
\label{sec:orgc1e7d55}
\paragraph{} Suspensões são uma estrutura semelhante a um fechamento sem argumentos, elas guardam uma expressão e um ambiente. Quando o
valor da suspensão for necessário, iremos interpretar a expressão dessa suspensão, com o ambiente contido nela.

\paragraph{} Por eficiência, sempre que avaliamos pela primeira vez uma suspensão específica, substituímos no ambiente global o seu valor
pelo valor retornado dessa avaliação. Assim, na próxima vez que precisarmos do valor dessa suspensão, poderemos utilizá-la sem ter que
recalcular o seu valor.

\paragraph{} Para todo esse sistema funcionar, necessitamos de um novo tipo de funções que serão chamadas \textbf{Funções Estritas}. Essas funções
irão avaliar seus argumentos imediatamente, ou seja, caso recebam uma suspensão como argumento, elas irão expandir essa suspensão.
Exemplificando:

\begin{minted}[]{scheme}
(if (equal? (+ 1 2) 3) (alguma_coisa) (outra_coisa))
\end{minted}

\paragraph{} Nesse caso, para podermos continuar o programa, é necessário que avaliemos o valor da condição do if imediatamente. Ou seja,
a condição do if é estrita, significando que ela nunca irá criar suspensões novas, e sempre irá avaliar as suspensões dadas.

\paragraph{} Alguns outros exemplos de funções estritas são:

\begin{itemize}
\item Operações aritméticas, já que precisamos saber em quais valores estamos aplicando essa operação (não faz sentido somar duas suspensões)
\item Car e cdr, já que quando queremos um elemento de uma lista, queremos o elemento em si e não uma suspensão
\end{itemize}

\section{Vantages e Desvantagens}
\label{sec:orge285622}
\paragraph{} Primeiro, a grande desvantagem da avaliação por demanda é a seguinte: já que uma suspensão guarda também um ambiente,
dependendo do jeito que implementarmos esse sistema, pode ser que o custo de memória fique muito alto e imprevisível. Além disso, funções
com efeitos colaterais podem quebrar as coisas, já que podemos alterar suspensões antes de termos usado seu valor para o que queríamos.

\paragraph{} Agora, as vantagens principais desse método de avaliação são:

\begin{itemize}
\item Aumento na performance da linguagem, já que iremos evitar avaliações desnecessárias
\item Podemos ter estruturas de dados infinitas, pois só iremos calcular os elementos necessários dessa estrutura. Exemplo em haskell:
\begin{minted}[]{haskell}
fibs = 0 : 1 : zipWith (+) fibs (tail fibs)
      -- Código retirado de wiki.haskell.org/The_Fibonacci_sequence
      -- Nesse exemplo montamos uma lista que contém TODOS os números da
      -- sequência de Fibonacci. O único problema com isso é que se pedirmos algo que
      -- exige a lista toda (como o tamanho da lista), o programa irá quebrar.
\end{minted}
\end{itemize}
\end{document}